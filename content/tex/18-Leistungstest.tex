%ju 26-Dez-22 18-Leistungstest.tex
ANTWORT: Vgl. handschriftliche Notizen (verifiziert)

\textbf{1. Nennen Sie die wichtigsten Additive im Dieselkraftstoff und
begründen Sie deren Notwendigkeit.}

\textbf{2. Erläutern Sie die Vorteile, die sich aus dem Einsatz
variabler Ventiltriebe ergeben}

\textbf{3. Nennen Sie verschiedene Möglichkeiten zur Drehung der Ventile
im Fahrbetrieb. Warum ist diese Drehung notwendig?}

\textbf{4. Beschreiben Sie Aufbau und Funktion der Registeraufladung}

\textbf{5. Welche Folgen ergeben sich aus der Umstellung vom Kammermotor
zum Direkteinspritzer?}

\textbf{6. Welche Spannung ist zum Öffnen eines Piezo-Injectors
notwendig? Erläutern Sie, warum diese Spannung benötigt wird.}

\textbf{7. Wozu dient das SCR-Abgasnachbehandlungsystem? Welcher Vorteil
ergibt sich aus dem Einsatz dieses Systems gegenüber anderer Systeme mit
der gleichen Aufgabe?}

\textbf{8. Was versteht man unter dem hydrodynamischen Druckkeil? Wie
entsteht dieser?}

\textbf{9. Wie groß sollte das Spiel einer mechanischen
Kupplungsbetätigung am Pedal und am Ausrücker sein?}

\textbf{10. Ihnen steht folgendes Datenblatt eines Mercedes C63 AMG
Black Series zur Verfügung:}

\begin{itemize}
\item
  Bohrung $102,2~mm$
\item
  Hub $94,6~mm$
\item
  Bauart V8
\item
  Ventile 4/Zyl.
\item
  Verdichtung $11,3 : 1$
\item
  Leistung 380 KW (517PS) bei $6800~min^{-1}$
\end{itemize}

\begin{enumerate}
\item
  \textbf{Berechnen Sie die Größe des Gesamthubraums}
\item
  \textbf{Berechnen Sie die mittlere Kolbengeschwindigkeit bei maximale
  Leistung}
\item
  \textbf{Handelt sich bei diesem Motor um einen Kurz-, Lang- oder
  Quadrathuber?}
\item
  \textbf{Ausgehend von einem atmosphärischen Luftdruck von 1 bar und
  einem theoretischen Liefergrad von 1: Welcher Druck würde sich am Ende
  der Verdichtung im Verdichtungsraum ungefähr einstellen, wenn
  Druckverluste unberücksichtigt bleiben?}
\end{enumerate}
