%ju 26-Dez-22 16-Direktschaltgetriebe.tex
\textbf{Woraus besteht das Direktschaltgetriebe?}

\begin{itemize}
\item
  Es besteht im Prinzip aus 2 Teilgetrieben. Jedes Teilgetriebe ist in
  seiner Wirkung wie ein Handschaltgetriebe aufgebaut. Jedem
  Teilgetriebe ist eine Kupplung zugeordnet. Die beiden Eingangswellen
  sind als hohle Außen- und volle Innenwelle ausgeführt. Alle
  Übersetzungsstufen sind, wie bei normalen Schaltgetrieben, als
  Zahnradpaare auf Eingangs- und Nebenwellen untergebracht.
\end{itemize}

\textbf{Wie sind die 6 Gänge im Direktschaltgetriebe angeordnet?}

\begin{itemize}
\item
  Auf der Innenwelle befinden sich die ungeraden Gänge 1, 3 und 5, sowie
  der Rückwärtsgang. Auf der Außenwelle sind die geraden Gänge 2, 4 und
  6 angeordnet.
\end{itemize}

\textbf{Wie werden beim Direktschaltgetriebe die Gänge geschaltet?}

\begin{itemize}
\item
  Die Schaltaktoren verschieben mit Hilfe der Schaltgabeln die
  Schaltmuffen. Dabei sorgen sie für die Synchronisierung, legen die
  Gänge ein und nehmen sie wieder heraus.
\end{itemize}

\textbf{Wie wird die Position der Schaltgabeln im Direktschaltgetriebe
erfasst?}

\begin{itemize}
\item
  Die Position der Schaltgabeln wird mit integrierten Magneten ständig
  über Hallsensoren erfasst.
\end{itemize}

\textbf{Wie arbeiten die beiden Lamellenkupplungen während eines
Schaltvorgangs, z.B. unter Last vom 1. in den 2. Gang?}

\begin{itemize}
\item
  Sobald die, in der Software programmierte Schaltdrehzahl erreicht ist,
  beginnt der Schaltvorgang. Die Hydraulik drückt die Lamellen der
  Kupplung K2 gegen die Federkraft zusammen. Die Kupplung K2 beginnt
  Kraft zu übertragen, dadurch sinkt geringfügig die Motordrehzahl.
  Gleichzeitig wird der Anpressdruck der Kupplung K1 verringert, während
  im gleichen Maße der Druck in der Kupplung K2 erhöht wird. Zur
  Unterstützung der Drehzahlanpassung und zur Erhöhung der
  Schaltgeschwindigkeit wird in der Überschneidungsphase der beiden
  Kupplungen kurzzeitig das Motordrehmoment verringert. Der
  Schaltvorgang ist abgeschlossen, wenn die Kupplung K2 das gesamte
  Drehmoment im 2. Gang überträgt und die Kupplung K1 vollständig
  geöffnet ist.
\end{itemize}
