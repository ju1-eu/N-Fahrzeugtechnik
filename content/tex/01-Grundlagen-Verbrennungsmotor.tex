%ju 17-Sep-22 01-Grundlagen-Verbrennungsmotor.tex
\section{Motorbauformen}\label{motorbauformen}

\begin{itemize}
\item
  Reihenmotor, V-Motor, VR-Motor, Boxermotor
\item
  Hubkolbenmotor, Kreiskolbenmotor
\end{itemize}

\section{Hubraum - Brennraum -
Verdichtungsraum}\label{hubraum-brennraum-verdichtungsraum}

$\text{Hubraum} \, V_h = \frac{\pi \cdot d^2}{4} \cdot s \,[cm^3]$

$\text{Gesamthubraum} \, V_H = V_h \cdot z$

$\text{Brennraum} = V_h + V_c$

$\text{Verdichtungsraum} \, V_c = \frac{V_h}{\epsilon - 1}$

$\text{Verdichtungsverhältnis} \, \epsilon = \frac{V_h + V_c}{V_c}$

\section{Arbeitsweise}\label{arbeitsweise}

\begin{enumerate}
\def\labelenumi{(\arabic{enumi})}
\item
  Vier-Takt-Arbeitsverfahren (1 Arbeitsspiel = 2 Kurbelwellenumdrehungen
  = 4 Kolbenhübe)
\item
  Zwei-Takt-Arbeitsverfahren (1 Arbeitsspiel = 1 Kurbelwellenumdrehung =
  2 Kolbenhübe)
\end{enumerate}

\section{Ansaugen}\label{ansaugen}

\begin{itemize}
\item
  Abwärtsgehen des Kolbens
\item
  Volumenvergrößerung, Druckdifferenz (Zylinderdruck versus höhere
  Außendruck)
\item
  Einströmen der Luft (Ansaugen der Luft)
\item
  zündfähiges Kraftstoff-Luft-Gemisch innen/außen bilden (Zylinder,
  Ansaugrohr)
\item
  Kraftstoff (Kohlenstoff -- Wasserstoff - Verbindung)
\end{itemize}

\subsection{Luftdruck}\label{luftdruck}

Luftdruck in Abhängigkeit der geodätischen Höhe

Luftdruck bezogen auf Meereshöhe $(1013~hPa = 1013~mbar, ca.\,1~bar)$

Erdanziehung ist von der Masse abhängig

Das Gewicht der Umgebungsluft drückt auf die Erdoberfläche und erzeugt
einen Druck, Atmosphärendruck.

\subsection{Absolutdruck}\label{absolutdruck}

Druck gegenüber Null (Vakuum, luftleeren Raum)

\subsection{Relativer Druck}\label{relativer-druck}

Druck messen gegenüber Absolutdruck

\subsection{Zusammensetzung der Luft
(Prüfung)}\label{zusammensetzung-der-luft-pruefung}

\begin{itemize}
\item
  $78~\%$ Stickstoff
\item
  $21~\%$ Sauerstoff
\item
  $0,9~\%$ Edelgase
\item
  $0,1~\%$ Partikel, Feinstaub (Zell Gängigkeit, Blutkreislauf,
  Erbgutveränderung > Mutation)
\item
  $0,040~\% ~ CO_2$
\end{itemize}

\section{Verdichten}\label{verdichten}

\begin{itemize}
\item
  Aufwärtsgehen des Kolbens
\item
  Kraftstoff-Luft-Gemisch wird verdichtet
\end{itemize}

\subsection{Hoch verdichtete Motoren}\label{hoch-verdichtete-motoren}

$10:1$

\subsection{Verdichtungsverhältnis}\label{verdichtungsverhaeltnis}

\begin{itemize}
\item
  \textbf{Mazda Motor:} $14:1$
\item
  \textbf{Turbo Motor:} $7 - 8:1$
\item
  \textbf{Direkt:} $17 - 18:1$
\item
  \textbf{Indirekt} (Wirbel, Vorkammer): $21 - 36:1$

  \begin{itemize}
  \item
    Vorkammer > Kugel > Wärmeabgabe
    > höher Verdichten
  \end{itemize}
\end{itemize}

\subsection{Wärme}\label{waerme}

entsteht durch Reibung, Form von Energie, Bewegungsenergie kleiner
Teilchen

\subsection{Wärmeabführung}\label{waermeabfuehrung}

an der Oberfläche, Luft (Isolator)

\subsection{Verdichten z.~B. 10:1}\label{verdichten-z.-b.-101}

$10~l$ großes Volumen wird auf den zehnten Teil verkleinert, also
$1~l$

maximales Volumen zu minimales Volumen

Vgl. >>\emph{Kapitel Rechenbeispiele / Motor - Hubraum - Verdichtung}<<

\subsection{Verdichtungsendtemperatur}\label{verdichtungsendtemperatur}

$600 - 900^\circ\text{C}$ (Diesel: Zündung wird eingeleitet)

\subsection{Entzündungstemperatur
Diesel}\label{entzuendungstemperatur-diesel}

$230 - 250^\circ\text{C}$

\subsection{Zündverzug}\label{zuendverzug}

$Entflammungsphase: \frac{1}{1000} s$

Ziel: Vollständige Verbrennung (feine Zerstäubung und hohe Temperaturen,
das muss schnell gehen)

\begin{enumerate}
\def\labelenumi{(\arabic{enumi})}
\item
  \textbf{Benzin} Beginn Zündfunken bis zur Verbrennung
\item
  \textbf{Diesel} Beginn des Einspritzens bis zur Verbrennung
\end{enumerate}

\subsection{Thermodynamischer Kreisprozess (keine
Prüfung)}\label{thermodynamischer-kreisprozess-keine-pruefung}

Wärmekraftmaschine: Verhältnis von Druck, Volumen und Temperatur beim
Verdichten

Verhindert man die Ausdehnung, z.~B. beim Verdichten, so verdoppelt sich
der Druck.

Pro Grad der Erwärmung steigt der Druck um 273sten Teil seines Volumens
im geschlossenen System

Erwärmt man das Gas um $273~K$, so dehnt es sich auf das doppelte
Volumen aus. Die Temperatur steigt und damit der Druck.

Grundlage: \textbf{1. Satz der Thermodynamik} >>Die Energie bleibt in
einem geschlossenen System konstant.<<

\textbf{Faktor 2} $\Rightarrow \frac{546}{273}$

Ziel: von $20^\circ\text{C} \Rightarrow 600^\circ\text{C}$
Verdichtungsendtemperatur

z.~B. Verdichtung: $(10:1)$ $1~bar \Rightarrow 20~bar$
Verdichtungsenddruck $(10~bar \cdot 2 = 20~bar)$

Vgl. >>\emph{Kapitel Rechenbeispiele / Druck - Volumen - Temperatur}<<

\textbf{Diesel} Gleichdruckverbrennung \textbf{Benzin}
Gleichraumverbrennung \textbf{Anwendung} Zylinderabschaltung, Studium

\subsection{Grad Celsius}\label{grad-celsius}

als Fixpunkte werden die Temperaturen vom Gefrier- $(0^\circ\text{C})$
und Siedepunkt $(100^\circ\text{C})$ des Wassers verwendet

\subsection{Aggregatzustand}\label{aggregatzustand}

physikalischer Zustand eines Stoffs. Beispiel: fest, flüssig, gasförmig,
plasma

\subsection{Kelvin}\label{kelvin}

absoluten Nullpunkt der Teilchen $(0K = -273^\circ\text{C})$

$0~K = -273,15^\circ\text{C}$

$273,15~K = 0^\circ\text{C}$

$373,15~K = 100^\circ\text{C}$

\textbf{Umrechnung}

$Kelvin = T_{Grad~Celsius} + 273,15$

$Grad~Celsius = T_{Kelvin} - 273,15$

\section{Arbeiten}\label{arbeiten}

\begin{itemize}
\item
  Verbrennung wird durch den Zündfunken eingeleitet
\item
  Die Expansion der Gase treibt den Kolben nach UT
\item
  Wärmeenergie wird in mechanische Energie umgewandelt
\end{itemize}

\subsection{Verbrennungstemperatur}\label{verbrennungstemperatur}

$2000 - 2500^\circ\text{C}$

\subsection{Kolbenmaterial}\label{kolbenmaterial}

Aluminium-Silizium-Legierung

\subsection{Thermische Belastung
Kolben}\label{thermische-belastung-kolben}

\textbf{Problem} Kolbenbodentemperaturen

\begin{enumerate}
\def\labelenumi{(\arabic{enumi})}
\item
  $390^\circ\text{C}$ Diesel
\item
  $290^\circ\text{C}$ Benziner
\item
  $421^\circ\text{C}$ Stahlkolben
\end{enumerate}

\textbf{Kerbe} = Sollbruchstelle

\subsection{Kolbenfläche berechnen}\label{kolbenflaeche-berechnen}

$A = \frac{\pi \cdot d^2}{4} = [mm^2]$

Vgl. >>\emph{Kapitel Rechenbeispiele / Druckberechnung am Pleuellager}<<

\subsection{Kolbenwärme abführen}\label{kolbenwaerme-abfuehren}

\begin{itemize}
\item
  Kolbenboden
\item
  $80~\%$ über 1. Kolbenring, Kompressionsring, Minutenring,
  (trapezförmig)
\item
  Zylinderwand
\end{itemize}

\subsection{Kolbenkippen Ursache}\label{kolbenkippen-ursache}

\begin{itemize}
\item
  Druckverlagerung
\item
  Desachsierung des Kolbens
\item
  Wann liegt der Kolbendruck an?
\item
  Wann Übergehen der Kolbengleitbahnen (deshalb die Desachsierung des
  Kolbens)
\item
  Welche Kolbenbauform?
\item
  ausreichend langes Kolbenhemd $\to$ Problem mehr Masse
\item
  Masse einmal beschleunigt, wenn Kolben nach unten (möchte durch die
  Ölwanne in den Boden) oder Kolben nach oben (durch die Motorhaube in
  den Himmel) davon hält das Pleuel ab
\end{itemize}

\section{Ausstoßen}\label{ausstossen}

\begin{itemize}
\item
  Abgase verlassen mit Überschallgeschwindigkeit den Zylinder
\end{itemize}

\subsection{Abgastemperatur}\label{abgastemperatur}

$600 - 900^\circ\text{C}$

\subsection{Wirkungsgrad (Effizienz)}\label{wirkungsgrad-effizienz}

\begin{enumerate}
\item
  Dieselmotoren ca. $46~\%$ und
\item
  Ottomotoren ca. $35~\% \to$ werden in \textbf{Bewegungsenergie} als
  Antriebsenergie für Motor verwendet
\end{enumerate}

Rest in \textbf{Reibung und Wärme}

\subsection{Wie bewegt sich die
Kurbelwelle?}\label{wie-bewegt-sich-die-kurbelwelle}

\textbf{Ungleichförmige Drehbewegung}

Hubkolbenbewegung Reihenmotor (abhängig Zylinderanzahl, Zündreihenfolge)

Kompensieren: Kurbelwellenrad exzentrisch ausgeführt (unterschiedliche
Hebellängen)

Beschleunigt (Zündungstakt)

\begin{itemize}
\item
  alle zwei NW Umdrehungen
\item
  alle vier KW Umdrehungen
\end{itemize}

\subsection{Hydrodynamischer
Schmierkeil}\label{hydrodynamischer-schmierkeil}

\begin{itemize}
\item
  durch ungleichförmige Drehbewegung
\item
  wird eine Volumenvergrößerung zwischen Kurbelwelle und Lagerung
  erreicht
\item
  dadurch ein Einströmen des Lageröls in die Lagerstelle begünstigt
\item
  Und wenn jetzt der Verbrennungsdruck durch die Verbrennung erzeugt auf
  die Kurbelwellenlagerstelle schneller erfolgt, als die Verdrängung des
  Öls aus der Lagerstelle heraus
\item
  Aufschwimmen auf unserem Lager
\item
  Wir bewegen uns auf dem hydrodynamischen Schmierkeil
\item
  Ist in der Lage hohe Drücke auszuhalten
\end{itemize}

Wie kann es sein, dass ich mit einem Versorgungsdruck von $5~bar$
einen Öldruck in den Lagern der Kurbelwelle einen Spitzendruck bis
$1000~bar$ kompensieren kann > Hydrodynamischer
Schmierkeil

Vgl. >>\emph{Kapitel Rechenbeispiele / Druckberechnung am Pleuellager}<<

\section{Kurbelgehäusearten}\label{kurbelgehaeusearten}

\begin{enumerate}
\item
  Closed-Deck-Ausführung

  \begin{itemize}
  \item
    Dichtfläche bis auf Kühl- oder Ölkanäle geschlossen gegenüber
    Zylinderkopf
  \item
    Niederdruckguss-Verfahren (AlSi-Legierung)
  \end{itemize}
\item
  Open-Deck-Ausführung

  \begin{itemize}
  \item
    Wassermantel um die Zylinderbohrungen offen gegenüber Zylinderkopf
    (geringe Steifigkeit, erfordert Metall-Zylinderkopfdichtung)
  \item
    Druckgussverfahren
  \end{itemize}
\end{enumerate}

\section{Zylinderkopfdichtung}\label{zylinderkopfdichtung}

Gasabdichtung bei allen Betriebszuständen

\begin{enumerate}
\item
  Metall-Weichstoff-Zylinderkopfdichtung
\item
  Metall-Zylinderkopfdichtung
\end{enumerate}
