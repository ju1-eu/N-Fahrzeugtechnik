%ju 26-Dez-22 14-Antrieb.tex
\textbf{Wie ist eine Viscokupplung aufgebaut?}

\begin{itemize}
\item
  Ein zylindrisches, mit dem Antrieb verbundenes Gehäuse
\item
  Gelochte Außenlamellen, drehfest mit dem Gehäuse verbunden
\item
  Innenlamellen, drehfest mit der Abtriebswelle verbunden
\item
  Füllung mit hochviskoser Silikonflüssigkeit.
\end{itemize}

\textbf{Warum sind bei einer Gelenkwelle 2 Kreuzgelenke erforderlich?}

Kreuzgelenke sind nicht homokinetisch. Nach dem ersten gebeugten
Kreuzgelenk ist aus der ursprünglich gleichförmigen Bewegung der Welle
eine ungleichförmige Bewegung geworden.

Um diesen Effekt wieder auszugleichen benötigt man ein zweites
Kreuzgelenk, dessen Gelenkgabel in derselben Ebene, wie die des ersten
liegen muss.

\textbf{Was sind homokinetische Gelenke? Nennen Sie mindestens 3
homokinetische Gelenke mit Ihren spezifischen Eigenschaften.}

Homokinetische Gelenke sind Gleichlaufgelenke. Durch ihren Einsatz kommt
es zu keinerlei Veränderungen in der radialen Bewegung der Welle.

\begin{enumerate}
\item
  \textbf{Kugelgelenk} Beugewinkel bis zu 38° (47°), keine axiale
  Verschiebung möglich
\item
  \textbf{Topfgelenk} Beugewinkel bis zu 22°, axiale Verschiebung um
  max. 45 mm
\item
  \textbf{Tripodegelenk} Beugewinkel bis zu 26°, axiale Verschiebung um
  max. 55 mm
\item
  \textbf{Doppelgelenk} Beugewinkel bis zu 50°, keine axiale
  Verschiebung möglich
\item
  \textbf{Gewebescheibengelenk} (Hardyscheibe) Beugewinkel bis zu 5°,
  axiale Verschiebung um max. 1,5 mm
\end{enumerate}

\textbf{Welcher Unterschied besteht zwischen}

\textbf{a) dem normalen Kegelradantrieb?} \textbf{b) dem Hypoidantrieb?}

\begin{enumerate}
\item
  Beim normalen Kegelradantrieb liegen die Achsen von Teller und
  Kegelrad in einer Ebene.
\item
  Beim Hypoidantrieb ist die Mittellinie des Kegelrades gegenüber der
  Mittellinie des Tellerrades nach unten versetzt.

  \begin{itemize}
  \item
    größere Kraftübertragung, da mehr Zähne im Eingriff
  \item
    Gelenkwelle kann nach unten versetzt werden

    \begin{itemize}
    \item
      $\to$ niedrigerer Schwerpunkt des Fahrzeugs
    \item
      $\to$ flacherer Gelenkwellentunnel
    \end{itemize}
  \end{itemize}
\end{enumerate}

\textbf{Aus welchem Grund ist beim Kraftfahrzeug ein Ausgleichsgetriebe
erforderlich?}

Das Ausgleichsgetriebe ist notwendig, um die Wegunterschiede
auszugleichen, die bei Kurvenfahrt zwischen dem kurveninneren Rad
(kleiner Radius und damit Umfang) und dem kurvenäußeren Rad (großer
Radius und damit Umfang) liegen.

Würde dies nicht geschehen,

\begin{itemize}
\item
  würde das Fahrzeug stark untersteuern
\item
  würden die Räder Schlupf aufweisen und entsprechende Geräusche
  entstehen
\item
  würden Bauteile durch die starken Torsionskräfte zerstört
\end{itemize}

\textbf{Wie ist die Wirkungsweise des Ausgleichsgetriebes bei
Geradeausfahrt?}

Das Tellerrad mit angeflanschtem Ausgleichsgehäuse wird vom Kegelrad
angetrieben. Vom drehenden Ausgleichsgehäuse werden der fest angeordnete
Differentialbolzen bzw. das Differentialkreuz und die, auf ihm
gelagerten Ausgleichskegelräder mitgenommen. Die Ausgleichskegelräder,
die mit den Achswellenkegelrädern im Eingriff sind, treiben diese an,
ohne sich um ihre eigene Achse zu drehen.

\textbf{Wie verhält sich das Ausgleichsgetriebe beim Anfahren mit
Schlupf?}

Das übertragene Antriebsmoment ist an beiden Rädern grundsätzlich gleich
groß. Hat also eines der Räder Schlupf, verringert sich das, an dem
anderen Rad übertragene Antriebsmoment. Liegt der Schlupf bei 100 \%,
das Rad dreht also durch, fällt der Vortrieb des Fahrzeuges weg, da das
andere Rad stehen bleibt.

\textbf{Wie ist die Wirkungsweise eines Selbstsperrdifferentials?}

Das Ausgleichsgehäuse besteht in diesem Differential aus zwei Teilen,
zwischen denen das Differentialkreuz gelagert ist. Kommt es im
Ausgleichsgetriebe zu Drehzahldifferenzen, so drücken die, durch das
Drehen der Ausgleichskegelräder entstehenden Spreizkräfte das
Ausgleichsgehäuse auseinander und pressen damit eine, zwischen den
Gehäusehälften und dem umfassenden Käfig gelagerte Lamellenkupplung
zusammen, die die Antriebswellen mit dem Gehäuse verbindet. Die Sperrung
des Differential ist proportional zur Drehzahldifferenz, wodurch der
maximale Sperrwert bei etwa 70 \% liegt.
