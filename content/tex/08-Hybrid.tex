%ju 17-Sep-22 08-Hybrid.tex
\textbf{Was ist ein Hybridfahrzeug? Hybrid-Vehicle (HV)} ist mit
mindestens zwei unterschiedlichen Energiewandlern und -speichern für den
Antrieb des Fahrzeuges ausgestattet.

Beispiel: Verbrennungsmotor und Kraftstofftank \& E-Motor und
HV-Batterie

\section{Arten von Hybriden}\label{arten-von-hybriden}

\begin{enumerate}
\item
  \textbf{Micro-Hybrid:} Regeneratives Bremsen, Start-Stopp
\item
  \textbf{Mild-Hybrid:} Regeneratives Bremsen, Start-Stopp,
  Drehmomentunterstützung
\item
  \textbf{Full-Hybrid:} Regeneratives Bremsen, Start-Stopp,
  Drehmomentunterstützung, Elektrisches Fahren
\item
  \textbf{Plug-in-Hybrid:} Regeneratives Bremsen, Start-Stopp,
  Drehmomentunterstützung, Elektrisches Fahren, Steckdosenaufladung
\end{enumerate}

\textbf{Boosten} Drehmomentunterstützung (Verbrennungsmotor
unterstützen)

\textbf{Rekuperation} Regeneratives Bremsen (Bremsenergierückgewinnung
$\to$ Batterie Laden)

\section{Antriebskonzepte}\label{antriebskonzepte}

Leistungsverzweigter Hybrid S. 52 (\textcite{schmidt:2021:hybrid}) und
Antriebskonzepte S. 1000 (\textcite{reif:2022:boschkraftfahrtechnisches}).

E-Maschine = \textbf{MG1 / MG2} (E-Motor / Generator), \textbf{V-Motor}
(Verbrennungsmotor), \textbf{Planetengetriebe} (Planetenradsätze und
Lamellenkupplung), Getriebe, Achsantrieb, Inverter (Pulswechselrichter)

\begin{enumerate}
\item
  \textbf{Serieller Hybrid}

  \begin{itemize}
  \item
    Antrieb: durch E-Motor
  \item
    Verbrennungsmotor wird im Drehzahlfenster betrieben (mechanisch
    nicht mit Antriebstrang verbunden)
  \item
    Serieller Energiefluss

    \begin{enumerate}
    \def\labelenumii{\arabic{enumii}.}
    \item
      V-Motor $\to$ Generator $\to$ Inverter $\leftrightarrow$
      HV-Batterie
    \item
      Inverter $\leftrightarrow$ E-Motor $\leftrightarrow$
      Achsantrieb
    \end{enumerate}
  \item
    Nachteil: schlechter Wirkungsgrad ($3\text{x}$ mehr Leistung
    reinstecken!)
  \item
    Beispiel: Range-Extender (Reichweitenverlängerung)
  \end{itemize}
\item
  \textbf{Paralleler Hybrid}

  \begin{itemize}
  \item
    Antrieb: durch E-Motor oder V-Motor (mechanisch mit Antriebstrang
    verbunden) oder beide parallel
  \item
    \emph{eine Kupplung zwischen E-Maschine $\parallel$ Getriebe -
    Achsantrieb}
  \item
    Energiefluss:

    \begin{enumerate}
    \def\labelenumii{\arabic{enumii}.}
    \item
      V-Motor $\to$ Achsantrieb
    \item
      Batterie $\leftrightarrow$ Inverter $\leftrightarrow$ MG
      $\leftrightarrow$ Achsantrieb
    \end{enumerate}
  \item
    \emph{zwei Kupplungen zwischen V-Motor $\parallel$ E-Maschine
    $\parallel$ Getriebe - Achsantrieb}
  \item
    Energiefluss:

    \begin{enumerate}
    \def\labelenumii{\arabic{enumii}.}
    \item
      V-Motor $\to$ Achsantrieb
    \item
      Batterie $\leftrightarrow$ Inverter $\leftrightarrow$ MG
      $\leftrightarrow$ Achsantrieb
    \end{enumerate}
  \end{itemize}
\item
  \textbf{Leistungsverzweigter Hybrid}

  \begin{itemize}
  \item
    Antrieb: durch zwei E-Motoren und V-Motor (mechanisch mit
    Antriebstrang verbunden)
  \item
    Nachteil: aufwändiges Getriebe
  \item
    Leistungsverzweigung (Planetengetriebe)
  \item
    V-Motor (Drehmoment aufteilen)

    \begin{enumerate}
    \def\labelenumii{\arabic{enumii}.}
    \item
      Teil $\to$ MG1 als Generator
    \item
      Teil $\to$ Fahrzeugantrieb
    \end{enumerate}
  \item
    V-Motor + E-Motor $\to$ Fahrzeugantrieb
  \item
    Schub- u. Bremsphase: Generator $\to$ Rekuperation
  \end{itemize}
\end{enumerate}
