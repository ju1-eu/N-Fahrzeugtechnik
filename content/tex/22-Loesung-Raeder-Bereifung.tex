%ju 31-Dez-22 22-Loesung-Raeder-Bereifung.tex
\textbf{1. Was sind Tiefbettfelgen?}

\begin{itemize}
\item
  Tiefbettfelgen sind einteilige Felgen, bei denen das Bett zum
  Radmittelpunkt hin vertieft ist, um die Montage des Reifens zu
  ermöglichen.
\end{itemize}

\textbf{2. Was bedeutet die Räderbezeichnung >>6J x 14 H2 ET49<<}

\begin{itemize}
\item
  \textbf{6} Maulweite in Zoll
\item
  \textbf{J} Kennbuchstabe für die Abmessungen des Felgenhorns
\item
  \textbf{x} Tiefbettfelgen
\item
  \textbf{14} Felgendurchmesser in Zoll
\item
  \textbf{H2} Doppel--Hump--Felge
\item
  \textbf{ET49} Einpresstiefe in mm
\end{itemize}

\textbf{3. Beschreiben Sie den Unterschied zwischen Diagonal- und
Radialreifen.}

\begin{itemize}
\item
  Beim Diagonalreifen verlaufen die Karkassfäden in einem Fadenwinkel
  von 26 -- 40° diagonal zur Fahrtrichtung.
\item
  Bei Radialreifen verlaufen die Karkassfäden in einem Fadenwinkel von
  90° zur Fahrtrichtung
\end{itemize}

\textbf{4. Erklären Sie folgende Bezeichnung eines Nfz-Reifens: >>315/80
R 22,5 154/149 M<<}

\begin{itemize}
\item
  \textbf{315} Reifenbreite (315 mm)
\item
  \textbf{80} Verhältnis Reifenhöhe zur Reifenbreite (80 \%)
\item
  \textbf{R} Reifenbauart: Radial-- oder Gürtelreifen
\item
  \textbf{22,5} Felgendurchmesser in Zoll
\item
  \textbf{154/149} Tragfähigkeitsindex für Einzel--/Zwillingsbereifung
\item
  \textbf{M} zugelassen für Geschwindigkeiten bis 130 km/h
\end{itemize}

\textbf{5. Was versteht man unter dem dynamischen Halbmesser?}

\begin{itemize}
\item
  Der dynamische Halbmesser ist der Abstand zwischen Fahrbahn und
  Radmitte, wenn der, nach Herstellervorgabe befüllte Reifen, mit der
  zulässigen Traglast belastet, mit einer Radumfangsgeschwindigkeit von
  60 km/h abrollt. Er ist kleiner, als der rechnerische und größer als
  der statische Halbmesser.
\end{itemize}

\textbf{6. Bei einem Pkw müssen 2 neue Reifen aufgezogen werden.}

\begin{itemize}
\item
  \textbf{a) Auf welcher Achse müssen diese Reifen montiert werden?}
\item
  \textbf{b) Mit welcher Begründung?}
\end{itemize}

\begin{enumerate}
\def\labelenumi{\alph{enumi})}
\item
  Die neuen Reifen sind, unabhängig von der Lage des Antriebs auf der
  Hinterachse zu montieren.
\item
  Die Hinterachse ist am Kfz die spurführende Achse. Kommt es hier zu
  unterschiedlichen Haftungsverhältnissen oder gar zu einem Totalausfall
  des Reifens, kann dies zum Verlust der Fahrstabilität führen. An der
  Vorderachse könnte ein solcher Schaden durch Gegenlenken ausgeglichen
  werden. Stark verminderte Haftungsverhältnisse der Hinterachse
  gegenüber der Vorderachse können zudem bei Bremsungen zu einer
  erhöhten Gierneigung des Kfz führen.
\end{enumerate}

\textbf{7. Was bedeutet der Schriftzug >>REGROOVABLE<< auf Nfz-Reifen?}

\begin{itemize}
\item
  Der Schriftzug >>REGROOVABLE<< kennzeichnet Nfz-Reifen, deren Profil
  nachgeschnitten werden kann. Sie haben einen Nachschneidindikator mit
  einer Nachschneidetiefe von 4 mm.
\end{itemize}

\textbf{8. Wie unterscheidet man Unwuchten am Rad und wie machen sich
diese bemerkbar?}

\begin{itemize}
\item
  statische Unwucht, die sich durch vertikale Schwingungen des Rades
  bemerkbar macht und die
\item
  dynamische Unwucht, deren Auswirkungen horizontale Schwingungen sind.
\end{itemize}

\textbf{9. Welche Prüfungen müssen vor dem Auswuchten eines Rades am
Fahrzeug durchgeführt werden?}

\begin{itemize}
\item
  Reifendruck
\item
  Radaufhängung
\item
  Sauberkeit von Rad und Reifen
\item
  Hören-- und Seitenschlag
\item
  Reifenschäden (Flachstellen)
\end{itemize}

\textbf{10. Was versteht man unter dem >>Matchen<< eines Reifens und wie
wird es durchgeführt?}

\begin{itemize}
\item
  Unter >>Matchen<< versteht man den Versuch, den Höhenschlag der Felge
  und den des Reifens gegeneinander aufzuheben. Beim Matchen wird der
  Reifen auf der Felge so gedreht, dass dadurch ein Rundlauf mit
  ausreichender Genauigkeit, d.h. mit einem minimalen Höhenschlag
  hergestellt wird.
\end{itemize}

\textbf{11. Beschreiben Sie die folgenden Reifen-Notlaufsysteme. Was ist
bezüglich der Räder bei den einzelnen Systemen zu beachten?}

\begin{itemize}
\item
  \textbf{a) Den Conti-Support-Ring}
\item
  \textbf{b) Self-Supporting-Tyres}
\item
  \textbf{c) Das PAX-System}
\end{itemize}

\begin{enumerate}
\def\labelenumi{\alph{enumi})}
\item
  Kommt zur Abstützung des Reifens bei Luftverlust ein metallischer
  Stützring zum Einsatz. Eine spezielle Felge ist nicht erforderlich.
\item
  Stützen sich bei Luftverlust auf ihrer verstärkten Flanke ab. Eine
  spezielle Felge ist nicht erforderlich.
\item
  Kommen spezielle Felgen und spezielle Reifen zum Einsatz. Im
  Pannenfall trägt eine, auf der Felge aufgebrachte Elastomereinlage das
  Fahrzeuggewicht.
\end{enumerate}

\textbf{12. Welche gesetzlichen Anforderungen werden an Fahrzeuge
gestellt, die mit einem Reifen-Notlaufsystem ausgestattet werden sollen?
Welche weiteren Hinweise sollten Sie dem Kunden zur Beachtung mitgeben?}

\begin{itemize}
\item
  Für Fahrzeuge mit Reifen--Notlaufsystemen ist gesetzlich eine
  Reifenpannenanzeige vorgeschrieben, die den Fahrer im Pannenfall auf
  den Schaden hinweisen soll.
\item
  Für die Weiterfahrt mit einem defekten Reifen gilt eine zulässige
  Höchstgeschwindigkeit von 80 km/h. Die maximale Fahrstrecke ist
  belastungsabhängig und liegt zwischen 80 und 250 km.
\end{itemize}
