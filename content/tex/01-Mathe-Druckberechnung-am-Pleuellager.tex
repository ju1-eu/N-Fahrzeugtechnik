%ju 17-Sep-22 01-Mathe-Druckberechnung-am-Pleuellager.tex
\textbf{Lösungshinweise zur Aufgabe 1}

\textbf{Kolbenflächenberechnung:}
$\boxed{A = \frac{d^2}{4} \cdot \pi}$

Kolbendurchmesser $d = 80~mm = 8~cm$

$A_{Kolben} = \frac{(80~mm)^2}{4} \cdot \pi = 5026,55~mm^2 = 50,27~cm^2$

\textbf{Kolbenkraftberechnung:}
$\boxed{\text{Druck} = \frac{\text{Kraft}}{\text{Fläche}}}$

$\boxed{p = \frac{F}{A}} \quad \boxed{p\,[N/cm^2] \quad F\,[N] \quad A\,[cm^2]} \quad \boxed{10~N/cm^2 = 1~bar}$

\textbf{Verbrennungsdrücke:}

$\text{Benzin} \to 65~bar = 650~N/cm^2 \quad \text{Diesel} \to 180~bar = 1800~N/cm^2$

$F = p \cdot A \\ F_{Kolben_B} = 50,27~cm^2 \cdot 650~N/cm^2  = 32675,5~N \\ F_{Kolben_D} = 50,27~cm^2 \cdot 1800~N/cm^2  = 90486~N$

\textbf{Kreisbogenberechnung:}
$\boxed{A = \frac{d \cdot \pi}{2} \cdot b}$

$d_{Kurbelwelle} = 60~mm = 6~cm \\ d_{Lager} = 25~mm = 2,5~cm$

$A_{Krb} = \frac{6~cm \cdot \pi}{2} \cdot 2,5~cm  = 23,56~cm^2$

\textbf{Druckberechnung Pleuelfuß:}

$p_{Pleuel_{Benzin}} = \frac{F}{A}  = \frac{32675,5~N}{23,56~cm^2}  = 1386,91~N/cm^2  = 138,69~bar \\ p_{Pleuel_{Diesel}} = \frac{F}{A}  = \frac{90486~N}{23,56~cm^2}  = 3840,66~N/cm^2  = 384,07~bar$

\textbf{Versorgungsdruck (Öldruck) max.} $5~bar$

$\to p_{Pleuel_{Benzin}}:\,138,69~bar$

$\to p_{Pleuel_{Diesel}}:\,384,07~bar$

Vgl. Kapitel >>\emph{Motormechanik / Hydrodynamischer Schmierkeil}<<
