%ju 17-Sep-22 07-Loesung-Gemischbildung-Diesel.tex
\textbf{1) Was versteht man unter innerer Gemischbildung beim
Dieselmotor?}

Darunter versteht man, dass das Gemisch im Zylinderraum des Motors
entsteht. Der Dieselkraftstoff also erst dort mit der Ansaugluft in
Berührung kommt.

\textbf{2) Definieren Sie Quantitätsregelung und Qualitätsregelung.}

\begin{enumerate}
\item
  Unter einer \textbf{Quantitätsregelung} versteht man, dass der Motor
  durch eine Anpassung der Gemischmasse zum Beispiel durch eine
  Drosselklappe geregelt wird.
\item
  Bei einer \textbf{Qualitätsregelung} erfolgt die Luftzufuhr nahezu
  Last unabhängig. Lediglich die Kraftstoffmasse wird mit zunehmender
  Last erhöht.
\end{enumerate}

\textbf{3) Bei welcher Temperatur wird Dieselkraftstoff zündfähig und
wann entzündet er sich von selbst?}

\begin{enumerate}
\item
  zündfähig bei etwa $60^\circ\text{C}$
\item
  Selbstzündungstemperatur ca. $220 - 255^\circ\text{C}$
\end{enumerate}

\textbf{4) Erklären Sie ausführlich den Verbrennungsablauf im Brennraum
eines direkt einspritzenden Dieselmotors.}

Gegen Ende des Verdichtungstaktes wird Dieselkraftstoff fein zerstäubt,
in die $600 - 900^\circ\text{C}$ heiße Luft eingespritzt. Diese
verdampft über die Oberfläche der entstandenen Kraftstofftropfen. Je
feiner die Zerstäubung des Kraftstoffes, desto größer ist die Oberfläche
im Verhältnis zum Volumen, was den Vergasungsprozess begünstigt. Die
gasförmigen Kohlenwasserstoffe vermischen sich mit der Ansaugluft und
erhitzen sich, bis sich bei ca. $220^\circ\text{C}$ zur
Selbstentzündung des Kraftstoffes kommt. Je nach System wird dieser
Vorgang Last- und Drehzahlabhängig in mehreren Vorgängen unterteilt, um
den Druckanstieg im Zylinder und damit die Laufkultur des Motors zu
begünstigen.

\textbf{5) Unter welchen motorischen Bedingungen entstehen beim
Dieselmotor unverbrannte Kohlenwasserstoffe?}

\begin{enumerate}
\item
  Kaltstart- und Warmlaufphase (kalter Motor hat geringere Kompression,
  Kondensationsverluste an Zylinderwand, Wärmeabgabe an Brennraumwände)
\item
  Volllastbetrieb (Kraftstoffüberschuss)
\item
  Luftmangel (verstopfter Luftfilter, defekter Turbolader, undichter
  Ladeluftkühler)
\item
  Defekte Injektoren (schlechtes Strahlbild, tropfen nach $\to$ mehr
  Kraftstoff)
\item
  Kompressionsverluste (Verdichtungstemperatur wird später erreicht)
\end{enumerate}

\textbf{6) Warum wird beim Dieselmotor bei Volllast über einen größeren
Zeitraum eingespritzt?}

Da bei Volllast der maximale Einspritzdruck eingestellt wird, geht die
Mengenregelung nur über die Zeit.

\textbf{7) Wie groß ist beim Dieselmotor der Luftüberschuss bei Leerlauf
/ Volllast?}

\begin{itemize}
\item
  Leerlauf $\lambda = 10 \dots 18$
\item
  Volllast $\lambda = 1,15 \dots 2$
\end{itemize}

(Luftzahl / Lambda)

\textbf{8) Definieren Sie Zündverzug. Wie groß ist der Zündverzug?}

Die Zeitspanne von Einspritzbeginn an der Einspritzdüse und dem
Zündbeginn des Kraftstoff-Luft-Gemisches im Brennraum.

betriebswarmer Motor: 1 ms

\textbf{9) Wodurch kann es bei betriebswarmem Motor zum Nageln kommen?}

Durch einen zu großen Zündverzug.

\textbf{Mögliche Ursachen:}

\begin{enumerate}
\item
  zu früher Einspritzbeginn
\item
  Mangelhafte Kompression
\item
  Luftmangel (verstopfter Luftfilter, defekter Turbolader, undichter
  Ladeluftkühler)
\item
  schlechte Kraftstoffqualität (Cetanzahl zu niedrig)
\item
  Nach tropfende Einspritzdüse
\end{enumerate}

\textbf{10) Warum muss beim indirekt einspritzenden Dieselmotor das
Verdichtungsverhältnis 18 - 24:1 betragen, während beim direkt
einspritzenden Dieselmotor ein Verdichtungsverhältnis von 14 bis 18:1
ausreicht?}

Die Oberfläche des Zylinderraumes ist bedingt durch die Vor- oder
Wirbelkammeroberfläche bei indirekt einspritzenden Dieselmotor größer,
wodurch es zu stärkeren Wärmeverlusten kommt. Diese müssen durch eine
stärkere Erwärmung der Luft, also durch ein höheres
Verdichtungsverhältnis, kompensiert werden.

(Eine höhere Kompression kostet aber auch mehr Kompressionsarbeit.)
