%ju 31-Dez-22 19-Loesung-Bremse-I.tex
\textbf{1) Erläutern Sie den Begriff Nass -- Siedepunkt.}

Nass-Siedepunkt ist der Siedepunkt bei 3,5 \% Wasseranteil.

\textbf{2) Beschreiben Sie den Druckaufbau in einem Tandem --
Hauptzylinder.}

\textbf{Druckaufbau}, bei Bremsbetätigung bewegen sich der
Druckstangenkolben und der Zwischenkolben und überfahren die
Ausgleichsbohrungen und drücken Bremsflüssigkeit über die
Druckanschlüsse in die Bremskreise. Mit steigendem Druck wird der
Zwischenkolben nicht mehr von der gefesselten Kolbenfeder, sondern vom
Druck der Bremsflüssigkeit bewegt.

Der Fahrer betätigt mithilfe des Bremspedals und der Unterstützung eines
Bremskraftverstärkers den Druckstangenkolben. Die Primärmanschette
überfährt die Ausgleichsbohrung, wodurch der erste Bremskreis druckdicht
abgeschlossen ist. Durch weiteres Betätigen verdrängt der
Druckstangenkolben die hinter ihm befindliche Flüssigkeit, wodurch sich
der Zwischenkolben verschiebt. Dessen Primärmanschette überfährt die
Ausgleichsbohrung des zweiten Bremskreises, sodass dieser ebenfalls
druckdicht abgeschlossen ist. Jede weitere Betätigung Druckstangenkolben
führt zu einem gleichmäßigen Druckaufbau in beiden Bremskreisen.

\textbf{3) Beschreiben Sie, wie ein Tandem -- Hauptzylinder arbeitet,
wenn der Druckstangenkolbenkreis ausfällt.}

Der Druckstangenkolben durchfährt den undichten Druckraum des ersten
Bremskreises und läuft mechanisch auf dem Zwischenkolben auf. Wird der
Druckstangenkolben weiter verschoben, verschiebt dieser den
Zwischenkolben, wodurch im zweiten Bremskreis Bremsdruck aufgebaut wird.
Längerer Pedalweg.

\textbf{4) Beschreiben Sie den Füllvorgang bei einem Hauptzylinder mit
Zentralventil, Primärmanschette und Füllscheibe.}

Wird das Bremspedsal losgelassen, so werden Druckstangenkolben und
Zwischenkolben durch die Kolbenfedern in ihre Ausgangslage
zurückgeschoben. Ist es durch den Verschleiß von Bremsbelag oder
Bremsscheibe während des Bremsvorganges zu einem Flüssigkeitsverlust
gekommen, so muss dieser ausgeglichen werden. Dies geschieht im ersten
Bremskreis durch Abklappen der Primärmanschette vom Druckstangenkolben.
Die Füllscheibe hebt sich ab und die Nachlaufbohrungen werden frei.
Hierdurch kann Bremsflüssigkeit an der Füllscheibe und der
Primamanschette vorbei nachfließen. Im zweiten Bremskreis geschieht dies
durch Öffnen des Zentralventils, welche so verbaut ist, dass ein
Nachfließen von Bremsflüssigkeit in den Bremskreis jederzeit möglich
ist.

\textbf{5) Beschreiben Sie die Funktionsweise des gestuften
Tandem--Hauptzylinders bei intakten Bremskreisen und bei Ausfall eines
Bremskreises.}

Bemerkung: TT - Bremskreisaufteilung (schwarz-weiß), die
Zylinderdurchmesser sind gestuft. (Druckstangenkolben $\to$ VA-Kreis,
Zwischenkolben $\to$ HA-Kreis), Ausfall VA-Kreises: über HA 30 \%
Bremswirkung, Ausfall HA-Kreises: über VA 70 \% Bremswirkung

\begin{itemize}
\item
  \textbf{intakten Bremskreisen}

  \begin{itemize}
  \item
    Vorteil gegenüber normalen HBZ: ist in der Verdrängung eines
    größeren Flüssigkeitsvolumens im ersten Bremskreis, was zu einem
    schnelleren Ansprechen der vorderen Radbremsen führt.
  \end{itemize}
\item
  \textbf{Ausfall des vorderen Bremskreises}

  \begin{itemize}
  \item
    läuft der Druckstangenkolben direkt auf den Zwischenkolben auf. Da
    dieser eine kleinere Kolbenfläche hat, als der Druckstangenkolben
    kommt es bei gleicher Pedalkraft zu einer Druckerhöhung im hinteren
    Bremskreis.
  \end{itemize}
\item
  \textbf{Ausfall des hinteren Bremskreises}

  \begin{itemize}
  \item
    wird der Zwischenkolben bis an das Gehäuse des Hauptbremszylinders
    geschoben. Es kommt zu einem Druckaufbau im vorderen Bremskreis. Am
    Bremsdruck ändert sich in diesem Fall nichts, da die wirksame
    Kolbenfläche die gleiche ist, wie bei intakten Bremskreisen.
  \end{itemize}
\end{itemize}

\textbf{6) Welche Fehler lassen sich bei einer Niederdruckprüfung an
einer hydraulischen Bremsanlage feststellen?}

\begin{itemize}
\item
  Undichtigkeiten der Dichtungen innerhalb des Hauptbremszylinders
\item
  Undichtigkeiten der Bremsanlage nach außen
\end{itemize}

\textbf{7) Beschreiben Sie, wie im Bremskraftverstärker eine
Teilbremsung gesteuert wird.}

Sobald das Bremspedal betätigt wird, verschiebt sich der, am Ende der
Kolbenstange befindliche Ventilkolben und schließt zunächst das
Trennventil. Unterdruck und Arbeitskammer sind voneinander getrennt.
Durch weiteres betätigen der Kolbenstange wird der Ventilkolben in eine
Reaktionscheibe gepresst und das Außenluftventil öffnet. Der
Arbeitskolben verschiebt sich und unterstützt die Fußkraft des Fahrers
an der Kolbenstange des Hauptbremszylinders. Hält der Fahrer die
Position des Bremspedals konstant, um eine Teilbremsung zu erreichen, so
verschiebt sich der Arbeitskolben noch so lange weiter, bis das
Außenluftventil wieder geschlossen ist. Da das Trennventil bis zum
Loslassen des Pedals ebenfalls geschlossen bleibt, gerät der
Bremskraftverstärker in eine Ruhelage und hält die Position.

\textbf{8) Erläutern Sie den Begriff Bremsenkennwert >>C<<.}

Faktor der Selbstverstärkung einer Trommelbremse

\textbf{9) Beschreiben Sie, wie das Lüftspiel zwischen Bremsbelag und
Scheibe entsteht und wie groß es i.d.R. ist.}

\begin{itemize}
\item
  Die Abdichtung zwischen Bremskolben und Gehäuse erfolgt bei der
  Scheibenbremse mittels eines Rechteckringes.
\item
  Wird die Bremse betätigt, so haftet dieser Rechteckring am
  Bremskolben, wodurch sich der Rechteckring verspannt.
\item
  Wird die Bremse gelöst, entspannt sich der Rechteckring wieder und
  >>zieht<< den Bremskolben zurück.
\item
  Der Vorgang kann durch eine Spreizfeder zwischen den Belägen
  unterstützt werden.
\item
  Lüftspiel ca. 0,15 mm.
\end{itemize}

\textbf{10) Erläutern Sie den Unterschied zwischen Bremsdruckreglern und
lastabhängigen Bremsdruckbegrenzern.}

\begin{itemize}
\item
  \textbf{Bremsdruckregler}

  \begin{itemize}
  \item
    Sorgen dafür, dass der Bremsdruck an der Hinterachse, und
    Bremsvorgängen nur noch verwendet Anstalt. Hiermit wird der
    dynamischen Achslastverlagerung Rechnung getragen.
  \end{itemize}
\item
  \textbf{lastabhängige Bremskraftbegrenzer}

  \begin{itemize}
  \item
    Begrenzen den Bremsdruck an der Hinterachse auf einen definierten
    Wert. Die Höhe des maximalen Bremsdruckes ist von der Beladung des
    Fahrzeuges abhängig und wird durch die Änderung der
    Federvorspannkraft des Begrenzers festgelegt.
  \end{itemize}
\end{itemize}
