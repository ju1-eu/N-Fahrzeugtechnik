%ju 31-Dez-22 11-Abgasnachbehandlung.tex
\textbf{Zusammenhang von Ruß und Stickoxide}

\begin{enumerate}
\item
  \textbf{Ruß} Volllast, kalter Motor, >>Luftmangel<<, hohe
  Kraftstoffmenge (Gemisch: Fett $\lambda < 1$)
\item
  $\ce{\bold{NOx}}$ hohe Temperaturen, >>Luftüberschuss<<, geringe
  Kraftstoffmenge, (Gemisch: Mager $\lambda > 1$)
\end{enumerate}

\textbf{Katalysator} ist einen Stoff (Platin, Rhodium oder Palladium),
der eine chemische Reaktion auslöst bzw. beschleunigt, ohne dabei selbst
verbraucht zu werden.

\section{Partikel / Feinstaub}\label{partikel-feinstaub}

\textbf{Partikelmasse} (PM, Particulate Matter) entstehen bei der
unvollständigen Verbrennung.

\textbf{Zusammensetzung von Partikeln}

\begin{itemize}
\item
  Ruß ca. $65~\%$ (zusammengeballter Kohlenstoff)
\item
  mit angelagerten unverbrannte Kohlenwasserstoffverbindungen
  (krebserregend)
\item
  Sulfate
\item
  Schwermetalle
\end{itemize}

Rußpartikel tragen zur Feinstaubbelastung bei.

\textbf{Partikelemission} (Massenmessung g/km oder Trübungsmessung
$m^{-1}$)

\textbf{Partikelanzahl} (PN, Particle Number, Partikelanzahlmessung
1/km)

\textbf{Einfluss der Partikelgröße} (Zusammenhang zwischen Luftqualität
und Gesundheitsgefährdung)

\begin{itemize}
\item
  Durchmesser $< 10~\mu m$ (Feinstaub)

  \begin{itemize}
  \item
    Partikel-Gängigkeit Atmung
  \end{itemize}
\item
  Durchmesser $< 1~\mu m$ (Nanopartikel)

  \begin{itemize}
  \item
    Zellgängigkeit $\to$ Lungenbläschen (Alveolen) $\to$
    Blutkreislauf

    \begin{itemize}
    \item
      $\to$ Herz-Kreislaufsystem
    \item
      $\to$ Atemorgane
    \item
      $\to$ Krebserkrankungen
    \end{itemize}
  \end{itemize}
\end{itemize}

\section{Abgasemissionen von
Verbrennungsmotoren}\label{abgasemissionen-von-verbrennungsmotoren}

Quelle: TAK, 2022

\textbf{Rohemissionen beim Ottomotor} Abgaszusammensetzung bei einem Pkw
mit Ottomotor bei $\lambda \sim 1$ ohne Katalysator.

\begin{enumerate}
\item
  \textbf{Ansaugluft}

  \begin{enumerate}
  \def\labelenumii{\arabic{enumii}.}
  \item
    Sauerstoff ($\ce{O2}$) $\sim 21~\%$
  \item
    Stickstoff ($\ce{N2}$) $\sim 78~\%$
  \item
    Rest-/Edelgase $\sim 1~\%$
  \end{enumerate}
\item
  \textbf{Ottokraftstoff}

  \begin{enumerate}
  \def\labelenumii{\arabic{enumii}.}
  \item
    Kohlenstoff ($\ce{C}$) $\sim 84~\%$
  \item
    Wasserstoff ($\ce{H2}$) $\sim 16~\%$
  \end{enumerate}
\item
  \textbf{Nichtschadstoffe im Abgas}

  \begin{enumerate}
  \def\labelenumii{\arabic{enumii}.}
  \item
    Stickstoff ($\ce{N2}$) $\sim 71,0~\%$
  \item
    Kohlendioxid ($\ce{CO2}$) $\sim 18,0~\%$
  \item
    Wasserdampf ($\ce{H2O}$) $\sim 8,9~\%$
  \item
    Sauerstoff ($\ce{O2}$) $\sim 1,0~\%$
  \item
    Schadstoffe $\sim 1,1~\%$
  \end{enumerate}
\item
  \textbf{Schadstoffe} $\sim 1,1~\%$

  \begin{enumerate}
  \def\labelenumii{\arabic{enumii}.}
  \item
    Kohlenmonoxid ($\ce{CO}$) $\sim 0,9~\%$
  \item
    Kohlenwasserstoff ($\ce{HC}$) $\sim 0,08~\%$
  \item
    Stickoxid ($\ce{NOx}$) $\sim 0,12~\%$
  \end{enumerate}
\end{enumerate}

\textbf{Rohemissionen beim Dieselmotor} Abgaszusammensetzung bei einem
Pkw mit Dieselmotor bei $\lambda \sim 3$ ohne Katalysator.

\begin{enumerate}
\item
  \textbf{Ansaugluft}

  \begin{enumerate}
  \def\labelenumii{\arabic{enumii}.}
  \item
    Sauerstoff ($\ce{O2}$) $\sim 21~\%$
  \item
    Stickstoff ($\ce{N2}$) $\sim 78~\%$
  \item
    Rest-/Edelgase $\sim 1~\%$
  \end{enumerate}
\item
  \textbf{Dieselkraftstoff}

  \begin{enumerate}
  \def\labelenumii{\arabic{enumii}.}
  \item
    Kohlenstoff ($\ce{C}$) $\sim 86~\%$
  \item
    Wasserstoff ($\ce{H2}$) $\sim 14~\%$
  \end{enumerate}
\item
  \textbf{Nichtschadstoffe im Abgas}

  \begin{enumerate}
  \def\labelenumii{\arabic{enumii}.}
  \item
    Stickstoff ($\ce{N2}$) $\sim 76,0~\%$
  \item
    Kohlendioxid ($\ce{CO2}$) $\sim 7,0~\%$
  \item
    Wasserdampf ($\ce{H2O}$) $\sim 7,0~\%$
  \item
    Sauerstoff ($\ce{O2}$) $\sim 9,7~\%$
  \item
    Schadstoffe $\sim 0,3~\%$
  \end{enumerate}
\item
  \textbf{Schadstoffe} $\sim 0,3~\%$

  \begin{enumerate}
  \def\labelenumii{\arabic{enumii}.}
  \item
    Kohlenmonoxid ($\ce{CO}$) $\sim 0,05~\%$
  \item
    Kohlenwasserstoff ($\ce{HC}$) $\sim 0,03~\%$
  \item
    Partikel ($\ce{PM}$) $\sim 0,05~\%$
  \item
    Stickoxid ($\ce{NOx}$) $\sim 0,15~\%$
  \item
    Schwefeldioxid ($\ce{SO2}$) $\sim 0,02~\%$
  \end{enumerate}
\end{enumerate}

\textbf{Welche Auswirkungen haben Abgasemissionen auf den Menschen?}

\begin{enumerate}
\item
  $\ce{CO}$ schwerer als Luft (Vorsicht: Grube, ohnmächtig, führt zur
  Erstickung), giftig, geruchs- und farblos
\item
  $\ce{HC}$ krebserregend
\item
  $\ce{NOx}$ Reizen die Atemwege und tragen als Katalysator zu
  sommerlichen Ozonbildung bei.
\end{enumerate}

\textbf{Übersäuerung der Meere} Säure ist nicht gut für Kalk

\textbf{Wann entstehen die Schadstoffe?}

\begin{enumerate}
\item
  $\ce{CO} \text{ und } \ce{HC}$ unvollständige Verbrennung,
  Sauerstoffmangel oder Zündgrenze Fett
\item
  $\ce{PM}$ lokaler Sauerstoffmangel, Kraftstoff unter
  Sauerstoffmangel verbrannt wird
\item
  $\ce{NOx}$ hohe Verbrennungsdrücke und Verbrennungstemperaturen
\item
  $\ce{SO2}$ Bestandteil des Kraftstoffs
\end{enumerate}

\textbf{Unvollständige Verbrennung} $\ce{HC}, \ce{CO}$, Partikel und
Ruß (Sauerstoffmangel)

\textbf{Light-off-point} (Prüfung) beschreibt das Temperaturniveau (ca.
$350^\circ\text{C}$) bei dem eine Abgaskonvertierung von ca. 50 \%
stattfindet (Schadstoffumwandlung, Zusammenhang von
Katalysatortemperatur und Konvertierungsgrad).

\section{Schadstoffreduzierung oder
Minderung}\label{schadstoffreduzierung-oder-minderung}

\begin{enumerate}
\item
  \textbf{Außermotorisch} (Bauteile können getauscht werden)

  \begin{enumerate}
  \def\labelenumii{\arabic{enumii}.}
  \item
    Katalysator (Oxydationskatalysator bis 90 \% Konvertierungsrate in
    nicht Schadstoffe)
  \item
    Sekundärluftsystem (Luft in den Abgaskrümmer einblasen,
    Nachverbrennen der Abgase, Abgastemperatur erhöhen, schnell
    Light-off-Point erreichen)
  \item
    DPF
  \item
    SCR-Kat ($\ce{NOx}$ reduzieren)
  \item
    $\ce{NOx}$-Speicherkatalysator
  \item
    DPNR (Toyota, Diesel Particulate - NOx Reduction System, Rußpartikel
    und Stickoxid)
  \end{enumerate}
\item
  \textbf{Innermotorisch}

  \begin{enumerate}
  \def\labelenumii{\arabic{enumii}.}
  \item
    AGR (Hochdruck, Niederdruck)
  \item
    Brennraumoptimierung (Kolbenform: Muldenkolben oder Kompressionsraum
    $\to$ bessere Vermischung, bessere Verdunstung des Kraftstoffs,
    vollständigeren Verbrennung, weniger Ruß)
  \item
    Mehrventiltechnik (Liefergrad, Füllungsgrad bedarfsgerecht)
  \item
    Erhöhung der Einspritzdrücke
  \item
    Ladedruckregelung und Ladeluftkühlung
  \item
    Verfeinerung des Strahlbildes (Injektor)
  \item
    Optimierung von Einspritzbeginn und Menge (über Druck und Zeit,
    Mehrfacheinspritzung)
  \item
    Glühzeitsteuerung
  \item
    Einlasskanalsteuerung / Einlasskanalabschaltung (vollständigeren
    Verbrennung, Drallkanal, Füllungskanal mit Klappe, niedriger Last
    und Drehzahl vs.~steigender Motordrehzahl bzw. -last)
  \item
    Temperaturmanagement (höhere Betriebstemperatur der Motoren, weniger
    Kraftstoff notwendig)
  \end{enumerate}
\end{enumerate}

\section{Leitgrößen in der Abgasuntersuchung
(AU)}\label{leitgroessen-in-der-abgasuntersuchung-au}

\begin{enumerate}
\item
  $\ce{HC}$
\item
  $\ce{CO}$
\item
  $\ce{CO2}$
\item
  $\ce{O2}$
\item
  Leerlaufdrehzahl
\item
  Viergasmessung
\item
  Plakettenwert, Trübungswert (k-Wert)
\item
  Rauchgastrübungsmessung (Trübung des Abgases, Rauchbildung)
\item
  Partikelanzahlmessung (Diesel ab EURO 6, ersetzt
  Rauchgastrübungsmessung)
\item
  Motorkontrollleuchte (MIL) und Kontrollleuchte für den Partikelfilter
  muss nach Motorstart erloschen sein. Ist dies nicht der Fall, gilt die
  AU als nicht bestanden.
\item
  \textbf{RDE-Test} (Real Driving Emissions, dt.: reales Abgasverhalten,
  die Schadstoffemissionen $\ce{NOx, Partikel, CO, HC}$ werden auf der
  Straße ermittelt)
\end{enumerate}

\section{Herkunft der Kraftstoffe und ihre Nutzung in
Antrieben}\label{herkunft-der-kraftstoffe-und-ihre-nutzung-in-antrieben}

\begin{enumerate}
\item
  \textbf{Primärenergieträger}

  \begin{enumerate}
  \def\labelenumii{\arabic{enumii}.}
  \item
    Fossile Energieträger

    \begin{itemize}
    \item
      Erdöl, Kohle, Uran, Erdgas $\to$ Wasserstoff
    \end{itemize}
  \item
    Regenerative Energieträger

    \begin{itemize}
    \item
      Wind, Wasser, Solar, Biomasse $\to$ Bio-Diesel, Bio-Ethanol
    \end{itemize}
  \end{enumerate}
\item
  \textbf{Nutzenergieträger}

  \begin{enumerate}
  \def\labelenumii{\arabic{enumii}.}
  \item
    Kraftstoffe

    \begin{itemize}
    \item
      Benzin, Diesel, Erdgas (CNG), Autogas (LNG) $\to$
      Verbrennungsmotor
    \end{itemize}
  \item
    Elektrische Energie $\to$ Batterie
  \item
    Wasserstoff $\to$ Brennstoffzelle oder Verbrennungsmotor
  \end{enumerate}
\item
  \textbf{Antrieb}

  \begin{enumerate}
  \def\labelenumii{\arabic{enumii}.}
  \item
    Verbrennungsmotor
  \item
    E--Motor (Brennstoffzelle oder Batterie)
  \item
    Hybrid (Verbrennungsmotor und E--Motor)
  \end{enumerate}
\end{enumerate}

\section{Abgasrückführung (AGR)}\label{abgasrueckfuehrung-agr}

\textbf{Abgasrückführung,} Maßnahme zur Verminderung des
Stickoxidausstoßes.

Abgas wird aus dem Auspuff entnommen $\to$ über Abgaskühler abgekühlt
und über ein AGR-Ventil in das Saugrohr geleitet $\to$ vermischt mit
Frischluft und dem Brennraum wieder zugeführt. Alternative, eine
übergroße Ventilüberschneidung.

Die Abgase vermindern die angesaugte Frischluftmenge und senken die
Verbrennungstemperaturen. Zudem wirken die nicht brennbaren Gase kühlend
(Abgastemperatur $<1000^\circ\text{C} \iff$ Verbrennungstemperatur
$\le2500^\circ\text{C}$). NOx-Ausstoß sinkt.

\begin{itemize}
\item
  weniger Sauerstoff $\to$ geringere Kraftstoffmenge (homogen)
\item
  Verbrennungsdruck und Verbrennungstemperatur verringern sich
\item
  $\ce{NOx}$ sinkt
\end{itemize}

Bei hohen Drehzahlen und Volllast wird AGR abgeschaltet, sonst steigen
die Rußpartikel durch Frischluftmangel.

\textbf{Wodurch entstehen $\ce{\bold{NOx}}$?}

\begin{enumerate}
\item
  hohe Brennraumtemperaturen und Verbrennungsdrücke
\item
  hohe Flammgeschwindigkeiten bei Sauerstoffüberschuss
\end{enumerate}

\textbf{AGR-Rate} bis zu 60 \% bei Teillast und niedrige Last (80 Km/h
Landstraße, Pkw-Direkteinspritzern)

\textbf{Varianten der Abgasrückführung}

\begin{enumerate}
\item
  \textbf{interner AGR}

  \begin{itemize}
  \item
    über die Ventilsteuerzeiten wird die zurückgeführte Abgasmenge
    beeinflusst
  \item
    \emph{Nachteil:} hohe Temperaturen der Abgase begrenzt die mögliche
    AGR-Rate
  \end{itemize}
\item
  \textbf{externer AGR}

  \begin{enumerate}
  \def\labelenumii{\arabic{enumii}.}
  \item
    \textbf{Hochdruck-AGR}

    \begin{itemize}
    \item
      \emph{Entnahmestelle:} das Abgas wird vor dem >>Turbolader<<
      entnommen und über das AGR-Ventil dem Saugrohr zugeführt
    \item
      \emph{Nachteil:} heißes, unbereinigtes Abgas, voller Rußgehalt
    \item
      \emph{Vorteil:} Warmlaufphase verkürzen, Katalysator schnell auf
      Betriebstemperatur bringen
    \end{itemize}
  \item
    \textbf{Niederdruck-AGR}

    \begin{itemize}
    \item
      \emph{Entnahmestelle:} das Abgas wird hinter dem
      >>Oxydationskatalysator/DPF<< entnommen, über den AGR-Kühler und
      AGR-Ventil vor dem Verdichterrad des Turboladers die Frischluft
      wieder zugeführt.
    \item
      \emph{Vorteile:}

      \begin{enumerate}
      \def\labelenumiii{\arabic{enumiii}.}
      \item
        das zurückgeführte Abgas ist gereinigt und gekühlt
        (Verdichterrad)
      \item
        dem Turbolader steht die volle Abgasenergie zur Verfügung
        (Ansprechverhalten)
      \item
        gute Vermischung von Frischluft und Abgas (Gleichverteilung,
        höhere AGR-Raten)
      \end{enumerate}
    \item
      \emph{Nachteil:} größerer Bauaufwand
    \end{itemize}
  \end{enumerate}
\end{enumerate}

\section{Sekundärluftsysteme}\label{sekundaerluftsysteme}

Schnelles Aufheizen der Abgasnachbehandlungssysteme durch Erhöhen der
Abgastemperatur. Das Einbringen von Frischluft in den Abgaskrümmer
unmittelbar hinter die Auslassventile führt zu einer Nachverbrennung der
heißen Abgase.

\begin{itemize}
\item
  \textbf{Sekundärluft} wird über eine \textbf{elektrische
  Sekundärluftpumpe} und ein \textbf{Sekundärluftventil} in das
  Abgas-System eingeblasen.
\item
  \textbf{Rückschlagventil} verhindert ein Zurückströmen heißer Abgase.
\item
  \textbf{Motorsteuergerät} steuert Sekundärluftpumpe und
  Sekundärluftventil zeitgleich an.
\end{itemize}

\section{3-Wege-Katalysator (Ottomotor mit
Saugrohreinspritzung)}\label{wege-katalysator-ottomotor-mit-saugrohreinspritzung}

\textbf{Aufbau}

\begin{itemize}
\item
  Keramikträger oder Edelstahlträger (hohe Hitzebeständigkeit, erreicht
  schnelle Betriebstemperatur)
\item
  Durch das Aufbringen einer Zwischenschicht aus Aluminiumoxid
  vergrößert sich die Grundfläche des Trägers. Auf diese Fläche wird der
  eigentliche Katalysator aufgedampft.
\item
  \emph{Edelmetalle:} Platin, Palladium, Rhodium
\item
  Ottomotor: wird im Lambdafenster von $0,997 <\lambda< 1,004$
  betrieben
\item
  Magermix-Motoren: $\lambda>1,004$ nimmt die Reduktion von
  Stickoxiden aufgrund des Sauerstoffüberschusses ab
  (NOx-Speicherkatalysator notwendig)
\end{itemize}

\textbf{Funktion}

\begin{itemize}
\item
  \textbf{Chemische Prozesse}

  \begin{enumerate}
  \item
    Reduktion $\ce{NOx} \to \ce{N2 + O2}$
  \item
    Oxidation $\ce{CO} \to \ce{CO2}$
  \item
    Oxidation $\ce{HC} \to \ce{CO2 + H2O}$
  \end{enumerate}
\item
  \textbf{Betriebstemperatur} ca. $400 - 800^\circ\text{C}$

  \begin{itemize}
  \item
    fördert und beeinflusst die Oxidation- und Reduktionsvorgänge
  \end{itemize}
\item
  \textbf{Konvertierungsrate} bis zu $90~\%$
  (Schadstoffumwandlungsrate, Verbrennung des Kraftstoff-Luftgemisches
  bei $\lambda = 1$, stöchiometrisch $\hat{=}$ Lambda 1)
\item
  \textbf{Mindesttemperatur} ca. $350^\circ\text{C}$
\item
  thermischen Alterung ca. $> 900^\circ\text{C}$
\item
  Zerstörung des Katalysators ca. $>1000^\circ\text{C}$
\item
  \textbf{Einbauposition des Katalysators,} motornahe Einbauposition,
  dadurch bekommt er heißere Abgase und erreicht somit schneller seine
  Betriebstemperatur. Risiko einer Überhitzung des Katalysators bei
  hoher Motorlast bzw. Drehzahl.
\end{itemize}

\section{Oxydationskatalysator}\label{oxydationskatalysator}

\textbf{Aufbau}

Keramikträger oder Edelstahlträger, Zwischenschicht (Wash Coat)

Auf dem Keramikträger ist zur Vergrößerung der wirksamen Oberfläche eine
Beschichtung aus Aluminiumoxid aufgebracht. Auf der Trägerschicht
befindet sich der eigentliche Katalysator (Platin).

\textbf{Funktion}

\begin{itemize}
\item
  Oxidieren $\hat{=}$ Verbrennen
\item
  \textbf{Chemische Prozesse}

  \begin{enumerate}
  \item
    $\ce{CO} \to \ce{CO2}$
  \item
    $\ce{HC} \to \ce{CO2 + H2O}$
  \item
    $\ce{NO1} \to \ce{NO2}$ (Stickstoffmonoxid $\to$
    Stickstoffdioxid, Fahrzeuge mit Partikelfilter)

    \begin{itemize}
    \item
      $\ce{NO2}$ senkt die Zündtemperatur von Rußpartikeln auf ca.
      $300 - 450^\circ\text{C}$ und fördert die Regeneration von
      Rußpartikel (Beispiel: längerem Volllastbetrieb bei einer
      Autobahnfahrt)
    \item
      $\ce{NO/NO2}$-Verhältnis (vgl. SCR-Kat.)
    \end{itemize}
  \end{enumerate}
\item
  \textbf{Temperatur} wegen des hohen Sauerstoffgehalts im Abgas setzt
  die Wirkung bei ca. $> 170^\circ\text{C}$ ein
\end{itemize}

\textbf{Warum können bei einem Oxydationskatalysator keine
$\ce{\bold{NOx}}$ reduziert werden?}

Durch den Luftüberschuss im Abgas.

Volllast: $\lambda \approx 1,3$, Leerlauf: $\lambda \approx 18$

\section{Partikelfilter (DPF Dieselpartikelfilter / OPF
Ottopartikelfilter)}\label{partikelfilter-dpf-dieselpartikelfilter-opf-ottopartikelfilter}

\textbf{Wovon ist die Bildung von Rußpartikeln abhängig?}

\begin{enumerate}
\item
  Luftzufuhr
\item
  Zerstäubung des Kraftstoffs und
\item
  Flammenausbreitung
\end{enumerate}

\textbf{Wodurch können die Partikel begrenzt werden?}

\begin{enumerate}
\item
  Kraftstoff unter hohem Druck fein zerstäubt in der richtigen Menge und
  zum günstigsten Zeitpunkt eingespritzt wird.
\item
  Kraftstoffqualität
\item
  Erhöhung der Cetanzahl (Zündwilligkeit)
\item
  Verringerung des Schwefelgehaltes
\end{enumerate}

\textbf{Vollstrom-Partikelfilter (Hauptstromfilter, geschlossenes
System, über 98 \%) vs.~Teilstrom-Partikelfilter (Nebenstromfilter, 30
-- 70 \% Partikel zurückhalten)}

Filtermaterial: Keramik, Siliziumcarbid oder Sintermetall

Im Keramikkörper sind die Kanäle wechselseitig verschlossen.

Das gesamte Abgas muss durch das poröse Keramikmaterial strömen.
Porengröße: gasförmige Abgase strömen durch und Partikel bleiben zurück.

\textbf{Erfassung des Beladungszustands}

Nimmt die Beladung weiter zu, erhöht sich der Abgasgegendruck des
Filters. Der Differenzdrucksensor misst den Druckunterschied vor und
nach dem Partikelfilter. Im SG sind Kennlinien mit Sollwerten für einen
defekten, sauberen und einen beladenen Filter abgelegt.

\begin{itemize}
\item
  Partikelmasse (PM) $\to$ Ruß (Kohlenstoff)
\item
  \textbf{Chemische Prozesse}

  \begin{itemize}
  \item
    Rußpartikel $\to \ce{CO2}$ verbrannt
  \item
    Ruß kann bei Temperaturen von ca. $550^\circ \text{C}$ zu
    gasförmigen Kohlendioxid verbrannt werden. Rest ist Asche.
  \item
    Additiv: Absenkung der Partikelabbrenntemperatur
  \item
    Größe: 1/1000 -- 1/100 mm
  \end{itemize}
\item
  \textbf{3x Arten der Regenerierung des DPF} Zündtemperatur von
  Rußpartikel ca. $550 - 600^\circ\text{C}$

  \begin{enumerate}
  \item
    \textbf{natürliche Regenerierung}

    \begin{itemize}
    \item
      Regenerierung erfolgt nach spätestens 1000 km
    \item
      längere Autobahnfahrt mit höherer Motorleistung erforderlich,
      Prozess läuft langsam ab
    \item
      Oxy-Kat erforderlich:

      \begin{itemize}
      \item
        $\ce{NO} \to$ $\ce{NO2}$ für Oxidation von Ruß
      \item
        Nachoxydation von $\ce{CO und HC} \to$ um Abgastemperatur
        anheben
      \end{itemize}
    \end{itemize}
  \item
    \textbf{erzwungene, aktive Regenerierung} (bei häufigem
    Kurzstreckenbetrieb)

    \begin{enumerate}
    \def\labelenumii{\arabic{enumii}.}
    \item
      \textbf{Stufe} Abgastemperatur durch eine Nacheinspritzung
      erhöhen, um den Oxydationskatalysator auf Betriebstemperatur zu
      bringen. (Motorbrenner)
    \item
      \textbf{Stufe} Nacheinspritzung erfolgt spät, sodass der
      Kraftstoff im Zylinder nur verdampft. Der gasförmige Kraftstoff
      verbrennt erst im Katalysator vor dem Partikelfilter und heizt den
      Filter auf die Zündtemperatur der Partikel auf (Katbrenner).

      \begin{itemize}
      \item
        Dauer: 5 -- 15 Min.
      \item
        Motorlast anheben: zusätzliche Verbraucher, Glühkerzen,
        Heckscheibenheizung, AGR abschalten
      \item
        \textbf{Temperaturerhöhung}

        \begin{enumerate}
        \def\labelenumiii{\arabic{enumiii}.}
        \item
          \emph{Pkw} Nacheinspritzung in den Zylinder zum Ende des
          Arbeitstaktes
        \item
          \emph{Lkw} Nacheinspritzung über einen Auspuffinjektor
          (HC-Doser) in das Auspuffrohr (verhindert eine Verdünnung des
          Motoröls)

          \begin{itemize}
          \item
            Katalysator Nachverbrennung
          \item
            Anstieg der Abgastemperatur $\to$ Ruß abbrennen $\to$
            Rest-Asche
          \end{itemize}
        \end{enumerate}
      \end{itemize}
    \end{enumerate}
  \item
    \textbf{Zwangsregenerierung} (Kfz-Werksatt, wenn Regeneration durch
    Motorstopp unterbrochen wurde)

    \begin{itemize}
    \item
      Kontrollleuchte DPF an
    \item
      Motor auf Betriebstemperatur
    \item
      mit einem Diagnosetester kann die Regeneration im Stand ausgelöst
      werden. Dabei läuft der Motor unter Aktivierung der
      Nebenverbraucher mehrere Minuten mit hohen Drehzahlen. Durch die
      hohen Abgastemperaturen sollte man keine Abgasanlage anschließen.
    \item
      Ölwechsel (Empfehlung wegen Ölverdünnung)
    \item
      Dauer: 5 -- 45 Min.
    \end{itemize}
  \end{enumerate}
\item
  \textbf{bei längerem Kurzstreckenverkehr} Abgastemperatur nur selten
  $> 250^\circ\text{C}$, findet keine oder eine unzureichende
  Regeneration statt. Partikelfilter setzt sich zu und es kommt zu einem
  erhöhten Abgasgegendruck. Kontrollleuchte für den Partikelfilter wird
  angesteuert.
\end{itemize}

\section{NOx-Speicherkatalysator zur
NOx-Minderung}\label{nox-speicherkatalysator-zur-nox-minderung}

\begin{itemize}
\item
  \textbf{Anordnung der Katalysatoren im Abgasstrang}

  \begin{itemize}
  \item
    Motor $\to$ Oxydationskatalysator (DOC, Oxidation von
    $\ce{CO und HC}$) $\to$ DPF (Rußpartikel filtern) $\to$
    $\ce{NOx}$-Speicherkatalysator ($\ce{NOx}$ speichern 50 -- 70
    \%)
  \end{itemize}
\item
  \textbf{Chemische Prozesse}

  \begin{enumerate}
  \item
    Beladungsphase $\lambda > 1$ (mager Betrieb), Dauer: 30 -- 300 s

    \begin{itemize}
    \item
      Bedingung: im Oxykat: muss $\ce{NO+O2} \to$ $\ce{NO2}$
      umgewandelt werden (wir brauchen einen hohen Anteil
      Stickstoffdioxid)
    \item
      Speichermedium Bariumoxid reagiert zusammen mit $\ce{NOx}$ zu
      Bariumnitrat
    \end{itemize}
  \item
    NOX-Speicherkatalysator regenerieren (Speicherfähigkeit erschöpft)
    $\lambda < 1$ (fetten Betrieb), Dauer: 2 -- 10 s

    \begin{itemize}
    \item
      Reduktionsmittel: unverbrannte Kohlenwasserstoffe (durch Anfetten)
      reagieren mit dem Sauerstoff von Bariumnitrat und reduzieren ihn
      zu Bariumoxid
    \end{itemize}
  \end{enumerate}
\item
  \textbf{Was ist eine Schwefelvergiftung,} Schwefel reagiert mit
  Bariumoxid zu Bariumsulfat, wodurch die Speicherfähigkeit des
  Katalysators herabgesetzt wird. Das SG erkennt durch immer kürzer
  werdenden Beladungsphasen und leitet eine Schwefelregeneration ein.
\item
  \textbf{Schwefelregeneration} (Entschwefelung, Desulfatisierung):

  \begin{itemize}
  \item
    nach etwa 5000 km frei brennen
  \item
    Durch Nacheinspritzung wird die Abgastemperatur im
    NOx-Speicherkatalysator auf etwa $> 650^\circ\text{C}$ gebracht.
    Schwefel verbrennt auf der Oberfläche.
  \end{itemize}
\item
  \textbf{Temperaturen} zwischen ca. $250 - 550^\circ\text{C}$ wird
  für eine \textbf{höchste $\ce{NOx}$-Umwandlung} benötigt
\item
  \textbf{Temperaturen} $> 750^\circ\text{C}$ \textbf{schädigen} den
  Speicherkatalysator
\end{itemize}

\section{SCR-Katalysator}\label{scr-katalysator}

\begin{itemize}
\item
  \textbf{AdBlue} hochreine, 32,5 \% - ige wässrige Harnstofflösung

  \begin{itemize}
  \item
    Ammoniakträger AdBlue, Reduktionsmittel Amoniak ($\ce{NH3}$)
  \item
    Mischungsverhältnis (Harnstoff 32,5 \%, Wasser 67,5 \%)
  \item
    Gefrierpunkt $< -11^\circ\text{C}$
  \item
    Refraktometer: Harnstoffgehalt messen
  \end{itemize}
\item
  \textbf{SCR} selective, also bevorzugte katalytische Reduktion
  (vordergründig Stickoxide verringern)
\item
  \textbf{Anordnung der Katalysatoren im Abgasstrang}

  \begin{itemize}
  \item
    Motor $\to$ Oxydationskatalysator ( = Diesel Oxidation Catalytic
    Converter, Oxidation von $\ce{CO und HC}$) $\to$ DPF
    (Rußpartikel filtern) $\to$ AdBlue eindüsung $\to$
    SCR-Katalysator ($\ce{NOx}$-Reduktion)
  \end{itemize}
\end{itemize}

\textbf{Funktion}

\begin{itemize}
\item
  Nach Eindüsung des Reduktionsmittels (AdBlue) in den Abgasstrang muss
  zunächst Ammoniak ($\ce{NH3}$) gebildet werden.
\item
  Ammoniakbildung erfolgt in zwei Phasen

  \begin{enumerate}
  \item
    \textbf{Thermolyse} Harnstoff wird durch Einfluss der Abgaswärme
    $\to$ in Ammoniak $\ce{NH3}$ und Isocyansäure umgewandelt
  \item
    \textbf{Hydrolyse} durch Einfluss von Wasser wird die unerwünschte
    Isocyansäure in $\ce{CO2}$ und Ammoniak $\ce{NH3}$ umgewandelt
  \end{enumerate}
\item
  Das entstandene Ammoniak muss im Katalysator an einer
  Speicherkomponente zwischengespeichert werden.

  \begin{itemize}
  \item
    $\ce{NH3}$-Speicherkomponente: Zeolith, Titan, Vanadium,
    Wolfram-Oxid-Verbindungen (Beschichtung)
  \end{itemize}
\item
  \textbf{Chemische Prozesse}

  \begin{itemize}
  \item
    Stickoxide ($\ce{NO und NO2}$) treffen im Katalysator auf den
    zwischen gespeicherten Ammoniak ($\ce{NH3}$) und werden in
    Stickstoff ($\ce{N2}$) und Wasserdampf ($\ce{H2O}$) umgewandelt
    und in Kohlendioxid ($\ce{CO2}$).
  \item
    Reaktion läuft bei Temperaturen $> 200^\circ\text{C}$ ab
  \item
    schnelle SCR-Reaktion

    \begin{itemize}
    \item
      ausgewogenes $\ce{NO/NO2}$-Verhältnis wird im
      Oxidationkatalysator durch Oxidation von $\ce{NO} \to \ce{NO2}$
      hergestellt
    \end{itemize}
  \end{itemize}
\item
  \textbf{Temperatur} ca. $250 - 450^\circ\text{C}$
\end{itemize}

\section{Fragen zur
Abgasnachbehandlung}\label{fragen-zur-abgasnachbehandlung}

\textbf{Motorische Notlaufmaßnahmen dienen $\ldots$}

Der Erhaltung der Fahrsicherheit, der Vermeidung von Folgeschäden und
der Minimierung von Abgasemissionen.

\textbf{Welche Gemischbildung wird bei Dieselmotoren und direkt
einspritzenden Ottomotoren eingesetzt?}

Innere Gemischbildung

\textbf{Wofür steht die Abkürzung SCR?}

Selective Catalytic Reduction

\textbf{Welche Aufgabe hat der NOx-Sensor nach dem SCR-Katalysator?}

Er dient unter anderem zur Überwachung des Wirkungsgrades des
SCR-Systems.

\textbf{Was misst der $\ce{NOx}$-Sensor?}

Die Menge an $\ce{NOx}$ messen.

\textbf{Welche Aussage ist richtig bezüglich der Reduzierung der
Abgasemissionen?}

Die Emissionsreduzierung kann zumindest teilweise durch innermotorische
Maßnahmen umgesetzt werden.

\textbf{Was versteht man unter der >>passiven Regeneration<< eines
Dieselpartikelfilters?}

Die Rußpartikel werden ohne Eingriff der Motorsteuerung kontinuierlich
verbrannt.

\begin{enumerate}
\item
  \textbf{passive Regeneration DPF} durch den täglichen Betrieb wird die
  Abgas-Regeneration eingeleitet.
\item
  \textbf{aktive Regeneration DPF} SGe Eingriff in das System

  \begin{itemize}
  \item
    Gefahr der Öl-Verdünnung, wenn Regeneration unterbrochen wird.
  \end{itemize}
\end{enumerate}

\textbf{Welche Aussage ist richtig bezüglich der Entstehung von
Rußpartikeln?}

Rußpartikel entstehen in den lokal fetten Gemischzonen des Brennraums.

\textbf{Was ist AdBlue?}

AdBlue ist ein Betriebsstoff zur Abgasnachbehandlung, der aus Harnstoff
und aus destilliertem/demineralisiertem Wasser besteht.

\textbf{Welche Aufgabe hat das Reduktionsmittel AdBlue?}

AdBlue reduziert die Stickoxidemissionen fast vollständig und senkt den
Partikelausstoß.

\textbf{Das Umkehrventil beim AdBlue-System hat die Aufgabe, das
$\ldots$}

Kein AdBlue in der Förderleitung und im Einspritzventil bei kalten
Außentemperaturen gefrieren kann.

\textbf{Durch welche der aufgeführten Stoffe werden das Verbrennen der
Rußpartikel bei der Regeneration des Dieselpartikelfilters verbessert?}

Platin

\textbf{Bei Motor fernen Dieselpartikelfiltern kommt ein
Kraftstoff-Additiv zum Einsatz. Welche Aufgabe hat dieses Additiv?}

Durch das Additiv sinkt die Verbrennungstemperatur der Rußpartikel auf
ca. $500^\circ\text{C}$, damit die Regeneration des Partikelfilters
auch bei Teillastbetrieb möglich ist.

\textbf{Welches Sensorsignal wird vom Motorsteuergerät zur Mittlung der
Rußbeladung des Partikelfilters unter anderem benötigt?}

Das Signal des Abgastemperatursensors vor dem Partikelfilter.

\textbf{Wie groß ist in etwa ein Rußpartikel, das beim
Verbrennungsprozess eines Dieselmotors entsteht?}

$0,02 - 0,05~\mu m$

\textbf{Warum ist Feinstaub gefährlicher als Rußpartikel?}

Feinstaub ist Zell gängig $\to$ Krebs

\textbf{Welche Eigenschaft hat das Reduktionsmittel AdBlue?}

Es gefriert bei einer Temperatur $\leq -11^\circ\text{C}$

\textbf{Welche Aufgabe hat die Abgasrückführung?}

Durch die Abgasrückführung werden die $\ce{NOx}$-Emissionen gesenkt.

\textbf{Welche Auswirkung hat der Betrieb mit einem vollständig
entleerten Reduktionsmittel Tank bei Pkw?}

Der Verbrennungsmotor lässt sich beim nächsten Versuch nicht mehr
starten.

\textbf{Stickoxide aus der motorischen Verbrennung schädigen $\ldots$}

Die Umwelt durch Smog und >>Sauren Regen<<.

\textbf{Eine Schaltdiode wird eingesetzt $\ldots$}

Zum schnellen Umschalten von hoher auf niedrige Impedanz und umgekehrt.

\textbf{Durch die OBD-Gesetzgebung muss $\ldots$}

Fehlerspeicherinformationen und der Zugriff auf diese Informationen
standardisiert werden.

\textbf{Bei der Desulfatisierung (Entschwefelung) wird $\ldots$}

Der $\ce{NOx}$-Speicherkatalysator zur Schwefelregenerierung für eine
bestimmte Zeit (z. B. 5 Minuten) auf über $650^\circ\text{C}$
aufgeheizt und mit fettem Abgas $\lambda < 1$ beaufschlagt.

\textbf{Aufgrund gesetzlicher Vorgaben müssen Systeme für die Erkennung
von Verbrennungsaussetzern $\ldots$}

Bei Ottomotoren in jedem Betriebspunkt und bei Dieselmotoren nur im
Leerlauf aktiv sein.
