%ju 31-Dez-22 11-Loesung-Schadstoffminderung-Diesel.tex
\textbf{1. Welches Ziel wird mit der Abgasrückführung verfolgt?}

Durch die Rückführung von nicht brennbaren Abgasen in den Zylinder wird
die Verbrennungsspitzentemperatur gesenkt, um den Ausstoß von
Stickoxiden zu reduzieren.

\textbf{2. Beschreiben Sie Aufbau und Funktion der
Einlasskanalsteuerung.}

Motoren mit einer Einlasskanalsteuerung verfügen einlassseitig über
einen Drall- und Füllungskanal.

Der Drallkanal versetzt die durchströmende Luft in Drehbewegung. Im
unteren Drehzahl und Lastbereich ist nur Drallkanal geöffnet. Die
angesaugte Luft gerät Strudel förmig in den Zylinder, wodurch es zu
einer besseren Durchmischung des eingespritzten Kraftstoffes kommt.

Um bei hohen Drehzahlen und Lasten eine ausreichende Zylinderfüllung zu
gewährleisten, wird hier der Füllungskanal durch die Füllungsklappe
freigegeben. Der hohe Luftdurchsatz im Drallkanal sorgt dafür, dass die
Luft, die ihm durchströmt, angedreht wird.

Die Luft aus dem Füllungskanal wird im Zylinder von der Luft aus dem
Drallkanal mitgerissen und gerät so ebenfalls in Drehbewegung.

\textbf{3. Welcher Teil der Abgase eines Dieselmotors gehört zu den
Schadstoffen / Nichtschadstoffen?}

\begin{enumerate}
\item
  \textbf{Nichtschadstoffe im Abgas}

  \begin{enumerate}
  \def\labelenumii{\arabic{enumii}.}
  \item
    Stickstoff ($\ce{N2}$) $\sim 76,0~\%$
  \item
    Kohlendioxid ($\ce{CO2}$) $\sim 7,0~\%$
  \item
    Wasserdampf ($\ce{H2O}$) $\sim 7,0~\%$
  \item
    Sauerstoff ($\ce{O2}$) $\sim 9,7~\%$
  \end{enumerate}
\item
  \textbf{Schadstoffe} $\sim 0,3~\%$

  \begin{enumerate}
  \def\labelenumii{\arabic{enumii}.}
  \item
    Kohlenmonoxid ($\ce{CO}$) $\sim 0,05~\%$
  \item
    Kohlenwasserstoff ($\ce{HC}$) $\sim 0,03~\%$
  \item
    Partikel ($\ce{PM}$) $\sim 0,05~\%$
  \item
    Stickoxid ($\ce{NOx}$) $\sim 0,15~\%$
  \item
    Schwefeldioxid ($\ce{SO2}$) $\sim 0,02~\%$
  \end{enumerate}
\end{enumerate}

\textbf{4. Wie werden die Stickoxide im NOx-Speicherkatalysator
gehalten?}

$\ce{NO}$ Bestandteile des Abgases werden mithilfe des vorgeschalteten
Oxidationskatalysators und evt. unter zu Hilfename eines beschichteten
Dieselpartikelfilters zu $\ce{NO2}$ oxidiert. Dieses lagert sich als
Nitrat $\ce{NO3}$ an der Bariumoxidbeschichteten Oberfläche des
Katalysators an.

\textbf{5. Was ist eine >>Schwefelvergiftung<< in Bezug auf
Diesel--Abgasnachbehandlungssysteme? Wie entsteht diese und wie wird sie
behoben?}

Als Schwefelvergiftung bezeichnet man die Bildung von Sulfat
$\ce{SO2}$ an der Bariumoxidbeschichteten Oberfläche des NOx
Speicherkatalysators. Diese Sulfat behindert die Bindung der Stickoxide
mit dem Bariumoxid und somit deren Speicherung. Um die Funktion des
Katalysators wiederherzustellen muss er für einen Zeitraum für mind. 5
Minuten auf eine Temperatur von mehr als $650^\circ\text{C}$

\textbf{6. Beschreiben Sie ausführlich (inkl. Temperaturangaben) den
Regenerationsprozess des Diesel--Partikel -- Hauptstromfilters.}

Das Steuergerät erfasst mittels eines Differenz- oder Abgasdrucksensors
kontinuierlich den Beladungszustand des Partikelfilters. Steigt dieser
über den vom Hersteller definierten Grenzwert wird die Regeneration
eingeleitet. Ab einer Temperatur von ca. $600^\circ\text{C}$ im
Partikelfilter oxidieren die Partikel mit dem im Abgas enthaltenen
Sauerstoff.

Um diese Temperatur zu erreichen wird je nach System die Motorlast durch
zuschalten von Nebenverbrauchern erhöht.

Eine Nacheinspritzung zur Erhöhung der Abgastemperatur eingeleitet oder
direkt in die Abgasanlage eingespritzt. Die Temperatur darf
$700^\circ\text{C}$ nicht übersteigen, da dies zur einer Zerstörung
des Partikelfilters führen würde. Einige Hersteller geben den Kraftstoff
zusätzlich ein additiv bei, welches die Partikelabbrenntemperatur auf
ca. $450 - 500^\circ\text{C}$ absenkt. Bei längerer Fahrt unter hoher
Last regeneriert der Partikelfilter sich durch die hohen
Abgastemperaturen selbsttätig. Ein Steuergeräteeingriff ist nicht
notwendig.

\textbf{7. Warum lässt sich ein Hauptstrom -- Dieselpartikelfilter nur
in wenigen Fällen nachrüsten?}

Ein Hauptstrom-Dieselpartikelfilter lässt sich nur bei Fahrzeugen, die
vom Hersteller bereits dafür vorgesehen sind nachrüsten. Weil das
Motorsteuergerät nur in diesen fällen auf die Überwachung und
Regeneration des Filters vorbereitet ist. In allen anderen Fahrzeugen
ist eine gezielte Regeneration aufgrund der fehlenden Funktionen nicht
möglich. Der Anstieg des Abgasgegendrucks und damit der Ausfall des
Fahrzeugs wäre die Folge.

\textbf{8. Wozu dient das SCR--Abgasnachbehandlungssystem?}

Das SCR-Abgasnachbehandlungssystem dient zur Reduktion von Stickoxiden
$\ce{NOx}$. Ammoniak reagiert mithilfe des Katalysators Titanium ab
ca. $170^\circ\text{C}$ mit den Stickoxiden. Reduziert diese zu
Stickstoff $\ce{N2}$ und oxidiert die übrigen Bestandteile zu Wasser
$\ce{H2O}$.
