%ju 31-Dez-22 05-Loesung-Betriebs-u-Hilfsstoffe.tex
\textbf{1) Was versteht man unter einer fraktionierenden Destillation
und welche Produkte fallen dabei an?}

Unter fraktionierender Destillation versteht man das Aufteilen von Rohöl
nach Siedebereichen. Hierzu wird das Rohöl erhitzt und in eine Kolonne
geleitet, wo die einzelnen Bestandteile kondensieren und über
Glockenböden abgeführt werden.

\emph{Gase die dabei entstehen sind:}

\begin{enumerate}
\item
  Propan
\item
  Butan
\item
  Methan
\end{enumerate}

\emph{Die anfallenden Produkte sind}

\begin{enumerate}
\item
  Leicht- und Schwerbenzin
\item
  Petroleum
\item
  Diesel
\item
  Gas- und Spindelöle
\item
  Mineralische Motoröle
\item
  Zylinderöl
\item
  Bitumen
\end{enumerate}

\textbf{2) Was versteht man unter Viskosität?} (Prüfung)

Unter Viskosität versteht man die Eigenschaft einer Flüssigkeit ihrer
Verformung einen Widerstand entgegenzusetzen.

\textbf{3) Was versteht man unter Cracken und was wird dadurch erreicht?
Welche Arten von Cracken gibt es?}

Cracken nennt man das Spalten von schwer siedende (langkettige)
Kohlenwasserstoffmolekülen in leicht siedende (kurzkettige)
Kohlenwasserstoffmoleküle, wodurch die Klopffestigkeit eines Kraftstoffs
erhöht wird. Langkettige Kohlenwasserstoffmoleküle sind >>schwer siedend
und reaktionsfreudig<<. Kurzkettige Kohlenwasserstoffmoleküle sind
>>leicht siedend und reaktionsträge<<.

\emph{Crackarten:} thermisches Cracken, katalytisches Cracken und
Hydrocracken.

\textbf{Katalysator} ist ein Stoff, der die Reaktionsgeschwindigkeit
einer chemischen Reaktion beeinflusst, ohne dabei selbst verbraucht zu
werden.

\textbf{4) Was gibt die Cetanzahl an?}

Das ist ein Maß für die Zündwilligkeit von Dieselkraftstoff.

\textbf{5) Welche Aufgabe haben Biozide als Additiv im
Dieselkraftstoff?}

Biozide sollen das Bakterienwachstum im Dieselkraftstoff verhindern.
Diese würden zu folgenden Problemen im Einspritzsystem führen:

\begin{itemize}
\item
  Verstopfen der Filtersysteme durch die lebenden oder abgestorbenen
  Bakterien
\item
  Erosionsschäden durch die mit hoher Geschwindigkeit und hohem Druck
  durch das Einspritzsystem geförderten Bakterien (ähnlich
  Sandstrahleneffekt)
\item
  Korrosionsschäden durch die Bakterien Exkremente $H_2SO_3$ und
  $H_2SO_4$
\end{itemize}

\textbf{6) Warum dürfen moderne Dieselmotoren keinesfalls mit
Ottokraftstoff betrieben werden?}

Dieselkraftstoffe erfüllen neben ihrer Hauptaufgabe, als
Energielieferant noch die Aufgabe, die Bauteile des Einspritzsystems zu
schmieren.

Diese Aufgabe kann Ottokraftstoff nicht übernehmen. Der Schmierfilm in
den Bauteilen der Einspritzanlage, vorwiegend in der Hochdruckpumpe,
reißt ab. Es kommt zur Trockenreibung und damit zur Zerstörung der
Bauteile.

Alte Dieseleinspritzsysteme waren demgegenüber weniger anfällig, weshalb
je nach Hersteller eine Beimischung von bis zu $50~\%$ Ottokraftstoff
möglich war.

\textbf{7) Was versteht man unter E10-Kraftstoff?}

Unter E10 Kraftstoff versteht man ein Gemisch aus $90~\%$ Benzin und
bis zu $10~\%$ Bioethanol. Es wurde eingeführt, um den Anteil
regenerativer Energiequellen am Kraftstoff zu erhöhen.

\textbf{8) Welche Anforderungen werden an Motoröl gestellt?}

\begin{enumerate}
\item
  hoher Viskositätsindex (Änderung der Viskosität über der Temperatur)
\item
  hohe Temperaturbeständigkeit
\item
  geringe Verdampfungsneigung (geringer Ölverbrauch)
\item
  Öl soll einen niedrigen Stockpunkt haben
\item
  soll weiterhin eine hohe Schmutz- und Säureaufnahmefähigkeit besitzen
\end{enumerate}

\textbf{9) Welche Gruppen von Fetten unterscheidet man?}

\begin{enumerate}
\item
  Lithiumseifenfett
\item
  Natriumseifenfett
\item
  Kalziumseifenfett
\end{enumerate}

\textbf{10) Welche Eigenschaften hat Bremsflüssigkeit (DOT4)?}

\begin{itemize}
\item
  hat einen hohen Trockensiedepunkt $\geq 230~^\circ\text{C}$
  (Mindestsiedepunkt)
\item
  hat einen hohen Nasssiedepunkt $\geq 170~^\circ\text{C}$
\item
  hygroskopisch, entzieht der atmosphärischen Luft die Feuchtigkeit und
  speichert diese.
\item
  ist hochgiftig, $100~cm^3$ sind bereits tödlich
\item
  Aggressiv, greift Lacke und die menschliche Haut an
\end{itemize}
