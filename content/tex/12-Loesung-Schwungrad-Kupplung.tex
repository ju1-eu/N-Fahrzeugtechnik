%ju 26-Dez-22 12-Loesung-Schwungrad-Kupplung.tex
\textbf{1. Warum wird ein Zweimassenschwungrad verbaut?}

Es wird verbaut, um die Übertragung der Zündungs bedingten
ungleichförmigen Drehbewegungen der Kurbelwelle auf das Getriebe zu
dämpfen.

Außerdem vermindert es ein Motorschütteln, Getriebe rasseln, sowie
Dröhn- und Brummgeräusche in der Karosserie.

\textbf{2. Nennen Sie die Vorteile einer Membranfederkupplung gegenüber
der Schraubenfederkupplung.}

\begin{itemize}
\item
  geeignet für hohe Drehzahlen $>6000~min^{-1}$
\item
  geringer Platz bedarf
\item
  Kann zum Ausdrücken gedrückt oder gezogen werden
\item
  durch höhere Anpresskraft können höhere Drehmomente übertragen werden
\item
  Geringere Betätigungskräfte, da die Membranfeder den
  Auskupplungsvorgang unterstützt und die Anpresskraft bleibt vom
  Neuzustand bis zur Verschleißgrenze nahezu gleich groß
\end{itemize}

\textbf{3. Wo werden Lamellenkupplungen verwendet?}

\begin{itemize}
\item
  bei nass laufenden Kupplungen in Automatikgetrieben
\item
  Automatisierte Schaltgetriebe (DSG, Drehsinn vorwärts, rückwärts)
\item
  Sperrdifferenzial
\item
  Krafträder
\end{itemize}

\textbf{4. Welche Aufgabe haben die Belagfedern einer Kupplungsscheibe?}

Belagfedern bewirken ein weiches Einkuppeln und gleichmäßiges Anlegen
der Beläge.

\textbf{5. Welche Aufgabe hat die Torsionsdämpfung in einer
Kupplungsscheibe?}

Sie glättet die ungleichförmlichkeit des Motors. Die Reibringe in der
Dämpfungseinrichtung verhindern ein Aufschwingen der Torsionsfedern.

\textbf{6. Welche Fehler können ein schlechtes Trennen der Kupplung
verursachen?}

Durch Sog- und Saugwirkung kann die Kupplungsscheibe kleben. Die
Kupplungsscheibe bewegt sich schwer in der Nut der Antriebswelle. Das
Axialspiel der Kurbelwelle ist zu groß oder das Pilotlager ist defekt.

\textbf{7. Wodurch kommt bei der Magnetpulverkupplung ein Kraftschluss
zwischen Motor und Getriebe zustande?}

Durch Aufbau eines elektromagnetischen Feldes im mit der
Getriebeeingangswelle verbundenen Innenrotor wird dieser vom Eisenpulver
gefüllten, mit der Motorkurbelwelle verbundenen Außenrotor mitgenommen.
Das übertragbare Drehmoment wird durch die Magnetfeldstärke bestimmt.

\textbf{8. Beschreiben Sie die >>Kriechfunktion<< einer elektronisch
gesteuerten Kupplung.}

Sobald ein Gang eingelegt und die Bremse gelöst wird, setzt sich das
Fahrzeug im ersten und zweiten oder Rückwärtsgang langsam in Bewegung.
Erhöhung des Bedienkomforts im Stadtverkehr und beim Rangieren.

(Bemerkung: Schlupf $\to$ Reibung $\to$ Wärme $\to$ thermisch
überlastet)

\textbf{Automatisches Kupplungssystem} (AKS) Handschaltgetriebe ohne
kuppeln mit einem Pedal.

\begin{itemize}
\item
  \textbf{Stillstand mit getretener Bremse,} Kupplung wird getrennt.
\item
  \textbf{Bremse gelöst bei eingelegtem Gang}, schließt die Kupplung mit
  Schlupf und das Fahrzeug setzt sich langsam in Bewegung
  >>Kriechfunktion<<.
\item
  \textbf{Fahrpedal betätigt}, schließt die Kupplung vollständig und das
  Fahrzeug beschleunigt.
\item
  \textbf{Gangwechsel} erkennt das Fahrzeug durch Sensoren den
  Schaltwunsch und trennt die Kupplung. Nach dem Schaltvorgang wird der
  Kraftfluss wiederhergestellt.
\end{itemize}
