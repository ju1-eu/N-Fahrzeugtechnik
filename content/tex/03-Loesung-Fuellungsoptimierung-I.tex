%ju 17-Sep-22 03-Loesung-Fuellungsoptimierung-I.tex
\textbf{1) Warum werden die herkömmlichen Serienmotoren statt mit 2
häufig mit 3 oder 4 Ventilen ausgerüstet?}

Mehrventiltechnik ermöglicht eine \textbf{bessere Zylinderfüllung} durch
\textbf{Vergrößerung des Ein- und Auslassquerschnittes} und
\textbf{Verbesserung der Strömungsverhältnisse} im Zylinder. Dies wäre
bedingt auch durch größere Ein- und Auslassventile möglich. Würde aber
aufgrund der \textbf{größeren bewegten Massen} im Ventiltrieb die
\textbf{Drehzahlfestigkeit} herabsetzen.

\textbf{2) Warum rüstet man einen Dreiventilmotor mit 2 Zündkerzen und
Doppelzündung aus?}

\begin{enumerate}
\item
  Kontrollierte schneller Druckanstieg
\item
  Kondensierte Kraftstoffbestandteile an der Zylinderwand können durch
  den Verbrennungsbeginn in Zylinderwandnähe wieder vergasen und wieder
  an der Verbrennung teilnehmen.

  \begin{itemize}
  \item
    Geringere $\text{HC}$-Ausstoß
  \end{itemize}
\item
  Geringe Aufheizung des Gemisches vor der Verbrennung

  \begin{itemize}
  \item
    Geringe Klopfneigung und geringe $\text{NO}_\text{x}$-Ausstoß
  \end{itemize}
\end{enumerate}

\textbf{3) Was versteht man unter variabler Ventilsteuerung?}

Bei der variablen Ventilsteuerung werden die \textbf{Steuerzeiten} der
Einlass- und in manchen Fällen auch die der AV bedarfsgerecht \textbf{in
Abhängigkeit von Drehzahl und Last} verändert. Dies geschieht
\textbf{durch Verdrehen der Einlass- bzw. Auslass-NW}.

\textbf{4) Beschreiben Sie Aufbau und Funktion der >>Vario-Cam<< -
Nockenwellenverstellung.}

Das Vario-Cam System besteht aus einer direkt von der KW des Motors
angetriebenen Auslass-NW und einer von der Auslass-NW angetrieben
Einlass-NW.

Der \textbf{Kettenspanner} der zwischen den NW liegenden Steuerkette ist
in der Lage diese sowohl nach oben als auch nach unten zu spannen.

Spannt er die \textbf{Kette nach oben,} wird die \textbf{Einlass-NW
gegen den UZS} (Uhrzeigersinn) in die \textbf{Verstellposition spät}
gebracht.

Spannt der Kettenspanner die \textbf{Kette nach unten,} so verdreht die
\textbf{Einlass-NW im UZS} (Uhrzeigersinn) in \textbf{Verstellposition
früh}.

\textbf{5) Welchen Vorteil bietet das VTEC-System gegenüber einem
herkömmlichen Ventiltrieb?}

Beim VTEC-System kommen im unteren Drehzahlbereich \textbf{spitze} und
im oberen Drehzahlbereich \textbf{steilen Nocken} zum Einsatz.

Hierdurch wird gewährleistet, dass der Gaswechsel im Zylinder im
\textbf{unteren Drehzahlbereich} (viel Zeit) stattfinden kann,
\textbf{ohne die Beimischung von Altgas} durch zu frühes Öffnen der
Einlassventile zu riskieren.

Jedoch auch im \textbf{oberen Drehzahlbereich} (wenig Zeit) mithilfe
einer geänderten Nockenprofils mit längeren Ventilöffnungszeiten ein
\textbf{zuverlässiger Gaswechsel} gewährleistet werden kann.

\textbf{6) Wodurch erfolgt die Umschaltung zwischen den Nockenprofilen
beim Valvelift-System?}

Beim Valvelift-System wird, sobald das SG dies veranlasst, ein
\textbf{Elektromagnet bestromt,} wodurch ein \textbf{Metallstift}
ausfährt, der bei ablaufenden Nocken in eine dafür vorgesehene
\textbf{Verstellnut} einfährt und die gesamte Verstelleinheit auf der
Nockenwelle um ca. $7~mm$ verschiebt bis der \textbf{zweite Nocken}
gerade über den Rollenschlepphebel steht.

\textbf{7) Welche Aufgabe haben die Kompressions- und
Dekompressionsfedern eines elektromagnetischen Ventiltriebs?}

\begin{itemize}
\item
  \textbf{Unterstützung} des Elektromagneten \textbf{beim schnellen
  Öffnen und Schließen} des Ventils.
\item
  \textbf{Abbremsen des Ventils} kurz vor den Endstellungen geöffnet und
  geschlossen
\item
  Ventile beim abgeschalteten oder defekten Systems \textbf{in
  halbgeöffnete Stellung} bringen, um Motorschäden durch Aufsetzen der
  Ventile zu verhindern.
\end{itemize}
