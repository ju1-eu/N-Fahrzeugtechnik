%ju 26-Dez-22 15-CVT-Automatikgetriebe.tex
\textbf{Worin unterscheidet sich das stufenlose Automatikgetriebe CVT
von den herkömmlichen Automatikgetrieben?}

\begin{itemize}
\item
  Das CVT-Automatikgetriebe (Continuously Variable Transmission) hat
  keine Fahrstufen, sondern über den gesamten Fahrbereich eine
  stufenlose Übersetzung.
\end{itemize}

\textbf{Wie ist die Wirkungsweise des CVT-Automatikgetriebes?}

\begin{itemize}
\item
  Das Drehmoment wird über die Antriebswelle und den Planetenradsatz auf
  die Primärkegelscheibe und von dort über ein Schubgliederband oder
  eine Laschenkette auf die Sekundärkegelscheibe und die Abtriebswelle
  übertragen. Durch das wechselseitige Verändern der wirksamen Radien an
  Primärund Sekundärkegelscheibe wird das Übersetzungsverhältnis
  stufenlos angepasst.
\end{itemize}

\textbf{Wie wird das Übersetzungsverhältnis beim CVT-Automatikgetriebe
verändert?}

\begin{itemize}
\item
  Das Übersetzungsverhältnis lässt sich durch das gegenläufige Verändern
  der Durchmesser von Primär- und Sekundärkegelscheibe stufenlos
  bestimmen. Jeweils eine Scheibenhälfte der Primär- sowie der
  Sekundärkegelscheibe lässt sich durch hydraulischen Druck axial
  verschieben. Wird z.B. eine Primärkegelscheibenhälfte hydraulisch
  beaufschlagt und axial verschoben, so vergrößert sich dadurch der
  wirksame Radius für die Schubgliederlaufbahn. Gleichzeitig wird an der
  Sekundärkegelscheibenhälfte der hydraulische Druck verringert und der
  wirksame Radius verkleinert.
\end{itemize}

\textbf{Wann ist bei einem CVT-Automatikgetriebe das
Übersetzungsverhältnis am größten?}

\begin{itemize}
\item
  Das Übersetzungsverhältnis ist am größten, wenn an der
  Primärkegelscheibe der kleinste und an der Sekundärkegelscheibe der
  größte Durchmesser eingestellt ist.
\end{itemize}

\textbf{Worin unterscheidet sich die stufenlose Multitronic (Audi) von
der herkömmlichen stufenlosen CVT-Automatik?}

\begin{itemize}
\item
  Statt des Schubgliederbandes wird eine Laschenkette verwendet

  \begin{itemize}
  \item
    Größere übertragbare Zugkräfte (bis zu 1,7 t)
  \item
    Durch flachere Bauweise größerer Verstellbereich an den
    Kegelscheiben
  \item
    Größerer Übersetzungsfaktor zwischen der größten und der kleinsten
    Übersetzung (6,05)
  \end{itemize}
\item
  Nach dem Tiptronic -- Muster können 6 feste Gangstufen simuliert
  werden.
\item
  Das dynamische Regelprogramm der Steuerelektronik ist selbstlernend,
  es passt sich dem Fahrstil zwischen >>sportlich<< und
  >>wirtschaftlich<< an.
\end{itemize}
