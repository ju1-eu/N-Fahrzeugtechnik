%ju 26-Dez-22 19-Loesung-Bremse-II.tex
\textbf{1) Was versteht man unter Schlupf?}

Schlupf ist die Differenz zwischen der Fahrgeschwindigkeit und der
Radumfangsgeschwindigkeit.

\textbf{2) Was geschieht während der ABS -- Regelung?}

\begin{itemize}
\item
  Zunächst wird der Druckaufbau gestoppt
\item
  Wenn der Schlupf weiter zunimmt, wird der Druck geringfügig gesenkt
\item
  Wenn der Schlupf wieder abnimmt, wird der Druck erhöht, um im
  optimalen Bremsbereich zu bleiben
\end{itemize}

Bemerkung:

Das ABS-Steuergerät erhält zur Regelung des Bremsvorgangs die
Radgeschwindigkeit von den Raddrehzahlsensoren.

\begin{itemize}
\item
  sinusförmige Wechselspannung (Induktivgeber)
\item
  oder ein Rechtecksignal (Hallgeber)
\end{itemize}

Daraus wird eine Fahrzeug-Referenzgeschwindigkeit gebildet. Das
Steuergerät erkennt Drehzahländerungen an einem oder mehreren Rädern.

Sinkt die Raddrehzahl innerhalb einer Zeitspanne oder in Bezug auf die
Referenzgeschwindigkeit, wird das als Blockiergefahr erkannt.

\textbf{3) Wie viele ABS -- Regelkreise können bei einem Fahrzeug mit
hydraulischer Bremsanlage vorhanden sein? Wie sind diese ggf.
zugeordnet?}

\begin{itemize}
\item
  2 $\to$ je ein Regelkreis pro Vorderrad
\item
  3 $\to$ je ein Regelkreis für jedes Rad der Vorderachse und ein
  gemeinsamer Regelkreis für die Räder der Hinterachse.
\item
  4 $\to$ je ein Regelkreis für jedes Rad der Vorder- und Hinterachse
\end{itemize}

\textbf{4) Wie ist ein aktiver Drehzahlsensor aufgebaut?}

Der Hall-Sensor besteht im Wesentlichen aus einem stromdurchflossen,
Halbleiterbauelement und einem magnetischen, in Nord- und Südpole
unterteilten Impulsring. Ändert sich die Polarität des, auf das
Hall-Element wirkenden Magnetfeldes, ändert sich die Hallspannung
innerhalb des Halbleiters, woraus die integrierte Auswerteelektronik ein
Rechtecksignal formt und dieses an das Steuergerät sendet. Die Frequenz
des Rechteckssignals ist proportional zur Raddrehzahl.

\textbf{5) Warum lässt sich ein Fahrzeug bei Ausfall des ABS noch normal
bremsen?}

In der Ruhelage sind die Magnetventile in der Hydraulikeinheit
einlassseitig geöffnet und auslassseitig geschlossen.

\textbf{6) Wie ist die grundsätzliche Wirkungsweise der
Antriebsschlupfregelung (ASR)?}

Wird an einem der Antriebsräder ein erhöhter Schlupf (> 8
-- 35 \%) erkannt, so wird systemabhängig das Motordrehmoment reduziert
und/oder die durchdrehenden Rad/Räder mithilfe der Betriebsbremse
abgebremst.

\textbf{7) Was wird durch die Motorschleppmomentregelung (MSR)
verhindert?}

Verhindert durch automatische Erhöhung des Motormoments, dass die
Antriebsräder aufgrund der Abbremsung durch den Motor bei plötzlicher
Gaswegnahme oder beim Zurückschalten einen erhöhten Schlupf aufweisen.

\textbf{8) Was versteht man unter dem Begriff >>Giermoment<<?}

\textbf{Giermoment / Drehrate} Drehung des Fahrzeugs um die
Fahrzeughochachse

\textbf{9) Welche Systeme bilden mit ihren Funktionen das eigentliche
elektronische Stabilitätsprogramm (ESP)?}

\begin{enumerate}
\item
  (+) Antiblockiersystem (ABS)
\item
  (+) Elektronische Bremskraftverteilung (EBV)
\item
  (+) Antriebsschlupfregelung (ASR)
\item
  (+) Motorschleppmomentregelung (MSR)
\item
  (+) Giermomentregelung (GMR)
\item
  (=) Fahrdynamikregelsystem (FDR) = Elektronisches Stabilitätsprogramm
  (ESP)
\end{enumerate}

\textbf{10) Wie reagiert das ESP -- Steuergerät beim Übersteuern eines
Fahrzeugs mit Standardantrieb?}

\textbf{ESP-Eingriff beim Übersteuern:}

\begin{itemize}
\item
  \emph{ohne ESP} würde beim Fahrzeug das Heck ausbrechen und der Fahrer
  muss gegenlenken.
\item
  \emph{mit ESP} unterstützt den Fahrer durch einen Bremseingriff,
  vorwiegend am kurvenäußeren Vorderrad. Dadurch entsteht ein
  Giermoment/Drehrate um die Hochachse, das das Auto wieder auf den
  gewünschten Fahrkurs zieht.
\item
  Das Antriebsmoment der Hinterräder gesenkt und deren
  Seitenführungskräfte zu erhöhen
\end{itemize}

Bemerkung:

\textbf{ESP-Eingriff beim Untersteuern:}

\begin{itemize}
\item
  \emph{ohne ESP} würde das Fahrzeug über die Vorderräder aus der Kurve
  schieben.
\item
  \emph{mit ESP} unterstützt die Lenkkorrektur des Fahrers durch einen
  Bremseingriff, vorwiegend am kurveninneren Hinterrad. Dadurch entsteht
  ein Giermoment/Drehrate um die Hochachse, das das Auto in die Kurve
  hineindreht.
\end{itemize}
