%ju 31-Dez-22 17-Wechselgetriebe.tex
\textbf{Nennen Sie die Hauptaufgaben des Wechselgetriebes.}

\begin{itemize}
\item
  Drehmomente und Drehzahlen wandeln
\item
  Drehrichtungsänderung ermöglichen
\item
  Leerlauf ermöglichen
\end{itemize}

\textbf{Nennen Sie Merkmale eines gleichachsigen Getriebes.}

\begin{itemize}
\item
  Der Kraftausgang liegt in der gleichen Achse, wie der Krafteingang
\item
  Im direkten Gang ist die Eingangsdrehzahl gleich der Ausgangsdrehzahl
\item
  Das Getriebe hat 3 Wellen (Antriebs-, Vorgelege- und Hauptwelle)
\item
  Alle Gänge außer dem direkten und dem Rückwärtsgang benötigen 2
  Zahnradpaarungen
\item
  Alle Schaltvorgänge erfolgen auf der Hauptwelle
\item
  Die Drehrichtung von Eingangs- und Hauptwelle sind bei Vorwärtsfahrt
  gleich
\end{itemize}

\textbf{Warum sind Getriebezahnräder i.d.R. schrägverzahnt?}

Durch die Schrägverzahnung laufen die Zahnräder leise, da sich die Zähne
ineinander schieben.

\emph{Nachteil}

\begin{itemize}
\item
  Es treten axiale Kräfte auf
\end{itemize}

\textbf{Was versteht man unter der Bezeichnung >>Aphongetriebe<<?}

\begin{itemize}
\item
  Ein Getriebe, bei dem sämtliche Zahnradpaarungen mit Ausnahme des
  Rückwärtsgangs ständig miteinander im Eingriff und schrägverzahnt
  sind. Hierdurch läuft das Getriebe besonders leise.

  \begin{itemize}
  \item
    Aphon:

    \begin{itemize}
    \item
      A = Anti
    \item
      Phon = Laut, Schall
    \end{itemize}
  \end{itemize}
\end{itemize}

\textbf{Was bedeuten die Getriebeübersetzungen 3,31 und 0,78?}

\begin{itemize}
\item
  Diese Angaben geben das Übersetzungsverhältnis zwischen
  Kurbelwellendrehzahl und der Drehzahl des Getriebeausganges an.

  \begin{itemize}
  \item
    3,31:

    \begin{itemize}
    \item
      Kurbelwelle: 3,31 1/min
    \item
      Getriebeausgang: 1 1/min
    \end{itemize}
  \item
    0,78:

    \begin{itemize}
    \item
      Kurbelwelle: 0,78 1/min
    \item
      Getriebeausgang: 1 1/min
    \end{itemize}
  \end{itemize}
\end{itemize}

\textbf{Was versteht man unter einem sperrsynchronisierten Getriebe?}

\begin{itemize}
\item
  Es ist ein Getriebe, bei dem das Einlegen eines Ganges solange
  gesperrt wird, bis das Gangrad (Losrad) und die Getriebewelle auf
  gleiche Drehzahl (Gleichlauf) gebracht sind. Erst dann können die
  Gänge geschaltet werden.
\end{itemize}

\textbf{Bei einem Fahrzeug mit ungleichachsigem Getriebe läuft der Motor
im Leerlauf. Welche Teile im Getriebe drehen sich?}

\begin{itemize}
\item
  Die Antriebswelle mit den Synchronkörpern
\item
  Die Schaltmuffen und die Festräder der Antriebswelle
\item
  Die Losräder der Abtriebswelle
\end{itemize}
