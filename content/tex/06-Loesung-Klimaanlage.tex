%ju 31-Dez-22 06-Loesung-Klimaanlage.tex
ANTWORT: Vgl. Fotos 06-L-Klimaanlage-1 bis 3 -Limburg.jpg

\textbf{1. Welche Aufgaben hat eine Klimaanlage im Kfz?}

\textbf{2. Wer ist nach der UVV Sachkundiger im Umgang mit
Klimaanlagen?}

\textbf{3. Beschreiben Sie den Kältemittelkreislauf in einer
Kfz-Klimaanlage.}

\textbf{4. Wie werden in Kfz-Klimaanlagen die Druckbereiche unterteilt
und welche Drücke herrschen in den einzelnen Bereichen vor?}

\textbf{5. Auf welche Weise wird der Klima-Kompressor geschmiert?}

\textbf{6. Welche Aufgaben muss der Flüssigkeitsbehälter mit
Trocknereinsatz in einer Kfz- Klimaanlage übernehmen?}

\textbf{7. Welche Schutzausrüstung muss bei Arbeiten an der
Kfz-Klimaanlage getragen werden?}

\textbf{8. An welchen Orten dürfen Kältemittel gelagert werden?}

\textbf{9. Welche Angaben müssen an jeder Klimaanlage deutlich erkennbar
und dauerhaft angebracht sein?}

\textbf{10. Warum muss eine Kfz-Klimaanlage in gewissen Zeitabständen
gewartet werden?}
