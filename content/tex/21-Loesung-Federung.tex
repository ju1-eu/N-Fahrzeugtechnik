%ju 26-Dez-22 21-Loesung-Federung.tex
\textbf{1. Welche Aufgaben muss die Federung im Kfz übernehmen?}

\begin{itemize}
\item
  Sie soll die Stöße, die durch Fahrbahnunebenheiten auf die Karosserie
  einwirken, in Schwingungen umwandeln
\end{itemize}

Dies verbessert

\begin{itemize}
\item
  den Fahrkomfort für die Insassen
\item
  den Schutz des Ladegutes
\item
  die Bodenhaftung zur Erhöhung der Fahrsicherheit
\end{itemize}

\textbf{2. Welche Vorteile haben Schraubenfedern gegenüber Blattfedern?}

\begin{itemize}
\item
  Geringes Gewicht
\item
  Geringerer Bauraumbedarf
\item
  Reibungsarm

  \begin{itemize}
  \item
    Geringe Eigendämpfung

    \begin{itemize}
    \item
      Dämpfungsrate wird überwiegend durch Schwingungsdämpfung bestimmt
      und ist somit besser einstellbar
    \end{itemize}
  \item
    Wartungsfrei
  \end{itemize}
\item
  Auslegung linear und progressiv möglich
\item
  Kopplung mit der >>Active Body Control<< möglich
\end{itemize}

\textbf{3. Wie erreicht man bei Schraubenfedern eine progressive
Kennlinie?}

\begin{itemize}
\item
  Eine progressive Kennlinie, d.h. eine zunehmende Federhärte mit
  zunehmendem Federweg, erreicht man bei der Schraubenfeder durch
  unterschiedliche Windungsdurchmesser (taillierte Feder, Tonnenfeder,
  Minibloc-Feder) oder durch unterschiedliche Drahtdurchmesser in den
  einzelnen Windungen
\end{itemize}

\textbf{4. Welche Vorteile bietet die Luftfederung?}

\begin{itemize}
\item
  Gleichbleibend hoher Federungskomfort
\item
  Fahrzeugniveau unabhängig von Beladungszustand
\item
  Spur und Sturz bleiben unverändert
\item
  Fahrzeugniveau situationsabhängig adaptierbar

  \begin{itemize}
  \item
    Senkung bei Hochgeschwindigkeit
  \item
    Anhebung bei Schlechtwegfahrten
  \item
    Höhenanpassung bei Einstieg und Beladung
  \item
    Dynamischer Nick-- und Wankausgleich
  \end{itemize}
\item
  Keine >>trampelnden<< Achsen bei Leerfahrten
\item
  Kopplung mit Druckluftbremse möglich
\end{itemize}

\textbf{5. Wie ist die grundsätzliche Wirkungsweise der
hydropneumatischen Federung?}

\begin{itemize}
\item
  Beim Einfedern wird die Bewegung der Räder über einen Kolben auf ein
  Ölvolumen übertragen, welches sich wiederum auf ein Gaspolster
  abstützt.
\end{itemize}

\textbf{6. Welche Aufgaben erfüllen Stoßdämpfer am Kfz?}

\begin{itemize}
\item
  Die Stoßdämpfer wandeln die Schwingungen der Federung durch Reibung in
  Wärme um, um die Anzahl der Amplituden nach einem Fahrbahnstoß zu
  reduzieren

  \begin{itemize}
  \item
    Verbesserung der Fahrsicherheit durch gleichbleibend guten
    Bodenkontakt
  \item
    Erhöhung des Fahrtkomforts
  \end{itemize}
\end{itemize}

\textbf{7. Welcher Unterschied besteht zwischen}

\begin{itemize}
\item
  \textbf{a) dem Zweirohr-Stoßdämpfer und}
\item
  \textbf{b) dem Einrohr-Gasdruck-Stoßdämpfer?}
\end{itemize}

\begin{enumerate}
\def\labelenumi{\alph{enumi})}
\item
  Der Zweirohr-Stoßdämpfer besitzt zur Aufnahme des
  Kolbenstangenvolumens bei Einfedern einen Vorratsraum, der rund um den
  eigentlichen Arbeitsraum angeordnet ist. Dieser wirkt isolieren,
  wodurch die, im Arbeitsraum entstehende Wärmemenge nur schlecht
  abgeführt werden kann. Hierdurch hat der Dämpfer eine hohe Tendenz zur
  Ölverschäumung, die durch hochviskoses Öl unterbunden wird, was den
  Dämpfer jedoch im kalten Zustand hart und unkomfortabel macht.
\item
  Beim Einrohr-Gasdruck-Stoßdämpfer wird das Kolbenstangenvolumen durch
  ein Gaspolster aufgenommen. Der Vorratsraum entfällt, die Wärmeabfuhr
  verbessert sich und die Verwendung niedrigviskosen Öls ermöglicht eine
  annähernd temperaturunabhängige Dämpfungsrate.
\end{enumerate}

\textbf{8. Welcher Unterschied besteht bei einem Stoßdämpfer zwischen}

\begin{itemize}
\item
  \textbf{a) der Druckstufe und}
\item
  \textbf{b) der Zugstufe?}
\end{itemize}

\begin{enumerate}
\def\labelenumi{\alph{enumi})}
\item
  Die Druckstufe bezeichnet die Dämpfungsrate beim Einfedern
\item
  Die Zugstufe bezeichnet die Dämpfungsrate beim Ausfedern und ist in
  etwa doppelt so hoch wie die Druckstufe.
\end{enumerate}

\textbf{9. Wie unterscheidet sich ein >>passives<< von einem >>aktiven<<
Fahrwerk?}

\begin{itemize}
\item
  Ein passives Fahrwerk kann ausschließlich auf Karosseriebewegungen
  reagieren, wenn diese bereits vorhanden sind.
\item
  Ein aktives Fahrwerk kann bestimmte Karosseriebewegungen, wie Nicken,
  beim Beschleunigen und Bremsen, sowie Wanken bei Kurvenfahrt
  vorausberechnen und die Nick-- bzw. Wankneigung durch eine optimierte
  Fahrwerkskonfiguration schon vor deren Auftreten minimieren.
\end{itemize}

\textbf{10. Erläutern Sie die Begriffe:}

\begin{itemize}
\item
  \textbf{a) Hydropneumatische Federung}
\item
  \textbf{b) Aktive Fahrwerksstabilisierung}
\item
  \textbf{c) Active Body Control} (ABC)
\item
  \textbf{d) Semi-aktive Luftfederung}
\end{itemize}

\begin{enumerate}
\def\labelenumi{\alph{enumi})}
\item
  Die hydropneumatische Federung verwendet als Federelement ein
  Gaspolster.
\item
  Bei der aktiven Fahrwerkstabilisierung kann die Härte des
  Stabilisators dynamisch verändert werden, wodurch Nick-- und
  Wankneigung minimiert werden.
\item
  Bei einem Fahrzeug mit ABC--Fahrwerk kann das Niveau jedes Rades
  separat geregelt werden, wodurch Fahrzeugbewegungen deutlich reduziert
  werden.
\item
  Die semi-aktive Luftfederung ist eine Kombination aus der
  ABC--Fahrwerk und Luftfederung.
\end{enumerate}
