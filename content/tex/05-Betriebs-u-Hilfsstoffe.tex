%ju 26-Dez-22 05-Betriebs-u-Hilfsstoffe.tex
\section{Was sind Betriebsstoffe?}\label{was-sind-betriebsstoffe}

Sind Stoffe, die zum Betrieb des Kraftfahrzeuges nötig sind.

\textbf{Beispiele:} Kraftstoffe, Motoröl, Bremsflüssigkeit

\section{Was sind Hilfsstoffe?}\label{was-sind-hilfsstoffe}

Sind alle Stoffe, die zum Warten, Reinigen und Pflegen von Fahrzeugen
notwendig sind.

\textbf{Beispiele:} Politur, Bremsenreiniger, Scheibenreiniger

\textbf{Scheibenwaschwasserzusatz}

\begin{itemize}
\item
  \emph{Sommer} mit Enzymen (Insektenreste besser entfernen)
\item
  \emph{Winter} mit Gefrierschutz
\end{itemize}

\section{Woraus bestehen
Kraftstoffe?}\label{woraus-bestehen-kraftstoffe}

\textbf{Kraftstoffe} sind hauptsächlich Kohlen - Wasserstoff -
Verbindungen (geringer Anteil Schwefel). Die Anzahl der Atome und deren
Verbindungen bestimmen die Art des Kraftstoffes. Zur Verbesserung der
Eigenschaften werden Ihnen Additive zugefügt.

\subsection{Was unterscheidet Otto- vom
Dieselkraftstoff?}\label{was-unterscheidet-otto--vom-dieselkraftstoff}

\begin{itemize}
\item
  die Struktur der Verbindungen
\item
  Größe der Moleküle
\item
  zahlenmäßige Verhältnisse der Atome
\end{itemize}

Benzin: ringförmiger Molekülaufbau, Oktanzahl $\to$ Zündunwillig

Diesel: kettenförmiger Molekülaufbau, Cetanzahl $\to$ Zündwilligkeit

\subsection{Aufbau der
Kohlenwasserstoffmoleküle}\label{aufbau-der-kohlenwasserstoffmolekuele}

\begin{itemize}
\item
  \textbf{Paraffine} kettenförmiger Aufbau, wenig klopffest,

  \begin{itemize}
  \item
    flüssig -- bestandteile des Benzins und Dieselkraftstoffes,
    (Beispiel: Oktan, Cetan)
  \end{itemize}
\item
  \textbf{Isoparaffine} verzweigter kettenförmiger Aufbau, sehr
  klopffest

  \begin{itemize}
  \item
    Bestandteil des Eichkraftstoffes für Ottokraftstoffe, (Beispiel:
    Isooktan)
  \end{itemize}
\item
  \textbf{Aromaten} ringförmiger Aufbau, sehr klopffest, häufig mit
  Doppelbindung, (Beispiel: Benzol)
\end{itemize}

\section{Wirkungsgrad eines
Verbrennungsmotors}\label{wirkungsgrad-eines-verbrennungsmotors}

\begin{enumerate}
\item
  Dieselmotoren ca. $46~\%$ und
\item
  Ottomotoren ca. $35~\% \to$ werden in \textbf{Bewegungsenergie} als
  Antriebsenergie für Motor verwendet
\end{enumerate}

Rest in \textbf{Reibung und Wärme}

\textbf{Warum ist ein Dieselmotor effizienter als ein Ottomotor?}

\begin{itemize}
\item
  Energiedichte des Kraftstoff ist höher
\item
  Wirkungsgrad höher gegenüber Ottomotor
\item
  Wärmeabführung geringer
\item
  höherer Verdichtungsdruck, höherer Expansionsgrad, höhere Effizienz
\item
  keine Drosselverluste, weil keine Drosselklappe
\end{itemize}

\section{Herstellung von
Kraftstoffen}\label{herstellung-von-kraftstoffen}

\textbf{Wo kommen die Kraftstoffe her?}

\begin{enumerate}
\item
  \textbf{Erdöl} aus ca. 80 \% Kohlenstoff und 12 \% Wasserstoff, ca.
  1--3 \% Schwefel
\item
  \textbf{E-Fuels} Kraftstoffe aus dem $CO_2$ der Luft, klimaneutral
  \footnote{\url{https://www.youtube.com/watch?v=qq0fjl0LQXo}}

  \begin{itemize}
  \item
    Stromerzeuger: Windrad oder Solarenergie
  \item
    Offshore-Windparks sind Windparks, die im Küstenvorfeld der Meere
    errichtet werden.

    \begin{itemize}
    \item
      haben keine Speicher, Wechselspannung kann nicht gespeichert
      werden
    \end{itemize}
  \end{itemize}
\end{enumerate}

\subsection{Trennverfahren}\label{trennverfahren}

\begin{enumerate}
\item
  \textbf{Filtern} Verunreinigungen werden aus dem Rohöl entfernt
\item
  \textbf{Destillieren} Trennen, \emph{atmosphärische Destillation}
  (Druck bei $1013~mbar$) und \emph{Vakuum Destillation} (bei
  Unterdruck, um Siedepunkt herabzusetzen)
\item
  \textbf{Raffinieren} Nachbehandeln, Reinigen
\end{enumerate}

\subsection{Fraktionierende
Destillation}\label{fraktionierende-destillation}

darunter versteht man das Aufteilen von Rohöl nach Siedebereichen.
Hierzu wird das Rohöl erhitzt und in eine Kolonne geleitet, wo die
einzelnen Bestandteile kondensieren und über Glockenböden abgeführt
werden.

\emph{Gase die dabei entstehen sind:}

\begin{enumerate}
\item
  Propan
\item
  Butan
\item
  Methan
\end{enumerate}

\emph{Die anfallenden Produkte sind}

\begin{enumerate}
\item
  Leicht- und Schwerbenzin
\item
  Petroleum
\item
  Diesel
\item
  Gas- und Spindelöle
\item
  Mineralische Motoröle
\item
  Zylinderöl
\item
  Bitumen
\end{enumerate}

\subsection{Umwandlungsverfahren}\label{umwandlungsverfahren}

\begin{enumerate}
\item
  \textbf{Cracken} nennt man das Spalten von schwer siedende
  (langkettige) Kohlenwasserstoffmolekülen in leicht siedende
  (kurzkettige) Kohlenwasserstoffmoleküle, wodurch die Klopffestigkeit
  eines Kraftstoffs erhöht wird. Langkettige Kohlenwasserstoffmoleküle
  sind >>schwer siedend und reaktionsfreudig<<. Kurzkettige
  Kohlenwasserstoffmoleküle sind >>leicht siedend und reaktionsträge<<.

  \begin{itemize}
  \item
    \emph{Crackarten:} thermisches Cracken, katalytisches Cracken und
    Hydrocracken
  \end{itemize}
\item
  \textbf{Reformieren} Kettenförmige Paraffine aus der Destillation
  werden mit Katalysatoren (Platin) in klopffeste Isoparaffine und
  Aromate umgewandelt
\item
  \textbf{Polymerisieren}, die beim Cracken und Reformieren entstandenen
  gasförmigen Kohlenwasserstoffe werden über Katalysatoren zu größeren
  Molekülen zusammengeballt, hauptsächlich zu Isoparaffinen
\end{enumerate}

\textbf{Katalysator} ist ein Stoff, der die Reaktionsgeschwindigkeit
einer chemischen Reaktion beeinflusst, ohne dabei selbst verbraucht zu
werden.

\emph{Bemerkung:} Katalysator \textbf{altern} vs.~Beispiel Bremsbelag
\textbf{verschleißen} (Reibung)

\textbf{Schwefel} ist giftig und hat eine schmierende Wirkung $\to$
\emph{Ziel} Kraftstoff entschwefeln (Ersatzstoff gesucht) und
\textbf{Harz} betrifft Oldtimer, wo die Einspritzung verharzen kann.

\section{Ottokraftstoffe -- leicht siedende
Kraftstoffe}\label{ottokraftstoffe-leicht-siedende-kraftstoffe}

\emph{Bemerkung:}

\begin{itemize}
\item
  \textbf{Flammpunkt} beschreibt den Punkt oberhalb einer Flüssigkeit um
  eine Flamme entstehen lassen zu können durch eine fremde Zündquelle,
  d.h. Beginn des Ausgasens eines Kraftstoffes um über ihn ein
  zündfähiges Gemisch bilden zu können, was sich durch eine externe
  Zündquelle entzünden kann vs.~
\item
  \textbf{Selbstentzündungstemperatur} entzündet sich selbst
  (Dieselkraftstoff)
\item
  \textbf{Siedepunkt} Übergang vom flüssigen in den gasförmigen Zustand
\item
  \textbf{Klopffestigkeit} geringe Neigung eines Kraftstoffes, sich
  unter hohen Temperaturen und Drücken selbst zu entzünden
\item
  \textbf{Oktanzahl} Klopffestigkeit des Kraftstoffes

  \begin{itemize}
  \item
    >>Je klopffester der Kraftstoff ist, umso höher kann er eine
    thermische Belastung aushalten, ohne sich selbst zu entzünden.<<
  \end{itemize}
\item
  \textbf{Cetanzahl} Zündwilligkeit von Dieselkraftstoff. (Wie stark ein
  Kraftstoff zur Selbstzündung neigt)
\item
  \textbf{Zündverzug} $\frac{1}{1000}~s$ (eines intakten Motors ohne
  Verbrennungsstörung)
\end{itemize}

\subsection{Anforderungen an Ottokraftstoff
(Prüfung)}\label{anforderungen-an-ottokraftstoff-pruefung}

\begin{itemize}
\item
  leicht und vollständig vergasen, leicht siedend
\item
  hohe Klopffestigkeit
\item
  geringe Neigung zur Dampfblasenbildung
\item
  Korrosionsschutz Eigenschaft
\item
  hohe Alterungsbeständigkeit
\item
  geringe Belastung mit Emission fördernden Stoffen
\end{itemize}

\textbf{Flammpunkt} unter $<-35~^\circ\text{C}$

\textbf{Siedebereich} zwischen
$30^\circ\text{C} \text{ und } 215^\circ\text{C}$

\textbf{Kaltstartverhalten} damit ein kalter Motor bei niedrigen
Temperaturen sicher anspringt, benötigt er einen Kraftstoff mit
niedriger Siedekurve.

\begin{itemize}
\item
  \textbf{E70-Punkt} verdampfter Anteil bei $70~^\circ\text{C}$
\item
  \textbf{T10-Punkt} Temperatur, bei dem $10~\%$ des Kraftstoffs
  verdampft sind
\end{itemize}

\textbf{Heißstartverhalten} bei einem heißen Motor (sowie im Sommer)
besteht die Gefahr der Dampfblasenbildung im Kraftstoffsystem. (zu viel
Luft) Beispiel: K-Jetronic

\begin{itemize}
\item
  \textbf{E180-Punkt} verdampfter Anteil bei $180~^\circ\text{C}$
\item
  \textbf{T90-Punkt} Temperatur, bei dem $90~\%$ des Kraftstoffs
  verdampft sind
\end{itemize}

\subsection{ROZ und MOZ}\label{roz-und-moz}

Maß für die Klopffestigkeit ($\to$ wie stark ein Kraftstoff zur
Selbstzündung neigt)

\begin{enumerate}
\item
  ROZ (Research-Oktanzahl)
\item
  MOZ (Motor-Oktanzahl) $\to$ wird unter anderen Prüfbedingungen
  ermittelt
\end{enumerate}

\textbf{Was gibt die Oktanzahl an?}

wie viel Vol.-\% Iso-Oktan sich in einem Bezugskraftstoff befinden

\textbf{Oktanzahl bestimmen}

Beispiel: \textbf{Super 95} (ROZ 95
$\to 95~\% \text{ Isooktan und Normalheptan } 5~\%$)

Wird in einem Prüfmotor mit variablem Verdichtungsverhältnis ermittelt,
in dem der Kraftstoff mit einem Referenzkraftstoff aus Normalheptan (ROZ
= 0, klopffreudig) und Isooktan (ROZ = 100, klopffest) verglichen wird.

\subsection{Arten von Klopfbremsen}\label{arten-von-klopfbremsen}

Maßnahmen, um die Klopffestigkeit zu erhöhen

\begin{enumerate}
\item
  metallhaltig, sind verboten (verbleites Benzin)
\item
  metallfreien Klopfbremsen wie Benzol, sind sehr effektiv, aber auch
  stark krebserregend, daher begrenzt auf 1 Vol.-\%
\item
  organischen Sauerstoff-Verbindungen wie Alkohole (Ethanol)
\end{enumerate}

\section{Dieselkraftstoff -- schwer siedende
Kraftstoffe}\label{dieselkraftstoff-schwer-siedende-kraftstoffe}

\subsection{Anforderung an
Dieselkraftstoff}\label{anforderung-an-dieselkraftstoff}

\begin{itemize}
\item
  hohe Zündwilligkeit
\item
  gute Korrosionsschutz Eigenschaft
\item
  hohe Alterungsbeständigkeit
\item
  geringe Belastung mit Emission fördernden Stoffen
\item
  gute Schmiereigenschaft
\end{itemize}

\textbf{Flammpunkt} über $>55~^\circ\text{C}$

\textbf{Siedebereich} zwischen
$170^\circ\text{C} \text{ und } 380^\circ\text{C}$

\textbf{Selbstzündungstemperatur} (untere Grenze) $220^\circ\text{C}$
(im Mittel) bei ca. $350^\circ\text{C}$ Quelle: Bosch S. 562
(\textcite{reif:2022:boschkraftfahrtechnisches}).

\subsection{Additivierung / Additive und
Auswirkung}\label{additivierung-additive-und-auswirkung}

\begin{enumerate}
\item
  \textbf{Fließverbesserer} (kältefest, filtergängig)
\item
  \textbf{Schmierfähigkeit} (Schwefelersatz)
\item
  \textbf{Biozide} sollen das Bakterienwachstum im Dieselkraftstoff
  verhindern. Diese würden zu folgenden Problemen im Einspritzsystem
  führen:

  \begin{itemize}
  \item
    Verstopfen der Filtersysteme durch die lebenden oder abgestorbenen
    Bakterien
  \item
    Erosionsschäden durch die mit hoher Geschwindigkeit und hohem Druck
    durch das Einspritzsystem geförderten Bakterien (ähnlich
    Sandstrahleneffekt)
  \item
    Korrosionsschäden durch die Bakterien Exkremente ($H_2SO_3$ und
    $H_2SO_4$)
  \end{itemize}
\item
  \textbf{Zündbeschleuniger} (Cetanzahl erhöhen, Verringerung des
  Zündverzugs, schnelleres Eintreten des Verbrennungsprozesses)
\end{enumerate}

\textbf{Cetanzahl} ist ein Maß für die Zündwilligkeit von
Dieselkraftstoff.

\textbf{CFPP} (Cold Filter Plugging Point,
Kalter-Filter-Verstopfungs-Punkt) gibt die Temperatur an, dass den
Filter für Kraftstoff nicht mehr durchfließen kann.

\textbf{Winterdiesel} ist Dieselkraftstoff, der einen geringeren CFPP
aufweist. Dieselkraftstoff enthält Paraffin, das bei geringen
Temperaturen kristallisiert. Die entstandenen Kristalle setzen sich in
den Kraftstofffilter und verstopfen diesen.

\textbf{Schwefel} hat Schmierwirkung (Ausgleich durch Additive)

\begin{itemize}
\item
  beim Verbrennen: von Schwefel und Wasser $\to$ \textbf{Folge} Säure
  / Gefahr von Übersäuren (vgl. Eigenschaften von Biodiesel)
\item
  für NOx-Speicherkatalysator sollte Schwefelanteil gering sein
\end{itemize}

\section{Bioethanol (Ottomotoren)}\label{bioethanol-ottomotoren}

\textbf{Benzin}

\begin{enumerate}
\item
  \textbf{Super E5} ist ein Gemisch aus $95~\%$ Benzin und bis zu
  $5~\%$ Bioethanol
\item
  \textbf{E10} ist ein Gemisch aus $90~\%$ Benzin und bis zu $10~\%$
  Bioethanol
\item
  \textbf{E85} ist ein Gemisch aus $15~\%$ Benzin und bis zu $85~\%$
  Bioethanol $\to$ in Diskussion (vgl. Ethanol)
\end{enumerate}

Es wurde eingeführt, um den Anteil regenerativer Energiequellen am
Kraftstoff zu erhöhen.

\textbf{Ethanol}

\begin{itemize}
\item
  \emph{Nachteile:} geringer Heizwert (hoher Verbrauch, geringe
  Reichweite)
\item
  und wirkt als Lösungsmittel und kann Dichtungen angreifen
\item
  \emph{Vorteil:} Klopffest, steigert die Oktanzahl (ROZ104)
\item
  \emph{Eigenschaften:} ungiftig, regenerativ, korrosiv, hygroskopisch,
  bei Raumtemperatur flüssiger Alkohol
\end{itemize}

\section{Biodiesel -- Fatty Acid Methyl Esther (kurz:
FAME)}\label{biodiesel-fatty-acid-methyl-esther-kurz-fame}

entsteht, indem ölhaltige Erzeugnisse, wie Raps mithilfe von Ethanol
oder Methanol nachbehandelt werden. Dieser Vorgang wird als >>umestern<<
bezeichnet. Biodiesel kommt entweder als Reinkraftstoff oder als bis zu
7\%ige Beimischung zum Dieselkraftstoff (sog. Petroldiesel) zum Einsatz.

\subsection{Eigenschaften von Biodiesel und Folgen für den Einsatz im
Verbrennungsmotor}\label{eigenschaften-von-biodiesel-und-folgen-fuer-den-einsatz-im-verbrennungsmotor}

\begin{enumerate}
\item
  \textbf{Umweltfreundlich} (hängt stark von der Umsetzung ab, Einsatz
  fossiler Energieträger reduziert und den Ausstoß von Treibhausgasen
  mindert.)
\item
  \textbf{Korrosiv} (kann zur Zersetzung führen, Beispiel: Dichtungen
  und Schläuchen)
\item
  \textbf{Reinigend} (kann Filtersysteme oder Kraftstoff führende
  Bauteile verstopfen)
\item
  \textbf{Hygroskopisch} (zieht aufgrund seines Alkoholanteils Wasser
  an) \textbf{erhöhter Wasseranteil kann zu folgenden Erscheinungen
  führen}

  \begin{itemize}
  \item
    Heraufsetzung des Kalter-Filter-Verstopfungs-Punkt (\textbf{CFPP})

    \begin{itemize}
    \item
      Die Wasserbestandteile stocken wesentlich früher aus, was bei
      Temperaturen unter $0~^\circ\text{C}$ zum Verstopfen des
      Kraftstofffilters führen kann.
    \end{itemize}
  \item
    Übersäuerung des Kraftstoffs

    \begin{itemize}
    \item
      pH-Wert kann sinken, dass Korrosionsschutzschichten angegriffen
      werden.
    \end{itemize}
  \item
    Förderung des Wachstums von Bakterien (Vgl. \textbf{Dieselpest})

    \begin{itemize}
    \item
      Verstopfung von Filter durch Bakterienkulturen
    \item
      Erosive Schädigung des Einspritzsystems: Die Mikroorganismen
      werden mit hoher Geschwindigkeit durch das Einspritzsystem
      gefördert und tragen dabei oberflächlich Material ab. Das kann zu
      Undichtigkeiten führen (Beispiel: Dichtsitz des Injektors).
    \end{itemize}
  \item
    Kavitation

    \begin{itemize}
    \item
      Durch den Abfall des Siedepunkts (Diesel vs.~Wasser) kann es zu
      Folgeerscheinungen (Beispiel: Druckabfall, Undichtigkeit) kommen.
    \end{itemize}
  \end{itemize}
\item
  \textbf{Hoher Flammpunkt} (Biodiesel vs.~Diesel)

  \begin{itemize}
  \item
    Während Dieselkraftstoff bei betriebswarmen Motor zumindest
    teilweise verdampft und über die Kurbelgehäuseentlüftung abgeführt
    wird, bleibt der Biodiesel nahezu vollständig im Motoröl enthalten.
    Dies führt zu Ölverdünnung und Überfüllung.
  \end{itemize}
\item
  \textbf{Geringer Energiegehalt} (Leistungsrückgang bzw. ein
  Mehrverbrauch)
\item
  \textbf{Biologisch abbaubar}
\item
  \textbf{Nahezu schwefelfrei}

  \begin{itemize}
  \item
    Vorteil, wenn Fahrzeug über NOx-Speicherkatalysator verfügt.
  \end{itemize}
\end{enumerate}

\textbf{Erosion} feine Partikel in der Luft oder in Flüssigkeiten tragen
Material von der Oberfläche ab (durch Reibung oder Schleifen).

\section{Gasförmige Kraftstoffe (Motoren mit
Fremdzündung)}\label{gasfoermige-kraftstoffe-motoren-mit-fremdzuendung}

\subsection{Autogas (LPG)}\label{autogas-lpg}

Flüssiggas \textbf{LPG} (Liquefied Petroleum Gas)

\begin{itemize}
\item
  Gemisch aus Propan und Butan
\item
  Speicherung: \emph{flüssig} bei niedrigem Druck ca. 2 -- 10 bar
\item
  \textbf{Sommermischung} $60~\%$ Butan und $40~\%$ Propan
\item
  \textbf{Wintermischung} $40~\%$ Butan und $60~\%$ Propan
\end{itemize}

\subsection{Erdgas (CNG, LNG)}\label{erdgas-cng-lng}

\begin{itemize}
\item
  Gasgemisch, Hauptbestandteil ist \textbf{Methan}
\item
  Speicherung: \textbf{CNG} (Compressed Natural Gas, komprimiertes Gas)
  \emph{gasförmig} bei Umgebungstemperatur und 200 bar
\item
  Speicherung: \textbf{LNG} (Liquefied Natural Gas) \emph{flüssig} bei
  $- 160~^\circ\text{C}$ und 2 bar
\end{itemize}

\subsection{Wasserstoff}\label{wasserstoff}

ideale Kraftstoff (unbegrenzte Verfügbarkeit, Energiegehalt,
Verbindungseigenschaften)

\textbf{Wie wird Wasserstoff gewonnen?} Wasserstoff wird durch
Elektrolyse gewonnen. Dabei wird mithilfe der elektrischen Energie
Wasser in Wasserstoff ($H_2$) und Sauerstoff ($O_2$) zerlegt.

\textbf{Brennstoffzellen} (kalte Verbrennung) sind elektrochemische
Zellen, mit denen die chemische Energie eines geeigneten Brennstoffs
(Methanol) mit Sauerstoff ($O_2$) aus der Luft ununterbrochen in
elektrische Energie umgewandelt werden kann.

\section{Bremsflüssigkeit}\label{bremsfluessigkeit}

\textbf{Eigenschaften von DOT4}

\begin{itemize}
\item
  hat einen hohen Trockensiedepunkt $\geq 230~^\circ\text{C}$
  (Mindestsiedepunkt)
\item
  hat einen hohen Nasssiedepunkt $\geq 170~^\circ\text{C}$
\item
  hygroskopisch, entzieht der atmosphärischen Luft die Feuchtigkeit und
  speichert diese.
\item
  ist hochgiftig, $100~cm^3$ sind bereits tödlich
\item
  Aggressiv, greift Lacke und die menschliche Haut an
\end{itemize}

\textbf{DOT5}

\begin{itemize}
\item
  nicht mischbar mit DOT3/4
\item
  hydrophob (fehlende Wasseraufnahme, Wasser kann sich in Tropfenform
  bilden, vgl. hygroskopisch)
\item
  Einsatz: Militär, Harley
\item
  Farbe: blau
\end{itemize}

\textbf{hygroskopisch} entzieht aus der Umgebungsluft Feuchtigkeit, wird
eingezogen in die Bremsflüssigkeit und verteilt sich gleichmäßig im
System. (Vorteil)

\textbf{Trockensiedepunkte / Mindestsiedepunkte:} DOT3 =
$205~^\circ\text{C}$, DOT4 = $230~^\circ\text{C}$, DOT5.1 =
$260~^\circ\text{C}$

\textbf{Nass-Siedepunkt} ist der Siedepunkt bei $3,5~\%$ Wasseranteil.
(nach ca. 2 Jahren erreicht)

>>Je höher der Anteil an Wasser, desto niedriger wird der Siedepunkt.<<
Gefahr der Dampfblasenbildung durch die beim Bremsen entstehenden Wärme.
Vgl. Tabelle

\begin{table}[!ht]% hier: !ht 
\centering 
	\caption{}% \label{tab:}%% anpassen 
\begin{tabular}{@{}llll@{}}
\hline
\textbf{Wassergehalt} & \textbf{DOT3} & \textbf{DOT4} &
\textbf{DOT5.1} \\
\hline
$0,8~\%$ & $200~^\circ\text{C}$ & $220~^\circ\text{C}$ &
$245~^\circ\text{C}$ \\
$2~\%$ & $160~^\circ\text{C}$ & $190~^\circ\text{C}$ &
$210~^\circ\text{C}$ \\
$3,5~\%$ & $140~^\circ\text{C}$ & $170~^\circ\text{C}$ &
$180~^\circ\text{C}$ \\
\hline
\end{tabular} 
\end{table}

\section{Kühlflüssigkeit}\label{kuehlfluessigkeit}

\begin{itemize}
\item
  Gemisch aus Wasser und Gefrierschutzmittel
\item
  Gefrierschutzmittel besteht aus \textbf{Glykol}, senkt die
  Gefriertemperatur, Anteil zwischen 40 \% und 50 \%
\item
  \textbf{Standards:} von VW \textbf{G11} (grün/blaugrün), \textbf{G12}
  (Pink), \textbf{G13} (rotviolett), sowie von BASF \textbf{Glysantin}
\item
  G11 mit (G12 oder G13) \textbf{nicht mischbar} (Motor -
  >>Aluverträglichkeit<<)
\item
  \textbf{Messen:} Refraktometer oder Messspindel (Aräometer)
\end{itemize}

\textbf{G13}

\begin{enumerate}
\item
  wird aus \textbf{Glyzerin} hergestellt, ist weniger umweltschädlich
  als Glykol
\item
  hervorragende Kühleigenschaften
\item
  bietet Schutz vor Korrosion und Kalkablagerungen
\end{enumerate}

\section{AdBlue}\label{adblue}

\begin{itemize}
\item
  ist eine Mischung aus $32,5~\% (\text{ca. } \frac{1}{3})$ Harnstoff
  und Demineralisierten Wasser
\item
  Harnstoff dient als Trägerflüssigkeit für das giftige Ammoniak, der
  zur Reduktion von Stickoxiden durch das SCR-System benötigt wird.
  Thermische- und Hydrolysestrecke notwendig.
\item
  gefriert bei $- 11,5~^\circ\text{C}$
\end{itemize}

\emph{Bemerkung:} \textbf{Abgasnachbehandlung}

\begin{enumerate}
\item
  \textbf{Oxidationskatalysator} (Sauerstoff), Arbeitsbereich
  $400 - 800~^\circ\text{C}$, ca. $350~^\circ\text{C}$
  >>light-off-Point<< ($50~\%$ Umwandlungsrate)

  \begin{itemize}
  \item
    Benzin: Oxidation $CO + O_2 \to CO_2$,
    $HC + O_2 \to CO_2 + H_{2}O$, Reduktion
    $NO_\text{x} \to N_2 + O_2$
  \item
    Diesel: $CO \to CO_2$, $HC \to CO_2 + H_{2}O$
  \item
    (Kohlenmonoxid in Kohlenstoffdioxid, unverbrannte Kohlenwasserstoffe
    (Kraftstoff) in Kohlenstoffdioxid und Wasser, Stickoxide in
    Stickstoff und Sauerstoff)
  \end{itemize}
\item
  \textbf{Dieselpartikelfilter} (DPF-Regeneration) ab ca.
  $600~^\circ\text{C}$

  \begin{itemize}
  \item
    $\text{PM} + O_2 \to CO_{2}$ (Partikel und Sauerstoff in
    Kohlenstoffdioxid)
  \item
    Regeneration (angesammelten Partikel im Partikelfilter verbrennen,
    Staudruck, Differenzdrucksensor)

    \begin{itemize}
    \item
      Nacheinspritzung in Verbrennungsraum: Dabei wird der Kraftstoff
      erst spät in den Brennraum eingespritzt, wodurch die Flamme bis in
      den Ausstoßtakt brennt und die Abgastemperatur im Partikelfilter
      steigt.
    \item
      Auspufföffnungs-Einspritzung über ein EPI-Ventils (Exhaust Port
      Injection) direkt in den Abgaskrümmer
    \end{itemize}
  \end{itemize}
\item
  \textbf{SCR-Katalysator} (selektive katalytische Reduktion) ab ca.
  $170 - 250~^\circ\text{C}$

  \begin{itemize}
  \item
    \textbf{chemische Prozess} Dosierventil $\to$
    \textbf{Hydrolysestrecke} Harnstoff $\to$ Ammoniak
    (Reduktionsmittel, Gefahrstoff, giftig) $\to$ SCR-Katalysator
    $\to$ Stickoxidreduktion:
  \item
    $NO + NO + 2NH_3 \to 2N_2 + 3H_{2}O$ (Stickoxide und Ammoniak in
    Stickstoff und Wasser)
  \end{itemize}
\item
  \textbf{NOx-Speicherkatalysator} wird von Schwefelbestandteilen
  zugesetzt und muss regeneriert werden.
\end{enumerate}

Wenn der Schadstoffausstoß steigt, führt das zum Erlöschen der
Betriebserlaubnis (Abgasemissionsklasse nicht mehr gültig, bedeutet
\textbf{Steuerhinterziehung})

\section{Kältemittel}\label{kaeltemittel}

\textbf{R134a} (Tetrafluorethan) hat seinen Siedepunkt bei ca.
$-26~^\circ\text{C}$ bei atmosphärischem Druck. Bei 15 bar Überdruck
liegt der Siedepunkt bei ca. $55~^\circ\text{C}$.

\begin{itemize}
\item
  GWP-Faktor 1430
\end{itemize}

\textbf{R1234yf} (Tetrafluorpropen) verhält sich ähnlich
(Siedetemperatur bei Atmosphärendruck $-29~^\circ\text{C}$).

\begin{itemize}
\item
  GWP-Faktor 4
\end{itemize}

\textbf{R744} ($CO_2$) sind höhere Drücke in der Klimaanlage
erforderlich.

\begin{itemize}
\item
  GWP-Faktor 1
\end{itemize}

\textbf{R12} enthält Fluorchlorkohlenwasserstoffe (\textbf{FCKW}), die
in der Atmosphäre die Ozonschicht zerstören. Seit 1991 wurde daher R134a
verwendet und 2017 meist durch R1234yf abgelöst.

\textbf{Treibhauseffekt} bezeichnet die Erwärmung der Erde durch
Reflexion von Wärmestrahlung in der Atmosphäre.

\textbf{GWP} (Global Warming Potential, Treibhauspotenzial) gibt den
Treibhauseffekt eines Stoffes im Vergleich zu Kohlendioxid an. Welche
Auswirkungen ein Stoff auf die Umwelt hat?

\textbf{Diffusion} beschreibt einen physikalischen Prozess, bei dem sich
zwei Stoffe nach und nach durchmischen, bzw. ein Stoff einen anderen
durchdringt. (Beispiel: Bremsflüssigkeit, Kältemittel)

\section{Schmieröle}\label{schmieroele}

\subsection{Aufgaben von Motoröle}\label{aufgaben-von-motoroele}

\begin{enumerate}
\item
  Schmieren (Lager, Gleitstellen von Kolben und Zylinder)
\item
  Kühlen (ableiten der Wärme vom Kolben)
\item
  Abdichten (zwischen Kolbenringen und Zylinderlaufbuchsen,
  Feinabdichtung an Radialwellendichtringe)
\item
  Reinigen (Aufnehmen von Verbrennungsrückständen, Abrieb, Wasser,
  Säuren)
\item
  Geräusche dämpfen
\end{enumerate}

\subsection{Anforderung an
Motorenöle}\label{anforderung-an-motorenoele}

\begin{enumerate}
\item
  hoher Viskositätsindex (d.h. Änderung der Viskosität über der
  Temperatur)
\item
  hohe Temperaturbeständigkeit
\item
  geringe Verdampfungsneigung (geringer Ölverbrauch)
\item
  Öl soll einen niedrigen Stockpunkt haben
\item
  soll weiterhin eine hohe Schmutz- und Säureaufnahmefähigkeit besitzen
\item
  niedrigen Schwefel-, Asche- und Phosphorgehalt
\end{enumerate}

\subsection{Merkmale von Vollsynthetisches
Öl}\label{merkmale-von-vollsynthetisches-oel}

\begin{enumerate}
\item
  \textbf{Sehr hoher Viskositätsindex} (stabile Schmierung über einen
  großen Temperaturbereich)
\item
  \textbf{Gute Fließfähigkeit} (Kraftstoffeinsparung und schnelle
  Förderung des Öls an die Schmierstellen bei sehr niedrigen
  Temperaturen)
\item
  \textbf{Hohe Druckfestigkeit} (Schmierfilm wird auch bei starker
  Druckbelastung nicht unterbrochen)
\item
  \textbf{Gutes Schmutztrageverhalten} (Abrieb oder
  Verbrennungsrückstände werden im Öl in Schwebe gehalten)
\item
  \textbf{Sehr alterungsbeständig} (deshalb sind
  Langzeitölwechselintervalle bei Verbrennungsmotoren möglich)
\item
  \textbf{Geringe Verdampfungsverluste} (niedriger Ölverbrauch auch bei
  hohen thermischen Belastungen)
\end{enumerate}

Nachteil: Höhere Herstellungskosten

\emph{Bemerkung:} Beim Wechsel von Mineralöl zu Vollsynthetiköl muss das
erste Wechselintervall aufgrund der starken reinigenden Wirkung verkürzt
werden.

\subsection{Einteilung der Motoröle - SAE-Viskositätsklassen und
Klassifizierung}\label{einteilung-der-motoroele-sae-viskositaetsklassen-und-klassifizierung}

\begin{enumerate}
\item
  \textbf{SAE-Viskositätsklassen:} (Auswahl nach Temperaturbereich, seid
  1911, beginnt bei SAE 0 bis 60)

  \begin{itemize}
  \item
    Einbereichsölen (Beispiel: SAE 50)
  \item
    Mehrbereichsölen (Beispiel: SAE 0W-40)
  \end{itemize}
\item
  \textbf{Motoröl Klassifizierung}

  \begin{itemize}
  \item
    \textbf{API} höhere Anforderungen

    \begin{itemize}
    \item
      S-Klassen für Ottomotoren
    \item
      C-Klassen für Dieselmotoren
    \end{itemize}
  \item
    \textbf{ACEA} (europäische Klasse) Mindestanforderungen an die
    Qualität

    \begin{itemize}
    \item
      A-Klassen-Öle für Ottomotoren
    \item
      B-Klassen-Öle für Pkw-Dieselmotoren
    \item
      E-Klassen-Öle für Nfz-Dieselmotoren
    \end{itemize}
  \item
    \textbf{ILSAC} International
  \end{itemize}
\end{enumerate}

\subsection{Mehrbereichsöle}\label{mehrbereichsoele}

sind Schmieröle, die mehr als eine Viskositätsklasse abdecken.

\begin{itemize}
\item
  \textbf{Kaltstart} (Kaltstarterleichterung $\to$ geringe Reibung und
  schnelle Durchölung bei niedrigen Außentemperaturen) und
\item
  \textbf{Wärmebelastbarkeit} (Temperaturfestigkeit bei hohen
  Temperaturen und gute Schmierfähigkeit)
\end{itemize}

\emph{Beispiel:} \textbf{SAE 15W-40} verhält sich bei tiefen
Temperaturen wie ein Öl der Klasse 15W und bei hohen Temperaturen wie
ein SAE 40 Öl. >>Je kleiner die Zahl vor dem \emph{W}, desto
fließfähiger ist das Öl in der Kälte und je höher die SAE-Kennzahl,
desto zähflüssiger ist das Öl.<<

\textbf{Was bedeutet bei der SAE-Klasse der Buchstabe W?}

Der Buchstabe \emph{W} bedeutet Winter.

\textbf{Was sind Leichtlauföle?}

Als Leichtlauföle (z.B. 0W-30) bezeichnet man Mehrbereichsöle, die ein
sehr gutes Niedrigtemperaturverhalten (geringe Reibung bei Kaltstart)
und bei hohen Temperaturen eine Viskosität wie ein Einbereichsöl SAE 30
bieten.

\textbf{Warum sind bei Dieselmotoren mit Dieselpartikelfiltern spezielle
Öle erforderlich?}

Um einen niedrigen Schwefel-, Asche- und Phosphorgehalt zu erreichen.
Ascherückstände aus dem Öl lagern sich in den Dieselpartikelfiltern ab
und verringern dessen Speicherkapazität. Da Ascherückstände selbst bei
hohen Temperaturen nicht frei gebrannt werden können, kommt es zum
Ausfall des Filters.

\subsection{Viskosität}\label{viskositaet}

Es ist ein Maß für die Zähflüssigkeit von Flüssigkeiten.

Unter Viskosität versteht man die Eigenschaft einer Flüssigkeit ihrer
Verformung einen Widerstand entgegenzusetzen.

Öl hat eine \emph{niedrige Viskosität} und damit einen niedrigen
Verformungswiderstand, wenn es dünnflüssig ist und eine \emph{hohe
Viskosität}, wenn es zähflüssig ist.

Die Tragfähigkeit des Schmierfilms ist allerdings bei höheren Viskosität
besser als bei einer niedrigen.

\textbf{Ermittlung der Viskosität eines Öls}

\begin{enumerate}
\item
  Kapillarviskosimeter (Kinematische Viskosität)
\item
  Rotationsviskosimeter (Dynamische Viskosität)
\end{enumerate}

\subsection{Nenne 5 -- 7x Additive und
Eigenschaften}\label{nenne-5-7x-additive-und-eigenschaften}

\begin{enumerate}
\item
  \textbf{Detergants} Schmutz lösen
\item
  \textbf{Dispersants} Schmutz in der Schwebe halten
\item
  \textbf{Verschleißschutzzusätze} (EP - Extreme Pressure) (unter hohen
  Druck stehenden Gleitflächen, Beispiel: Zahnradflanken oder zwischen
  Nocken und Tassenstößel)
\item
  \textbf{Korrosionsschutzzusätze} (bauen wasserabweisende Schutzfilme
  auf, schützen vor aggressiven Verbrennungsrückständen und
  neutralisieren Säuren)
\item
  \textbf{Reibwertveränderer} (beeinflussen den Reibwert zwischen
  Materialpaarungen. Beispiel: bei Synchrongetrieben, Nasskupplungen,
  Lamellenkupplung in Automatikgetrieben)
\item
  \textbf{Alterungsschutzadditive} (verhindern die Oxidation des Öls
  unter Einfluss von Wärme und Sauerstoff)
\item
  \textbf{Stockpunkterniedriger} (verbessert die Fließeigenschaften des
  Öls bei tiefen Temperaturen. Beispiel: verringert Motorverschleiß bei
  Kaltstart)
\item
  \textbf{Antischaum} verhindern die Schaumbildung im Öl. (durch bewegte
  Teile)
\item
  \textbf{VI-Verbesserer} (Viskositätsverbesserer) sind im kalten
  Zustand zusammengeknäuelt im Öl enthalten. Erwärmt sich das Öl
  entknäueln sie sich und nehmen ein größeres Volumen ein. Dadurch
  wirken sie der zunehmenden Dünnflüssigkeit des Öls bei Erwärmung
  entgegen und können einen belastungsfähigen Schmierfilm aufbauen.
\end{enumerate}

\textbf{Pourpoint} (Grenzpumptemperatur) ist die Temperatur, bei der das
Öl gerade noch fließt. Dadurch ist gewährleistet, dass beim Motorstart
genügend Öl zur Ölpumpe und in den Schmierölkreislauf fließt.
\textbf{Stockpunkt} gibt die Temperatur an, bei der das Öl >>stockt<<.

\textbf{Schlammablagerungen} wird durch Alterungsprodukte, Ruß,
unverbrannte Kraftstoffreste, Stickoxide und Wasser verursacht. $\to$
\textbf{Folgen} sind Verstopfen von Ölleitungen und Ölfiltern, erzeugen
von Fressschäden an Kolben und Zylinderlaufbahnen sowie Lagerschäden.

\textbf{Schaumbildung} dadurch wird der Ölfilm unterbrochen, Ölalterung
beschleunigt und die Kompressibilität des Öls erhöht. $\to$
\textbf{Folgen} (1) Schmiereigenschaften verringert sich, dadurch sind
Fressschäden möglich. (2) Ölwechselintervalle verkürzen sich (3) Störung
bei der Kraftübertragung durch verringerten Druckaufbau in hydraulischen
Schaltelementen

\textbf{Wodurch altert Öl?} Ölanalyse \footnote{\url{https://de.oelcheck.com/}}

\begin{enumerate}
\item
  Druck und Temperatur
\item
  Sauerstoff ($O_2$)
\item
  Laufkilometer und Zeit
\end{enumerate}

Neues Öl ist basisch (Vgl. >>Motor sauer fahren<<).

\subsection{Ölverdünnung oder
Ölvermehrung}\label{oelverduennung-oder-oelvermehrung}

\emph{Bemerkung:} Ein zu hoher Ölstand kann ein Indiz für eine
Ölverdünnung sein durch häufige Kaltstarts.

\begin{itemize}
\item
  schädlich für Motoröl und Katalysator
\item
  beim Diesel: lange Stillstandszeiten (vgl. Biodiesel -- Wasser
  anziehend)
\item
  wenn Dieselkraftstoff in das Motoröl gelangt

  \begin{itemize}
  \item
    Beispiel: DPF-Regeneration $\to$ Nacheinspritzung und den an den
    Kolbenringen abfließenden Kraftstoff kommt es zu Motorproblemen
    durch Ölverdünnung (zu hoher Ölstand, schlechte Ölqualität)
  \end{itemize}
\end{itemize}

\section{Getriebeöle}\label{getriebeoele}

\textbf{Anforderungen}

\begin{enumerate}
\item
  Verschleißschutz
\item
  Unterschiedliches Reibverhalten
\item
  Alterungsschutz
\item
  Dichtungsverträglichkeit
\end{enumerate}

\textbf{Welche Viskositätsklassen gelten für Mehrbereichsgetriebeöle?}

\begin{enumerate}
\item
  SAE 80W-90
\item
  SAE 75W-90 (Leichtlauf-Getriebeöl)
\end{enumerate}

\textbf{Nennen Sie die Besonderheit der Getriebeöle für Hypoidachsen.}

Getriebeöle für Hypoidachsen sind mit EP-Zusätzen (Extrem Pressure,
Lasttrageverhalten) versehen, die an den Metalloberflächen
Schutzschichten bilden, damit der Schmierfilm zwischen den
Zahnradflanken nicht weggedrückt wird.

\textbf{Welche Aufgaben/Anforderungen haben Automatikgetriebeöle?}

\begin{enumerate}
\item
  Drehmomentübertragung von Pumpen- zum Turbinenrad
\item
  Schmieren von Lager, Planetenrädern und Freiläufe,
\item
  Betätigen von Lamellenkupplungen
\end{enumerate}

\section{Schmierfette}\label{schmierfette}

\textbf{Schmierfette} sind eingedickte Schmieröle.

\textbf{Gruppen und Eigenschaften von Schmierfetten} (vgl. Tabelle)

\begin{table}[!ht]% hier: !ht 
\centering 
	\caption{}% \label{tab:}%% anpassen 
\begin{tabular}{@{}llll@{}}
\hline
\textbf{Seifenbasis} & \textbf{wasserfest} & \textbf{Verwendung} &
\textbf{Temperaturbereich} $[^\circ\text{C}]$ \\
\hline
Kalziumseifenfett & ja & Abschmierfett & $-40 \ldots 60$ \\
Natriumseifenfett & nein & Wälzlagerfett & max. $100$ \\
Lithiumseifenfett & ja & Mehrzweckfett & $-20 \ldots 130$ \\
\hline
\end{tabular} 
\end{table}

NLGI-Klassen (vgl. Tabelle)

\begin{table}[!ht]% hier: !ht 
\centering 
	\caption{}% \label{tab:}%% anpassen 
\begin{tabular}{@{}lll@{}}
\hline
\textbf{Konsistenz} & \textbf{Eigenschaft} & \textbf{Verwendung} \\
\hline
Klasse 000 -- 1 & \textbf{sehr weich} & Fließfette \\
Klasse 2 -- 3 & \textbf{weich} & Abschmierfette \\
Klasse 4 -- 5 & \textbf{fest} & Wasserpumpenfette \\
\hline
\end{tabular} 
\end{table}

\textbf{Konsistenz} ist der Widerstand eines Fettes gegen Verformung.

\textbf{EP-Schmierfette} (Extreme-Pressure, können hohen Drücken
standhalten)

\textbf{Hochtemperaturfette} ($> 130^\circ\text{C}$)

\textbf{Tropfpunkt} ist die Temperatur, bei der unter Prüfbedingungen,
der erste Tropfen des schmelzenden Schmierfettes abtropft.
