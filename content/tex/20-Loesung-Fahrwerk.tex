%ju 26-Dez-22 20-Loesung-Fahrwerk.tex
\textbf{1. Was wird als Radsturz bezeichnet?}

\begin{itemize}
\item
  Als Radsturz bezeichnet man die Neigung der Radebene gegenüber einer
  gedachten, senkrecht zur Fahrbahn verlaufenden Ebene

  \begin{itemize}
  \item
    Radstellung geradeaus
  \item
    Blick parallel zur Fahrzeuglängsachse
  \end{itemize}
\end{itemize}

\textbf{2. Was ist ein >>Negativer Lenkrollhalbmesser<<? Welchen
Einfluss hat er auf das Fahrverhalten des Fahrzeugs?}

\begin{itemize}
\item
  Vom negativen Lenkrollhalbmesser spricht man, wenn sich die
  Mittellinien durch das Rad und durch die Spreizachse schon oberhalb
  der Fahrbahn schneiden.
\item
  Durch den negativen Lenkrollhalbmesser wird eine Selbststabilisierung
  des Fahrzeugs erreicht. Das bedeutet, der Fahrer braucht keine
  Kurskorrektur durchzuführen, wenn

  \begin{itemize}
  \item
    beim Bremsen die Bodenhaftung unterschiedlich ist
  \item
    die Bremswirkung an den Rädern unterschiedlich ist
  \item
    die Räder stark unterschiedlichen Reifendruck haben
  \end{itemize}
\end{itemize}

\textbf{3. Was versteht man beim Kfz als Übersteuern?}

\begin{itemize}
\item
  Beim Übersteuern drängt das Fahrzeug bei schneller Kurvenfahrt mit den
  Hinterrädern nach außen. Der Schräglaufwinkel der Hinterräder ist
  größer als der Schräglaufwinkel der Vorderräder.
\item
  Das Fahrzeug durchfährt einen kleineren Kurvenradius, als es der
  Lenkeinschlag vorgibt. Um der Kurve weiter zu folgen, muss
  gegengelenkt werden.
\end{itemize}

\textbf{4. Was zählt beim Kfz zu den ungefederten Massen?}

\begin{itemize}
\item
  Alle Bauteile unterhalb der Fahrzeugfederung, wie zum Beispiel Rad und
  Reifen, Radnabe, Bremsen und Achsschenkel.
\end{itemize}

\textbf{5. Welche Vor- und Nachteile hat eine Starrachse?}

\textbf{Vorteile}

\begin{itemize}
\item
  hohe statische Belastbarkeit
\item
  keine Änderung von Sturz, Spreizung und Spur beim Ein- und Ausfedern
\end{itemize}

\textbf{Nachteile}

\begin{itemize}
\item
  Hohes Gewicht und dadurch große ungefederte Masse
\end{itemize}

\textbf{6. Welche Aufgabe hat der Panhardstab?}

\begin{itemize}
\item
  Er dient zur seitlichen Führung der Karosserie auf der Hinterachse.
\end{itemize}

\textbf{7. Was erreicht man mit einer Raum-Lenker-Hinterachse?}

\begin{itemize}
\item
  Man erreicht mit bis zu 5 Lenkern pro Rad eine exakte Radführung, und
  eine geringfügige Änderung von Sturz und Spur beim Ein- und Ausfedern,
  Beschleunigen und Bremsen
\end{itemize}
