%ju 13-Aug-22 02-Loesung-Motorsteuerung.tex
\textbf{1) Welche Aufgabe übernehmen die Ventile eines
Verbrennungsmotors?}

Ermöglicht den Gaswechsel und dichten den Verbrennungsraum gegenüber dem
Saugrohr und Abgasanlage ab.

\textbf{2) Warum haben Einlassventile meist einen größeren
Ventiltellerdurchmesser als Auslassventile?}

Einlassventile haben oftmals einen größeren Ventiltellerdurchmesser, da
das einströmende Frischgas nur durch Unterdruck im Zylinder angesaugt
wird, während das im Zylinder befindliche Altgas noch unter Restdruck
aus der Verbrennung steht und den Zylinder somit auch über einen kleinen
Querschnitt zuverlässig verlässt.

\textbf{3) Wie hoch ist die thermische Belastung von Ein- und
Auslassventilen?}

\textbf{Einlassventile} bis ca. $500^\circ\text{C}$
\textbf{Auslassventile} bis ca. $900^\circ\text{C}$ (fehlende
Frischgaskühlung)

\textbf{4) Welche Aufgabe hat die Ventildrehvorrichtung?}

Die Ventildrehvorrichtung hat die Aufgabe, die Ventile bei laufendem
Motor kontinuierlich zu drehen.

\textbf{Dies verhindert:}

\begin{enumerate}
\item
  ungleichmäßige Erwärmung der Ventilteller
\item
  Undichtigkeiten durch Verzug (nicht sauberes Anliegen)
\item
  Störungen bei Wärmeabgabe
\item
  Hochtemperaturkorrosion an den heißesten Stellen (das Ventil
  verbrennt)
\item
  Abblättern der Verbrennungsrückstände (stetiges Einschleifen der
  Ventile)
\end{enumerate}

\textbf{Bauformen}

\begin{enumerate}
\item
  Rotocap
\item
  Rotomat
\end{enumerate}

\textbf{5) Beschreiben Sie Aufbau und Wirkungsweise eines Hohlventils.
(inkl. Temperaturangaben!)}

Hohlventile sind im Schaft, teilweise auch im Teller hohl. Dieser
Hohlraum ist zu ca. $60 - 70~\%$ mit metallischem Natrium gefüllt, bei
ca. $98^\circ\text{C}$ schmilzt das Natrium und bewegt sich
hervorgerufen durch die Ventilbewegung im Ventil auf und ab. Bei jeder
Bewegung nimmt es am Ventilteller Wärme auf und gibt diese am
Ventilschaft ab. Der Abkühleffekt am Ventilteller liegt bei ca.
$80 - 150^\circ\text{C}$. Durch die hohlgeborte Form des Ventils
verringert sich seine Masse. Kann auch als Einlassventil verwendet
werden.

\textbf{6) Warum besteht zwischen dem Sitzwinkel des Ventilsitzringes im
Zylinderkopf und dem am Ventil oftmals eine Differenz von ca. 1°?}

Durch die Sitzwinkeldifferenz ist die Dichtfläche schmal. Bei
Inbetriebnahme des Motors arbeiten sich Ventilteller und Sitzring
schnell aufeinander ein. Dadurch entfällt das Ventileinschleifen.

(Flächenpressung, Minutenring)

\textbf{7) Welche Auswirkungen hat ein zu großes/zu kleines
Ventilspiel?}

\textbf{Zu kleines Ventilspiel} (Nachteile)

\begin{itemize}
\item
  Ventil öffnet früher und schließt später
\item
  Ventil ist länger auf
\item
  kann dadurch nicht genügend Wärme abgeben über Ventilsitz
\item
  Ventilteller wird immer weiter einer höheren thermischen Belastung
  unterzogen und dadurch erhöhter Verschleiß
\item
  Am Ende ist das Ventil einer Hochtemperaturkorrosion unterworfen
  (Verbranntes Ventil)
\end{itemize}

\textbf{zu großes Ventilspiel} (Nachteile)

\begin{itemize}
\item
  Ventil öffnet zu spät, geht nicht ganz auf und schließt zu früh
\item
  Ventil ist kürzer auf
\item
  Klappergeräusche und erhöhter Verschleiß, \emph{Warum?} durch großes
  Ventilspiel, liegt nicht am Nockengrundkreis auf (Nocken schlägt auf
  Ventil)
\item
  Hieraus können folgen: schlechte Zylinderfüllung und die maximale
  erreichbare Leistung sinkt
\end{itemize}

\textbf{8) Wo im Ventiltrieb kann der Ventilspielausgleich eingesetzt
sein?}

Ventilspielausgleich kann sich zwischen Nocken und Ventil oder bei
Bauformen Schlepphebel am Aufnahmepunkt des Hebels befinden.

\textbf{9) Beschreiben Sie Aufbau und Wirkungsweise des hydraulischen
Stößel.}

\textbf{Ablaufender Nocken} (ohne Belastung)

\begin{itemize}
\item
  Entspannung des Systems
\item
  Spielausgleichsfeder drückt Druckbolzen nach oben bis Stößel am Nocken
  anliegt
\item
  Kugelventil öffnet sich, Raumvergrößerung im Arbeitsraum (Unterdruck)
\item
  Durch den Systemdruck strömt frisches Öl von außen ein und der
  Arbeitsraum wird befüllt
\end{itemize}

\textbf{Auflaufender Nocken} (mit Belastung)

\begin{itemize}
\item
  Kugelventil schließt sich, es baut sich Druck im System auf
\item
  durch die Inkompressibilität von Flüssigkeiten $\to$ starre
  Verbindung
\item
  Nocken wird auf den Stößel auflaufen können, ohne Spiel zu haben und
  das Ventil betätigen
\item
  \emph{Warum Ringspalt?} (Wärmeausdehnung des Öls ausgleichen)
\item
  Wärmeeintrag: je wärmer das Öl, umso dünnflüssiger
\item
  dadurch wird >>Öl<< durch den kleinen Ringspalt gepresst (definierte
  Menge an Öl)
\item
  erfordert die richtige Öl-Viskosität (Zähflüssigkeit,
  Temperaturabhängig, Fließverhalten), sind an diese Ringspalte
  angepasst
\end{itemize}

\textbf{10) Warum verwendet man bei herkömmlichen 4-Takt-Motoren und
Pkw-Dieselmotoren nur noch oben liegende Nockenwellen?}

Durch die obenliegende Nockenwelle können die bewegten Massen des
Ventiltriebs gering gehalten und somit höhere Drehzahlen erreicht
werden. (z. B. Stößelstange, mehr Bewegung $\to$ erhöht Reibung und
Masse)

\textbf{11) Welche Nockenausführungen findet man an den Nockenwellen von
Verbrennungsmotoren?}

\begin{enumerate}
\item
  \textbf{spitzer Nocken} (tagenden Nocken)
\item
  \textbf{steiler Nocken} (scharfer Nocken, Kreisbogen Nocken)
\item
  \textbf{unsymmetrischer Nocken}
\end{enumerate}

\textbf{12) Aus welchem Werkstoff können Nockenwellen bestehen? (keine
Prüfung)}

\begin{enumerate}
\item
  \textbf{Gegossene Nockenwelle}

  \begin{itemize}
  \item
    Gusseisen mit Lamellen- o. Kugelgrafit
  \end{itemize}
\item
  \textbf{Gebaute Nockenwelle}

  \begin{itemize}
  \item
    Einsatz-, Vergütungs- o. Nitrierstahl
  \end{itemize}
\end{enumerate}

(Eigenschaften: Welche Kräfte wirken? zäh fest versus beweglich)

\textbf{13) Was versteht man unter desmodromischer Ventilsteuerung?}

Bei desmodromischer Ventilsteuerung werden die Einlassventile und
Auslassventile jeweils durch einen Öffnungs- und Schließkipphebel
betätigt.

(Zwangssteuerung, Einsatz: hohe Drehzahlen, AV zuverlässig schließen)
