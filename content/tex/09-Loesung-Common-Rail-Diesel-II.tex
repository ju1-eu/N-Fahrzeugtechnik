%ju 31-Dez-22 09-Loesung-Common-Rail-Diesel-II.tex
\textbf{1) Beschreiben Sie ausführlich den Einspritzvorgang beim
Pumpe--Düse--System mit Magnetventil.}

\textbf{Befüllen} bei ablaufenden Einspritznocken wird der Pumpenkolben
nach oben gezogen und der Hochdruckraum wird mit Kraftstoff befüllt.
Solange Magnetventil offen ist, fließt der Kraftstoff durch den
Hochdruckraum in den Rücklauf ab. (kühlende Wirkung) Bei auflaufenden
Nocken wird der Pumpenkolben nach unten gedrückt. Magnetventil ist
geschlossen und verschließt Kraftstoff, Vor- und Rücklaufleitung. Es
baut sich ein Hochdruck auf.

\textbf{Voreinspritzung} (Beginn) bei etwa 180 bar ist der Druck größer
als die Düsen-Federkraft. Durch das Einspritzen des Kraftstoffs haben
wir einen kurzen Druckabfall und die Düsennadel schließt (Ende). Der
Ausweichkolben hat seine schräge Druckschulter freigegeben (große
Fläche, große Kraft), wodurch er nach unten gedrückt wird und spannt die
Düsenfeder vor. Magnetventil öffnet und der überschüssige Kraftstoff
entweicht in die Vor- und Rücklaufleitung.

\textbf{Haupteinspritzung} der Druck steigt weiter und drückt bei etwa
300 bar die Düsennadel nach oben (Ende). Durch den Druckabfall schließt
die Düsennadel.

\textbf{2) Wie lässt sich beim Pumpe--Leitung--Düse--System eine
Voreinspritzung realisieren?} (NFZ)

Durch Zweifeder-Düsenhalter (kleine Feder - Voreinspritzung, große Feder
- Haupteinspritzung). Bei geringem Kraftstoffdruck öffnet die Düse nur
teilweise.

\textbf{3) Welche Hauptvorteile bietet das Common--Rail--System
gegenüber allen anderen gängigen Einspritzsystemen?}

\begin{itemize}
\item
  Einspritzmenge lässt sich über Zeit und Einspritzdruck variieren (je
  nach Betriebszustand)
\item
  Kein Druckanstieg im Einspritzvorgang: (der Öffnungsdruck entspricht
  dem jeweiligen Einspritzdruck)
\item
  beliebige Anzahl von Einspritzvorgängen (ruhigen Motorleerlauf viele
  Einspritzungen, für maximales Drehmoment bei Volllast nur eine)
\item
  Einspritzung unabhängig von Kurbelwellen- und Nockenwellenposition
  (Kraftstoffdruck steht immer an)
\item
  Anpassung an unterschiedliche Motoren
\end{itemize}

\textbf{4) Welche Regelungsarten des Common--Rail--Systems kennen Sie?}

\begin{enumerate}
\item
  \textbf{Einsteller-Regelung}

  \begin{itemize}
  \item
    \textbf{Druckregelung hochdruckseitig} über ein
    \textbf{Druckregelventil} (DRV)

    \begin{itemize}
    \item
      Hochdruckpumpe fördert die maximale Fördermenge unabhängig von
      Bedarf an Kraftstoff
    \item
      Raildruck wird über ein \textbf{Druckregelventil} geregelt, nicht
      benötigter Kraftstoff fließt zurück in den Tank
    \item
      \textbf{Vorteil}

      \begin{itemize}
      \item
        schneller Druckaufbau möglich
      \item
        bei Lastwechsel agil
      \end{itemize}
    \item
      \textbf{Nachteil:} Hochdruckpumpe ist, konstant maximal belastet,
      dadurch

      \begin{itemize}
      \item
        geringere Nutzleistung
      \item
        erhöhter Kraftstoffverbrauch, Schadstoffausstoß, Verschleiß
      \item
        unnötige Erwärmung des Kraftstoffs
      \end{itemize}
    \end{itemize}
  \item
    \textbf{Druckregelung niederdruckseitig} (Mengenregelung) über eine
    \textbf{Zumesseinheit} (ZME)

    \begin{itemize}
    \item
      \textbf{Zumesseinheit} regelt die Zuflussmenge zur Hochdruckpumpe
    \item
      \textbf{Bedarfsregelung}, d.h. es gelangt nur so viel Kraftstoff
      zu Hochdruckpumpe wie für die Einspritzung benötigt wird
    \item
      \textbf{Druckbegrenzungsventil} verhindert zu hohen Raildruck bei
      Ausfall der Zumesseinheit
    \item
      \textbf{Vorteil}

      \begin{itemize}
      \item
        bedarfsgerechte Förderung
      \item
        Reduzierte Leistungsaufnahme der Pumpe
      \end{itemize}
    \item
      \textbf{Nachteil:}

      \begin{itemize}
      \item
        hohe Trägheit
      \item
        Drucksteigerung erfordert zunächst eine Erhöhung des
        Niederdrucks, dadurch erhöhte Förderleistung der Hochdruckpumpe
      \end{itemize}
    \end{itemize}
  \end{itemize}
\item
  \textbf{Zweisteller-Regelung} (\textbf{Druckregelung nieder- und
  hochdruckseitig})

  \begin{itemize}
  \item
    Zusammenspiel von Zumesseinheit (ZME) und Druckregelventil (DRV)
  \item
    \textbf{Vorteile}

    \begin{itemize}
    \item
      geringe Leistungsaufnahme der Hochdruckpumpe bei konstant hohen
      Drücken
    \item
      hohe Agilität bei geringen Drücken
    \end{itemize}
  \item
    \textbf{Kaltstart / Motorstart und Warmlaufphase} (angesteuert:
    \textbf{DRV}) >>Druckregelung hochdruckseitig geregelt<<

    \begin{itemize}
    \item
      Schneller Druckaufbau möglich. Bessere Kraftstoffvorwärmung.
      Verzicht auf Kraftstoffheizung möglich.
    \item
      Kraftstoff soll sich erwärmen, um ihn damit fließ- und zündfähiger
      zu machen
    \end{itemize}
  \item
    \textbf{betriebswarmer Motor: Leerlauf / Teilllast / Schub}
    (angesteuert: \textbf{DRV und ZME}) >>Zweisteller-Betrieb<<

    \begin{itemize}
    \item
      große Sprünge des Raildrucks sind wahrscheinlich (zwischen Motor
      im Leerlauf / Teillast und $\to$ Lastwunsch des Fahrers bei
      Volllast)
    \item
      Raildruckregelung bei reduzierter Leistungsaufnahme der Pumpe
    \end{itemize}
  \item
    \textbf{hohen Motorlast} (angesteuert: \textbf{ZME}) >>Druckregelung
    niederdruckseitig geregelt<<

    \begin{itemize}
    \item
      große Steigerungen des Raildrucks nicht möglich und deshalb ist
      die Trägheit vernachlässigbar
    \item
      Reduzierter Leistungsaufnahme der Pumpe. Gesamte Kraftstoffmenge
      der Pumpe wird eingespritzt.
    \end{itemize}
  \end{itemize}
\end{enumerate}

\textbf{5) Was passiert, wenn das Druckregelventil eines
Common--Rail--Systems ausfällt?}

Da das Druckregelventil im spannungsfreien Zustand nur ca. 100 bar
halten kann, der CR-Motor aber, um anzuspringen, 250 bar benötigt, läuft
der Motor bei Ausfall des Druckregelventils nicht.

\textbf{6) Warum wird der Druck in der Rail bei kaltem Motor
ausschließlich über das Druckregelventil geregelt?}

Um den Kraftstoff schneller zu erwärmen, um ihn damit fließ- und
zündfähiger zu machen.

\textbf{7) Welche Spannung ist zum Öffnen eines Magnetspuleninjektors
notwendig? Wie wird diese erzeugt?}

Um den Magnetventilspuleninjektor zu öffnen, wird eine Spannung von ca.
100 V bei einer Stromstärke von 20 A benötigt. Diese entsteht als
Selbstinduktionsspannung im Abschaltmoment des Injektors und wird im
Steuergerät in sogenannten Boosterkondensatoren gespeichert. Sind die
Boosterkondensatoren entladen, kann das Steuergerät die Entstehung einer
Selbstinduktionsspannung provozieren, in dem es die Injektoren mehrfach
mit Bordnetzspannung an taktet, wodurch diese zwar nicht öffnen, jedoch
die erforderliche Selbstinduktionsspannung entstehen kann.

\textbf{8) Welche Vorteile bietet der Piezoinjektor gegenüber dem
Magnetspuleninjektor? Erläutern Sie den technischen Hintergrund.}

Während die Anzahl der Einspritzvorgänge pro Arbeitstakt beim
Magnetspuleninjektor auf maximal fünf begrenzt ist, da es bei jeden
Magnetfeldaufbau in ihm zu einer Gegeninduktion kommt, die den
Magnetfeldaufbau hemmt und den einzelnen Einspritzvorgang damit
verlangsamt, kann der Piezoinjektor problemlos sieben und mehr
Einspritzvorgänge pro Arbeitstakt realisieren.

\textbf{9) Beschreiben Sie den Aufbau einer elektronisch geregelten
Glühstiftkerze. Wie schafft man es, dass diese Glühkerze deutlich
schneller auf Glühtemperatur ist, als die herkömmliche Glühkerze?}

Die elektronisch geregelte Glühstiftkerze besteht wie auch die
herkömmlich Selbstregelnde Glühkerze aus einer Heiz- und einer
verkürzten Regelwendel. Sie besitzt allerdings nur eine Nennspannung von
5 -- 8 V, um die Glühkerze besonders schnell auf Glühtemperatur zu
bekommen, wird sie durch ein PWM-Signal kurzzeitig mit ca. 11 V
angesteuert, also eigentlich überlastet. Hierdurch erreicht sie binnen 1
-- 2 Sekunden eine Glühtemperatur von bis zu $1000^\circ\text{C}$
bevor das Tastverhältnis, das PWM-Signal so weit verändert, dass an der
Glühkerze nur ihre Nennspannung anliegt. Die ausreichend ist, um sie auf
Temperatur zu halten.

\textbf{10) Welche Möglichkeiten der Motor- und Innenraum-Zuheizung
kennen Sie. Nennen Sie Vor- und Nachteile der einzelnen Systeme.}

\begin{table}[!ht]% hier: !ht 
\centering 
	\caption{}% \label{tab:}%% anpassen 
\begin{tabular}{@{}lcccc@{}}
\hline
\textbf{System} & \textbf{Innenraum} & \textbf{Innenraum + Motor} &
\textbf{Energie} & \textbf{Kühlen} \\
\hline
Kraftstoff-Zuheizer & & \checkmark & \checkmark & \\
PTC-Heizung & \checkmark & & \checkmark & \\
Elektrisches Zuheizsystem & & \checkmark & \checkmark & \\
Abgas-Wärmetauscher & \checkmark & & & \\
Wassergekühlter Generator & & \checkmark & & \checkmark \\
Kraftstoff-Wärmetauscher & & \checkmark & & \checkmark \\
\hline
\end{tabular} 
\end{table}

\textbf{Motorzuheizung} nach Motorstart wird beheizt.

(\textbf{Standheizung} ca. 10 Minuten vorheizen, bei Motorstart schon
warm.)
