%ju 31-Dez-22 10-Loesung-Kuehlsystem.tex
\textbf{1. Beschreiben Sie das Funktionsprinzip der
Thermosiphonkühlung.}

\begin{itemize}
\item
  \emph{physikalische Prinzip} wärmeres, leichteres Wasser (geringe
  Dichte) steigt nach oben und kälteres, schwereres Wasser (höhere
  Dichte) sinkt nach unten.
\item
  Anwendung: Motor abstellen, dann kommt es zum Kühlmittelfluss ohne
  Pumpe
\end{itemize}

Die Umwälzung des Kühlmittels erfolgt bei der Thermosiphon bzw.
Wärmeumlaufkühlung durch die temperaturbedingte Dichteänderung des
Kühlmittels.

Erwärmt der Motor das Kühlmittel, sinkt dessen Dichte und steigt
leichter als das übrige Kühlmittel nach oben in den Kühler.

Hier nimmt die Dichte des Kühlmittels wieder zu. Fällt nach unten und
wird dem Motor von unten wieder zugeführt.

\textbf{2. Warum gibt es bei der Zwangsumlaufkühlung zwei verschiedene
Kühlkreisläufe?}

Um ein schnelles Erreichen der Betriebstemperatur zu gewährleisten.

In der Warmlaufphase soll die Temperatur mithilfe des kleinen
Kühlkreislaufs bestmöglich im Motor gehalten und verteilt werden.

Hat der Motor Betriebstemperatur erreicht, wird überschüssige Wärme über
den großen Kühlkreislauf an die Umgebung abgegeben. Das dynamische Zu-
und Abschalten des großen Kreislaufes erfolgt in der Regel mittels eines
Dehnstoffthermostates.

\textbf{3. Beschreiben Sie stichwortartig Aufbau und Funktion des
elektronischen Thermostats.}

Das elektronische Thermostat oder auch kennfeldgeregeltes Thermostat
besteht aus einem Ventil und einem Dehnstoffelement, welches mittels
einer PTC-Heizung zusätzlich erwärmt werden kann.

Es wird eingesetzt, um die Motortemperatur höher als sonst üblich
ansteigen lassen zu können. Hierdurch würde der Kraftstoffverbrauch und
der Ausstoß von HC, CO, $CO_2$, PM reduziert.

Sollte hierdurch der $NO_x$ - Ausstoß zu stark ansteigen, kann das
Steuergerät das Dehnstoffelement beheizen und hierdurch die
Motortemperatur absenken.

Um den Motor auch bei Ausfall der Elektronik vor schädlichen
Übertemperatur zu schützen, öffnet das Dehnstoffelement bei ca.
$110^\circ\text{C}$ selbsttätig.

(Bemerkung: Zündfähiges homogen-mager-Gemisch, kurz über Zündgrenze, PTC
ist selbstregelnd)

\textbf{4. Nennen Sie Ursachen und Wirkungen von Kavitation.}

Zur Kavitation kommt es bei starken Druckgefällen in einem
Hydrauliksystem. Diese können zum einen durch das strömende Medium
(Strömungskavitation) und zum anderen durch schwingende Bauteile
(Schwingungskavitation) entstehen.

Kavitation mindert zunächst den Wirkungsgrad. Infolge von Kavitation
kommt es zu einer Materialverdichtung, die bei anhaltender Kavitation zu
einem Ausbruch von Material führt.

Dieses wiederum kann im System zu Erosionsschäden führen.

\textbf{5. Wie funktioniert ein Lüfterantrieb mit Visco -- Kupplung?}

Im kalten Zustand dreht sich ein Kupplungskörper um eine
Antriebsscheibe. Ein Pumpenkörper fördert Silikonflüssigkeit aus dem
Arbeitsraum in einen Vorratsraum, aus dem sie nicht mehr entweichen
kann.

Hat die Kupplung ihre vordefinierte Einschalttemperatur erreicht, so
öffnet ein Bimetallstreifen mittels eines Schaltstiftes das
Blattfederventil, welches den Zulauf zum Arbeitsraum freigibt.

Die Silikonflüssigkeit tritt in den Arbeitsraum und verbindet die
Antriebsscheibe mit dem Pumpenkörper und der Zwischenscheibe.

Der Lüfter dreht sich. Die Silikonflüssigkeit fließt kontinuierlich von
Arbeitsraum in einen Vorratsraum und über das Ventil zurück.

Nach dem Abschalten des Lüfters durch den Bimetallstreifen wird die
Silikonflüssigkeit in den Vorratsraum zurück gefördert, kann aus diesen
aber nicht mehr entweichen. Die Lüfterkupplung trennt.

\textbf{6. Worin liegen die Vorteile eines elektrischen Lüfterantriebs?}
(Prüfung)

\begin{itemize}
\item
  Gestufte oder stufenlose Drehzahlanpassung möglich
\item
  Lüfter kann auch bei abgestelltem Motor betrieben werden

  \begin{itemize}
  \item
    Vermeidung von Hitzestau
  \end{itemize}
\item
  Kühler kann unabhängig von der Lage des Motors eingebaut werden
\item
  Antriebsmoment wird nicht von der Kurbelwelle abgezweigt

  \begin{itemize}
  \item
    hierdurch haben wir eine höhere nutzbare Antriebsleistung
  \item
    Geringeren Kraftstoffverbrauch
  \item
    Geringerer Schadstoffausstoß
  \end{itemize}
\end{itemize}

\textbf{7. Was ist ein hydrostatischer Lüfterantrieb? Wo wird ein
hydrostatischer Lüfterantrieb i.d.R. eingesetzt?}

\begin{itemize}
\item
  Anwendung: Kommt in modernen Omnibusbereichen zum Einsatz (bei
  Fahrzeugen, die keinen Fahrtwind ausgesetzt sind).
\end{itemize}

Bei einem hydrostatischen Lüfterantrieb wird mittels einer vom Motor
angetrieben Hydraulikpumpe Öldruck erzeugt, welche zum Antrieb eines
Hydrokonstantmotors eingesetzt wird und an dessen Welle sich wiederum
der eigentliche Lüfter befindet. Die Regelung erfolgt in der Regel
mittels Öldrucksteuerung.
