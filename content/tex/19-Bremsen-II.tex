%ju 31-Dez-22 19-Bremsen-II.tex
\section{Grundlagen der
Fahrdynamikregelung}\label{grundlagen-der-fahrdynamikregelung}

\textbf{Kammscher Reibkreis}, die größte auf die Straße übertragbare
Kraft wird als Kreis dargestellt.

Die Kraft, die ein Rad maximal übertragen kann, setzt sich aus der
Aufstandskraft des Rades und der Haftreibung zwischen Reifen und
Fahrbahn zusammen.

\begin{itemize}
\item
  \emph{Innerhalb des Kreises} Kräfte übertragen
\item
  \emph{Außerhalb des Kreises} Schlupf
\item
  $F_\text{max} = F_N \cdot \mu_H$ (Antriebskraft, Bremskraft,
  Normalkraft, Haftreibungszahl)
\item
  $\mu = 0,1$ Eis
\item
  \textbf{stabilen Fahrzustand} Resultierende Kraft aus Umfangskraft und
  Seitenführungskraft muss innerhalb des Kreises liegen
\item
  \textbf{durchdrehende oder blockierende Räder}

  \begin{itemize}
  \item
    Umfangskraft max. und
  \item
    Seitenführungskraft 0
  \item
    Fahrzeug ist nicht mehr lenkbar
  \end{itemize}
\item
  \textbf{max. Kurvengeschwindigkeit}

  \begin{itemize}
  \item
    Seitenführungskraft max.
  \item
    Fahrzeug kann weder gebremst noch beschleunigt werden (würde
    ausbrechen)
  \end{itemize}
\end{itemize}

\textbf{Schlupf}

Das Verhältnis der Fahrgeschwindigkeit zu Radumfangsgeschwindigkeit oder
das Verhältnis der tatsächlich zurückgelegten Wegstrecke zum dynamischen
Abholumfang. (blockiert ein Rad, dann ist der Schlupf 100 \%)

\textbf{dynamischen Abrollumfang}

\begin{itemize}
\item
  Reifenaufstandsfläche = Latsch
\item
  Wenn sich das Rad dreht, verändert sich der Reifenumfang, als wenn das
  Fahrzeug steht.
\item
  Verlängerung des Weges durch Schlupf oder Verformung des Reifens.
\end{itemize}

\textbf{Typgenehmigung} für ABS nein und ESP ja

\section{ABS}\label{abs}

\textbf{ABS-Arbeitsbereich}

\textbf{stabilen Bereich}, um ein ausreichendes Maß an
Seitenführungskraft bereitstellen zu können. Zwischen 8 -- 35 \%
Schlupf.

\textbf{ABS-Aufgaben}

\begin{itemize}
\item
  Erhalt der Seitenführungskraft durch Begrenzung der maximalen
  Bremskraft
\item
  Vermeidung von Bremsplatten
\end{itemize}

\textbf{Maximale Bremskraft} Übergang von Haftreibung zur Gleitreibung

\textbf{ABS-Funktion}

\begin{itemize}
\item
  Erfassen der Raddrehzahlsignale, die miteinander verglichen werden
\item
  weicht ein Rad ab und bei kritischen Schlupfwerten
\item
  wird die Bremskraft begrenzt durch \textbf{3x Regelphasen}
\item
  Regelzyklus 4 -- 10x pro Sekunde (10 Hz)
\end{itemize}

\begin{enumerate}
\item
  \textbf{Druck halten}

  \begin{itemize}
  \item
    Einlassventil wird geschlossen und Druck an Radbremse konstant
    gehalten.
  \end{itemize}
\item
  \textbf{Druck ablassen}

  \begin{itemize}
  \item
    Auslassventil wird geöffnet und Bremsdruck wird gesandt.
  \end{itemize}
\item
  \textbf{Druck aufbau}

  \begin{itemize}
  \item
    Sinkt der Schlupf auf unkritischer Werte, öffnet das Einlassventil
    und wird mit maximaler Bremskraft verzögern.
  \end{itemize}
\end{enumerate}

\textbf{Regelungsarten des ABS}

\begin{enumerate}
\item
  \textbf{Select-Low-Regelung} (HA)

  \begin{itemize}
  \item
    Das Rad mit der geringeren Bodenhaftung bestimmt den gemeinsamen
    Bremsdruck einer Achse.
  \end{itemize}
\item
  \textbf{Individualregelung} (VA)

  \begin{itemize}
  \item
    wird jedem Rad die maximale übertragbare Bremskraft zugeteilt.
  \item
    Unterschiedliche Bremskräfte wirken.

    \begin{itemize}
    \item
      Ein Giermoment entsteht in Richtung des Rades mit der größeren
      Haftung. Kann durch Gegenlenken, ausgeglichen werden. (Vgl.
      Negativer Lenkrollradius)
    \end{itemize}
  \end{itemize}
\end{enumerate}

\textbf{Negativer Lenkrollradius} (Lenkrollhalbmesser) an Vorderachse

Erzeugt ein automatisches Gegenlenken, d.h. der Reifen schwenkt nach
innen an der Vorderachse.

\textbf{ABS-Sensoren}

\begin{itemize}
\item
  Geschwindigkeit von ca. 6 km/h notwendig
\item
  Induktiv-Sensor (Bosch: passiv)

  \begin{itemize}
  \item
    Spule und Magnetfeld $\to$ Wechselspannung
  \end{itemize}
\item
  Hall-Sensor (Bosch: aktiv)

  \begin{itemize}
  \item
    eigene Spannungsversorgung
  \item
    Kabel kann 2-adrig sein (Masseleitung = Signalleitung)
  \item
    Stromsignal an SG
  \item
    Prüfen: Strommessung
  \end{itemize}
\end{itemize}

\textbf{Regelkanäle von ABS}

\begin{enumerate}
\item
  2 Kanal ABS-System (VA und HA)
\item
  3 Kanal ABS-System (VA rechts + links und HA)
\item
  4 Kanal ABS-System (4S4M, Sensoren und Modulatoren)
\end{enumerate}

\section{ASR}\label{asr}

Erfassung des Radschlupfs an den Antriebsrädern durch Abgleichen der
Drehzahlsignale. Bei kritischen Schlupfwerten begrenzen der
Antriebskraft durch Verringern des Motordrehmoments oder Bremseingriff.

\section{FDR / ESP}\label{fdr-esp}

Fahrdynamische Prozesse erfassen und autonom regeln.

\textbf{Systeme}

\begin{enumerate}
\item
  (+) Antiblockiersystem (ABS)
\item
  (+) Elektronische Bremskraftverteilung (EBV)
\item
  (+) Antriebsschlupfregelung (ASR) mit Motorschleppmoment-Regelung
  (MSR)
\item
  (+) Giermomentregelung (GMR)
\item
  (=) Fahrdynamikregelsystem (FDR) = Elektronisches Stabilitätsprogramm
  (ESP)
\end{enumerate}

\textbf{Komponenten}

\begin{enumerate}
\item
  Hydraulikeinheit
\item
  Lenkwinkelsensor
\item
  Tandem-Hauptzylinder mit Drucksensoren
\item
  Querbeschleunigungs- und Gierratensensor
\item
  Raddrehzahlsensor
\item
  Motormanagement
\end{enumerate}

\textbf{Gierratenerfassung}

Erfassen der Drehbewegung des Fahrzeugs um seine eigene Hochachse
mithilfe der Corioliskraft (Piezo-Element wird gestaucht oder gestreckt
$\to$ SG).

\textbf{Bewegungen und Kräfte des Fahrzeugs}

\begin{enumerate}
\item
  Hochachse - Gieren (Schleudern, Drehbewegung)

  \begin{itemize}
  \item
    Radlast, Kräfte durch Fahrbahnunebenheiten
  \end{itemize}
\item
  Längsachse - Wanken (Kippbewegung)

  \begin{itemize}
  \item
    Antriebskraft, Bremskraft, Reibungskraft
  \end{itemize}
\item
  Querachse - Nicken (Drehbewegung)

  \begin{itemize}
  \item
    Fliehkraft, Seitenwindkraft, Seitenführungskraft
  \end{itemize}
\end{enumerate}
