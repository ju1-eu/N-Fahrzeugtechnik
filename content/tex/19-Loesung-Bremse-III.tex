%ju 31-Dez-22 19-Loesung-Bremse-III.tex
\textbf{1) Was versteht man unter der Bezeichnung >>Betriebsbremse<<
eines Fahrzeuges?}

Ermöglicht dem Fahrzeugführer mit abstufbarer Wirkung, die
Geschwindigkeit eines Fahrzeuges während seines Betriebs zu verringern
oder zum Stillstand zu bringen und zu halten.

\textbf{2) Welche Bremswirkung muss die Dauerbremse haben?}

Die Dauerbremse muss das Fahrzeug in einem niedrigen Gang gemäß
Herstellerangaben, in einem Gefälle von 7 \% und einer Länge von 6 km
auf 30 km/h halten können.

\textbf{3) Erklären Sie die Bezeichnung >>2--Leitungs--Bremsanlage<<.}

Bei der 2-Leitungs-Bremsanlage besteht eine Verbindung der Bremsanlage
vom Motorwagen zu der Bremsanlage des Anhängers durch zwei
Druckluftleitungen. Eine davon dient als Vorratsleitung zur Übertragung
der Vorratsluft, die andere als Bremsleitung zur Übertragung der
Steuerdrücke.

\textbf{4) Was versteht man unter dem >>Elektronischen Bremssystem<<
(EBS) bei der Druckluftbremse?}

Es ist eine Kombination aus einer 2-Kreis-Druckluftbremsanlage mit
Antiblockiersystem (ABS) und Antriebsschlupfregelung (ASR) und
zusätzlich einer elektronischen Steuerung zur schnellen Ansteuerung der
Radbremsen.

\begin{itemize}
\item
  Schnellere Ansprechzeiten
\item
  Kürzere Bremswege
\item
  Geringerer Verschleiß
\item
  verbesserte Lastzugabstimmung
\end{itemize}

\textbf{5) Welche Aufgabe hat die automatische Frostschutzpumpe?}

Die automatische Frostschutzpumpe soll bei Temperaturen um den
Gefrierpunkt und darunter pro Schaltimpuls des Druckreglers ca.
$0,6~cm^3$ Frostschutzmittel in das Leitungssystem der Druckluftanlage
einspritzen.

\textbf{6) Wie wird der Lufttrockner regeneriert?}

\textbf{Einkammer-Lufttrockner}

In der Füllphase wird ein Teil der geförderten Luft in einem
Regenerationsbehälter gespeichert. Mit dem Abschaltimpuls des
Druckregelers wird diese Luft in entgegengesetzte Richtung durch den
Lufttrockner geblasen, wodurch das Granulat getrocknet wird.

\textbf{Zweikammer-Lufttrockner}

In der Füllphase kommt nur eine Trocknerpatrone zum Einsatz. Die zweite
Patrone wird in diese Zeit durch einen Teil der getrockneten Luft aus
der ersten Patrone regeneriert. Alle 60 Sekunden wird zwischen den
beiden Trocknerpatronen umgeschaltet.

\textbf{7) Welche Aufgaben hat das >>4 -- Kreis -- Schutzventil<<?}

\begin{itemize}
\item
  Verteilung der Druckluft auf die einzelnen Bremskreise
\item
  Vorrangige Befüllung der Betriebsbremskreise
\item
  Drucksicherung in den übrigen Bremskreisen bei Druckabfall in einem
  Bremskreis
\item
  Anfahrsperre bei Druckverlust auf Bremskreis 1 (Bleed-Back)
\end{itemize}

\textbf{8) Wozu benötigt man die Kontrollstellung am Handbremsventil?}

Durch die Kontrollstellung kann der Fahrer feststellen, ob sein in
Hanglage abgestellter Lastzug auch dann noch gesichert bleibt, wenn sich
die Bremse des Anhängers infolge von Druckverlust löst.

\textbf{9) Warum ist eine Lastzugabstimmung erforderlich?}

Die Lastzugabstimmung gewährleistet, dass der ziehende Motorwagen und
der gezogene Anhänger oder Auflieger bei allen Bremsdrücken und
Lastzuständen ein etwa gleichmäßiges Bremsverhalten und einen etwa
gleichmäßigen Bremsbelagverschleiß haben.

\textbf{10) Was passiert bei Abriss der gelben Bremsleitung?}

\begin{itemize}
\item
  \textbf{Abriss der Bremsleitung} (Gelb)

  \begin{itemize}
  \item
    Zunächst bleiben die Bremsen gelöst. Beim Betätigen einer Bremse
    entweicht Luft am Anhängersteuerventil, wodurch dieses eine
    Vollbremsung im Anhänger auslöst. Nach dem Lösen der Bremse im
    Motorwagen wird auch die Anhängerbremse wieder frei.
  \end{itemize}
\end{itemize}

Bemerkung:

\textbf{Abriss der Vorratsleitung} (Rot) Anhänger muss automatisch
Bremsen
