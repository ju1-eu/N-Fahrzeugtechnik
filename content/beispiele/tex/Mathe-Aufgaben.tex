
\textbf{Aufgabe 1:} \enspace Jasmin hat ihre Freunde Nico, Laura und Anna zum Geburtstag eingeladen. Nico will nicht kommen, wenn Laura nicht kommt. Laura und Anna kommen beide oder kommen beide nicht. Aber Anna sagt: >>Wenn Nico und Laura beide nicht kommen, dann komme ich.<< Wer von den dreien wird unter diesen Bedingungen tatsächlich zum Geburtstag erscheinen?


\textbf{Aufgabe 2:} \enspace In der Anwaltsserie >>Suits<< (Staffel 4, Folge 1) kommt es zu Folgendem Gespräch zwischen Mike und seiner Sekräterin Amy.


\begin{enumerate}[label={\protect\ding{\value*}},start=192]
    \item Amy: >>Und wie lief dein Treffen mit dem geheimnisvollen Harvey Specter?<<
    \item Mike: >>Ein Arsch zu sein macht einen nicht geheimnisvoll.<<
    \item Amy: >>Na dann bist du ja ein ganz offenes Buch.<<
\end{enumerate}

Untersuchen Sie das Gespräch aussagenlogisch und prüfen Sie den Wahrheitswert von Amys Aussage.


\textbf{Aufgabe 3:} \enspace Formalisieren Sie die folgenden Aussagen und verneinen Sie sie anschließend (ohne das Wort nicht davor zu setzen) und übersetzen Sie wieder in Umgangssprache:

\begin{enumerate}[label=(\alph*)]
    \item Volksmund: >>Bei Nacht sind alle Katzen grau.<<
    \item Plakatwerbung: >>Wenn einer hochguckt, dann gucken alle.<<
    \item Gorbatschov: >>Wer zu spät kommt, den bestraft das Leben.<<
\end{enumerate}

\textbf{Aufgabe 4:} \enspace Wurzel

\begin{multicols}{3}
    \begin{enumerate}[label=(\alph*)]
        \item $\sqrt{169}$
        \item $\sqrt{0,36}$
        \item $\frac{\sqrt{45}}{\sqrt{80}}$
        \item $\sqrt{32}$
        \item $\sqrt{2}$
        \item $\sqrt{1,44}$
        \item $\sqrt{\frac{75}{12}}$
    \end{enumerate}
\end{multicols}